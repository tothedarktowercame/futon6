\chapter*{Introduction: Making of the First-Proof Cycle}
\addcontentsline{toc}{chapter}{Introduction: Making of the First-Proof Cycle}

This monograph documents both mathematical outputs and process outputs from a
compressed multi-agent proof cycle in \texttt{futon6}. The work started as
infrastructure (knowledge ingestion, tagging, and wiring-diagram metatheory),
then pivoted into rapid proof generation, adversarial review, and iterative
repair.

\section*{Snapshot}
\begin{itemize}[leftmargin=1.5em]
\item Initial proof-attempt burst: 10 problem writeups in under two hours.
\item Total development trace (through the retrospective window): 89 commits in roughly one day of active proof sprinting, plus prior infrastructure work.
\item Final project-level calibration from the making-of record: transition from optimistic ``all solved'' framing to explicit partition into proved / conditional / partial items.
\item Methodological outcome: wiring-diagram-based verification outperformed plain narrative proof drafting for gap localization and repair.
\end{itemize}

\section*{What Was New in the Process}
\begin{enumerate}[leftmargin=1.5em]
\item \textbf{Proof structure as graph}: logical dependencies represented as typed edges (assert/reference/derive), so critiques could target claims precisely by node and edge type.
\item \textbf{Generated verifier prompts}: Codex checks were dispatched per proof node, with local predecessor/successor context.
\item \textbf{Adversarial review loops}: repeated critic--responder cycles exposed confidence laundering and citation mismatches.
\item \textbf{Numerical honesty checks}: where claims were computationally testable (especially Problem 4), stress tests constrained overclaiming.
\item \textbf{Research handoff artifacts}: unresolved branches were converted into explicit gap memos and next-step query plans instead of being silently deferred.
\end{enumerate}

\section*{Interpretive Theme}
The making-of notes identify an asymmetry: generation is fast; trustworthy
validation is expensive. The practical leverage came from structured
verification infrastructure, not from single-pass drafting quality.

In that sense, this document should be read as a dual artifact:
\begin{itemize}[leftmargin=1.5em]
\item \emph{mathematical}: a set of problem writeups with varying completion status;
\item \emph{methodological}: a trace of how multi-agent proof systems stabilize claims under adversarial revision.
\end{itemize}

\section*{Current Calibration for This Manuscript}
\begin{itemize}[leftmargin=1.5em]
\item Problem with explicit remaining obligation (noted prominently in this volume): Problem 6.
\item Problem 7 is treated here as closed via the updated rotation-route surgery chain in \texttt{problem7-solution.md}, with supporting theorem-number context tracked in \texttt{problem7-g2-theorem-chain.md}.
\item Problem 4's higher-\(n\) branch is recorded as computationally closed in the repository's later postscript narrative, but still treated with explicit certification-context notes where appropriate.
\end{itemize}

\section*{Reading Order}
\begin{enumerate}[leftmargin=1.5em]
\item \textbf{Part I} for concise, citation-heavy orientation and status framing.
\item \textbf{Part II} for full draft arguments and technical detail.
\item Return to Problem 6 status notes after reading its full chapter; this remains the primary proof-hardening target in this version.
\end{enumerate}
