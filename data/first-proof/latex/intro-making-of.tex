\chapter*{Introduction: The Making of the First-Proof Cycle}
\addcontentsline{toc}{chapter}{Introduction: The Making of the First-Proof Cycle}

This monograph records both the mathematical results and the working process
behind a compressed, multi-agent proof cycle in \texttt{futon6}. The project
began as infrastructure---knowledge ingestion, tagging, and wiring-diagram
metatheory---and then pivoted into a rapid proof sprint with iterative
critique and repair.

\section*{Snapshot}
\begin{itemize}[leftmargin=1.5em]
\item Ten initial proof attempts drafted in under two hours.
\item Eighty-nine commits during roughly one day of active sprinting, in addition to earlier infrastructure development.
\item A shift in framing over the course of the sprint: from an early ``all solved'' posture to a clearer partition into proved, conditional, and partial results.
\item A methodological lesson: wiring-diagram--based verification localized gaps and inconsistencies more effectively than plain narrative drafting.
\end{itemize}

\section*{What Changed in This Process}
\begin{enumerate}[leftmargin=1.5em]
\item \textbf{Proofs as graphs.} Logical dependencies were represented explicitly as typed edges (assert / reference / derive), allowing critiques to target specific claims rather than whole arguments.
\item \textbf{Node-level verification.} Codex checks were issued per proof node, with local predecessor and successor context, rather than against entire drafts.
\item \textbf{Adversarial loops.} Repeated critic--responder cycles surfaced overconfidence, weak citations, and unstated assumptions.
\item \textbf{Numerical stress tests.} Where claims admitted computational checking (notably in Problem 4), targeted stress tests constrained overstatement.
\item \textbf{Explicit research handoffs.} Open branches were turned into gap memos and concrete next-step queries instead of being silently deferred.
\end{enumerate}

\section*{Interpretive Thread}
The working notes reveal a consistent asymmetry: generation is fast;
trustworthy validation is slow.

The real leverage came not from first-pass drafting quality, but from
structured verification---tools that made weaknesses visible early and locally.

This volume should therefore be read in two ways:
\begin{itemize}[leftmargin=1.5em]
\item \emph{Mathematical artifact}: a set of problem writeups at varying stages of completion.
\item \emph{Methodological artifact}: a record of how multi-agent proof systems stabilize claims under structured adversarial revision.
\end{itemize}

\section*{Current Status}
\begin{itemize}[leftmargin=1.5em]
\item Problem 6 remains open, with a clearly stated outstanding obligation.
\item Problem 7 is treated as closed via the updated rotation-route surgery chain (see \texttt{problem7-solution.md} and supporting theorem context in \texttt{problem7-g2-theorem-chain.md}).
\item Problem 4's higher-\(n\) branch has computational closure artifacts in the repository postscript, while analytic framing remains explicitly documented in the chapter-level notes.
\end{itemize}

\section*{Suggested Reading Path}
\begin{enumerate}[leftmargin=1.5em]
\item Begin with \textbf{Part I} for orientation and status framing.
\item Continue to \textbf{Part II} for full arguments and technical detail.
\item Use \textbf{Part III} as a provenance index for prompt-level pivot points and status changes.
\item Revisit the Problem 6 notes after reading its chapter; it remains the primary proof-hardening target in this version.
\end{enumerate}
