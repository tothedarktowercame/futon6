\chapter*{Introduction: The Making of the First-Proof Cycle}
\addcontentsline{toc}{chapter}{Introduction: The Making of the First-Proof Cycle}

This monograph records both the mathematical results and the working process
behind a compressed, multi-agent proof cycle in \texttt{futon6}.\footnote{Source repository: \url{https://github.com/tothedarktowercame/futon6}} The project
began as infrastructure---knowledge ingestion, tagging, and wiring-diagram
metatheory---and then pivoted into a rapid proof sprint with iterative
critique and repair.

The working process turned out to resemble Lakatos's \emph{Proofs and
Refutations}~\cite{lakatos1976} more than we expected.  In Lakatos's
dialogue, a conjecture is proposed, counterexamples surface, the conjecture
is patched, and the cycle repeats---with each iteration sharpening both the
theorem and the participants' understanding of the domain.  Our sprint
followed the same rhythm, except that the interlocutors were a human
dispatcher, multiple Claude instances (provers), and Codex (critic and
research assistant).  Conjectures were drafted quickly; adversarial review
surfaced overclaims and gaps; targeted repair narrowed the open obligations.
The wiring-diagram decomposition---which represents each proof as a typed
graph of claims and dependencies---served as the shared epistemic artefact
that Lakatos's blackboard served in his fictional classroom.

The title alludes to the sprint format: ten open problems attempted in
roughly 48~hours, with the process documented alongside the results.
The Lakatosian connection runs through Pease et
al.~\cite{pease2017}, which formalised Lakatos's informal logic of
mathematical discovery as a dialogue game over argumentation structures,
implemented in a system capable of mixed-initiative human--AI collaboration.
That paper asked whether computers could participate in mathematical
reasoning the way humans do---through conjecture, critique, and reform.
The present work is, in a sense, a large-scale empirical answer: ten
problems attempted by AI agents under human supervision, with the
Lakatosian cycle running at machine speed.

This connection is not accidental.  Corneli et al.~\cite{corneli2017}
developed a computational model of mathematical question-and-answer dialogues
grounded in Lakatos's framework, using Inference Anchoring Theory + Content
(IATC) to annotate MathOverflow threads with performative types
(assert, challenge, reform, clarify, etc.).  The IATC vocabulary reappears in
this project's Stage~7 thread-wiring pipeline (see Part~III, Act~VI), where
StackExchange threads are parsed into wiring diagrams whose edges carry the
same performative labels.  The earlier work on peer-produced mathematical
knowledge~\cite{corneli2013} established the empirical base: PlanetMath as a
community-maintained proof ecosystem, studied through the lens of paragogy
(peer-produced peer learning).  The present sprint can be read as a
machine-accelerated instance of the same phenomenon: collaborative,
incremental proof construction with explicit argumentation structure.

\section*{Snapshot}
\begin{itemize}[leftmargin=1.5em]
\item Ten initial proof attempts drafted in under two hours.
\item Over two hundred commits across roughly 48~hours of active sprinting (February 11--13, 2026), in addition to earlier infrastructure development.
\item A shift in framing over the course of the sprint: from an early ``all solved'' posture to a clearer partition into proved, conditional, and partial results.
\item A methodological lesson: wiring-diagram--based verification localized gaps and inconsistencies more effectively than plain narrative drafting.
\end{itemize}

\section*{What Changed in This Process}
\begin{enumerate}[leftmargin=1.5em]
\item \textbf{Proofs as graphs.} Logical dependencies were represented explicitly as typed edges (assert / reference / derive), allowing critiques to target specific claims rather than whole arguments.
\item \textbf{Node-level verification.} Codex checks were issued per proof node, with local predecessor and successor context, rather than against entire drafts.
\item \textbf{Adversarial loops.} Repeated critic--responder cycles surfaced overconfidence, weak citations, and unstated assumptions.
\item \textbf{Numerical stress tests.} Where claims admitted computational checking (notably in Problem 4), targeted stress tests constrained overstatement.
\item \textbf{Explicit research handoffs.} Open branches were turned into gap memos and concrete next-step queries instead of being silently deferred.
\end{enumerate}

\section*{Interpretive Thread}
The working notes reveal a consistent asymmetry: generation is fast;
trustworthy validation is slow.  This is the same asymmetry Lakatos
identified in his classroom: conjectures are cheap, but understanding
\emph{why} a conjecture fails (or holds) is the real mathematical work.

The real leverage came not from first-pass drafting quality, but from
structured verification---tools that made weaknesses visible early and locally.
In Lakatosian terms, the wiring diagrams served as \emph{proof analyses}:
decompositions of the argument into lemmas and sub-conjectures, each of
which could be independently attacked or defended.

This volume should therefore be read in three ways:
\begin{itemize}[leftmargin=1.5em]
\item \emph{Mathematical artifact}: a set of problem writeups at varying stages of completion.
\item \emph{Methodological artifact}: a record of how multi-agent proof systems stabilize claims under structured adversarial revision.
\item \emph{Continuation of a research programme}: the computational
  argumentation models of~\cite{corneli2017} and the peer-production
  ethnography of~\cite{corneli2013}, now instantiated with AI agents
  operating at sprint pace rather than community pace.
\end{itemize}

\section*{Terminology}
\begin{description}[leftmargin=1.5em,style=nextline]
\item[AIF] Active Inference Framework---used here as a conceptual lens for agent roles, intervention points, and coordination state; a full AIF policy loop for proof search is proposed future work, not yet implemented in this sprint.
\item[BSS] Batson--Spielman--Srivastava spectral sparsification framework.
\item[IATC] Inference Anchoring Theory + Content---an annotation scheme for mathematical dialogues used here to type dialogue moves and content links within the AIF process model~\cite{corneli2017}.
\item[MSS] Marcus--Spielman--Srivastava interlacing families framework.
\item[PSD] Positive semi-definite (for matrices).
\end{description}

\section*{Current Status}
\begin{itemize}[leftmargin=1.5em]
\item Problems 1, 2, 7, 8, and 9 now have node-level validation artifacts (\texttt{problemN-codex-results.jsonl}); each currently remains in \emph{partial validation} due to unresolved node-level gaps flagged by the verifier.
\item Problem 3 now has a validation-complete existence path under the scoped criterion used here (current run: 2 verified, 7 plausible, 0 gaps). Uniqueness/irreducibility is treated as optional and out of scope for the core existence claim.
\item Problem 10 is conditional under explicitly stated assumptions, with explicit necessity counterexamples for assumption violations.
\item Problem 5 is solved in a scope-limited form for $F_O$-local connective spectra.
\item Problem 6 is partial: the $K_n$ case is proved ($c_0 = 1/3$), with one remaining technical gap in the general-graph argument (formal $\bar{d} < 1$ bound at $M_t \ne 0$).
\item Problem 4 is partial: proved analytically for $n \le 3$; the $n=4$ branch has strong computational support with one certification step still pending; higher $n$ remains open.
\item Cross-problem closure-vs-validation inventory: \texttt{data/first-proof/closure-validation-audit.md}.
\end{itemize}

\section*{Suggested Reading Path}
\begin{enumerate}[leftmargin=1.5em]
\item Begin with \textbf{Part I} for orientation and status framing.
\item Continue to \textbf{Part II} for full arguments and technical detail.
\item Use \textbf{Part III} as a provenance index for prompt-level pivot points and status changes.
\item Read \textbf{Part IV} for cross-problem strategy patterns and the coaching breakthrough on Problem~6.
\item Revisit the Problem 6 notes after reading its chapter; the remaining gap is narrow and well-characterised.
\end{enumerate}
