\section{Problem 5: $\mathcal{O}$-Slice Connectivity via Geometric Fixed Points}

\subsection*{Problem statement}
For an incomplete transfer system from an $N_\infty$ operad, define the associated restricted regular slice filtration and characterize $\mathcal{O}$-slice connectivity using geometric fixed points.

\subsection*{Answer}
For $\mathcal{F}_{\mathcal O}$-local connective $G$-spectra, one can restrict the Hill--Yarnall connectivity criterion to subgroups in the admissible family $\mathcal{F}_{\mathcal O}$.

Solved in a scope-limited form: the subgroup-family-level characterization is established for $\mathcal{F}_{\mathcal O}$-local connective spectra; a full indexing-system-level extension is left open.

\phantomsection\label{sn:p5}
\statusnote{Closed only in the stated local scope
\((\mathcal{F}_{\mathcal O}\text{-local connective spectra})\). The
indexing-system-level extension remains an explicit open obligation.}

\gapnote{Open obligation (scope extension): move from subgroup-family indexing
to full Blumberg--Hill indexing-system data (admissible finite
\(H\)-sets, not only admissible subgroups). The current result uses
\(\mathcal{F}_{\mathcal O}=\{H:e\to H\ \text{admissible}\}\) and does not yet
encode finer \(K\to H\) admissibility for \(K\neq e\). Closing this gap
requires a full indexing-system formulation of \(\mathcal O\)-slice
connectivity and, in parallel, a non-local replacement (or discharge) of the
\(\mathcal{F}_{\mathcal O}\)-locality hypothesis in the reverse direction.}

\subsection*{Annotated proof sketch}
\begin{enumerate}[leftmargin=1.5em]
\item Build $\tau^{\mathcal O}_{\ge n}$ from regular slice cells indexed only by admissible subgroups.
\item Use closure properties of indexing systems to verify that this restriction is stable under conjugation and subgroup passage.
\item Adapt Hill--Yarnall's geometric fixed-point criterion: for $\mathcal{F}_{\mathcal O}$-local spectra, no tests are needed outside $\mathcal{F}_{\mathcal O}$ because those geometric fixed points vanish by locality.
\end{enumerate}

\subsection*{Selected citations}
\begin{itemize}[leftmargin=1.5em]
\item Hill--Yarnall (slice filtration reformulation): \url{https://arxiv.org/abs/1703.10526}
\item Blumberg--Hill ($N_\infty$ operads and indexing systems): \url{https://arxiv.org/abs/1309.1750}
\item HHR monograph context: \url{https://bookstore.ams.org/surv-346}
\item Geometric fixed points background: \url{https://ncatlab.org/nlab/show/geometric+fixed+points}
\end{itemize}

\subsection*{Background and prerequisites}
\textbf{Field:} Equivariant Stable Homotopy Theory.\\
\textbf{What you need:} Genuine $G$-spectra, geometric fixed-point functors, the slice filtration of Hill--Hopkins--Ravenel, and $N_\infty$ operads / indexing systems of Blumberg--Hill. Research-level; assumes graduate coursework in algebraic topology and stable homotopy theory.\\
\textbf{Way in:} Schwede, \emph{Global Homotopy Theory} (Cambridge, 2018) for equivariant foundations; the Hill--Hopkins--Ravenel Kervaire paper (Annals, 2016) for slice-filtration motivation.


\clearpage
