\section{Problem 4: Root Separation Energy Under Finite Free Convolution ($n=4$)}

\subsection*{Problem statement}
For monic real-rooted degree-$n$ polynomials, test
\[
\frac{1}{\Phi_n(p \boxplus_n q)} \ge \frac{1}{\Phi_n(p)} + \frac{1}{\Phi_n(q)}.
\]

\subsection*{Answer}
Partial result: proved analytically for $n \le 3$, and now proved for $n=4$
via the Path 2 algebraic certificate (3-piece Cauchy--Schwarz with
P-convexity of $K_{\mathrm{red}}$); higher $n$ remains open.

\phantomsection\label{sn:p4}
\statusnote{Problem 4 is proved for $n \le 3$ analytically (Cauchy--Schwarz),
and proved for $n=4$ via the Path 2 algebraic certificate (3-piece
Cauchy--Schwarz + P-convexity of $K_{\mathrm{red}}$). Higher $n$ remains open.}

\subsection*{Annotated proof structure}
\begin{enumerate}[leftmargin=1.5em]
\item Algebraic reduction at $n=4$: normalize coefficients, derive a polynomial numerator/denominator formulation for the surplus using the exact $\Phi_4\cdot\mathrm{disc}$ identity\footnote{Discriminant refresher: \url{https://planetmath.org/discriminantofapolynomial}.}.
\item Show denominator sign on the real-rooted domain, reducing inequality to nonnegativity of a single polynomial $-N$.
\item Boundary and symmetric-subcase analysis by exact elimination/resultant methods.
\item Path 2 closure for $n=4$: prove $T_2+R$ surplus nonnegative via
P-convexity/boundary analysis of $K_{\mathrm{red}}$, which supersedes the
case-by-case critical-point accounting branch.
\end{enumerate}

\subsection*{Selected citations}
\begin{itemize}[leftmargin=1.5em]
\item Finite free convolution (MSS): \url{https://arxiv.org/abs/1504.00350}
\item Interlacing families follow-up: \url{https://arxiv.org/abs/1507.05020}
\item Finite free cumulants perspective: \url{https://arxiv.org/abs/1707.02443}
\item MathOverflow discussion (finite free convolution): \url{https://mathoverflow.net/questions/375273/finite-free-convolution-of-polynomials}
\item PHCpack project page: \url{https://github.com/janverschelde/PHCpack}
\end{itemize}

\subsection*{Background and prerequisites}
\textbf{Field:} Free Probability / Real Algebraic Geometry / Polynomial Inequalities.\\
\textbf{What you need:} Real-rooted polynomials and discriminants, the finite free convolution $\boxplus_n$ of Marcus--Spielman--Srivastava, and basic free probability (free cumulants). Computational algebra tools (e.g.\ resultants, homotopy continuation) appear in the $n=4$ certification.\\
\textbf{Way in:} Marcus--Spielman--Srivastava, ``Interlacing Families I'' (arXiv:1304.4132) for finite free convolution; Nica--Speicher, \emph{Lectures on the Combinatorics of Free Probability} (Cambridge, 2006) for the probabilistic backdrop.


\clearpage
