\section{Problem 6: $\varepsilon$-Light Vertex Subsets of Graphs}

\subsection*{Problem statement}
Find a universal constant $c_0>0$ such that every weighted graph and every $\varepsilon\in(0,1)$ admit a subset $S$ with
\[
|S|\ge c_0\varepsilon n,\qquad L_{G[S]}\preceq \varepsilon L_G.
\]

\subsection*{Current status}
Partial (general case conditional). The $K_n$ case is proved with $c_0=1/3$;
for general graphs, the remaining issue is the Barrier Maintenance Invariant
(BMI): prove the full barrier-degree bound $\bar d_t<1$ when $M_t\neq 0$.
The current E/F regime reduction isolates this to two explicit lemmas.

\subsection*{Annotated proof structure}
\begin{enumerate}[leftmargin=1.5em]
\item Use leverage scores $\tau_e=w_eR_{\mathrm{eff}}(u,v)$ to define the heavy-edge graph and show any $\varepsilon$-light set must be independent there.
\item Apply the leverage-score trace identity and Turan's theorem to extract a large candidate set.
\item Resolve easy branches (independent-in-$G$ and directly bounded spectral sum) unconditionally.
\item Reduce the hard branch to barrier maintenance (BMI), then split by a
graph-adaptive E/F regime decomposition; substantial computational evidence is
available, but the final general proof still requires two regime lemmas.
\end{enumerate}

\gapnote{The unresolved step is the full BMI bound in the $M_t\neq 0$ regime.
The writeup reduces this to two explicit E/F regime lemmas. Until those lemmas
are proved (or replaced by an equivalent closure argument), the
universal-constant theorem remains conditional.}

\subsection*{Background and prerequisites}
\textbf{Field:} Spectral Graph Theory / Linear Algebra.\\
\textbf{What you need:} Graph Laplacians and their spectra, effective resistance and leverage scores, Loewner ordering of positive semidefinite matrices, and matrix concentration inequalities. Tur\'an's theorem from extremal combinatorics appears in the combinatorial step.\\
\textbf{Way in:} Spielman, ``Spectral and Algebraic Graph Theory'' (lecture notes, freely available); Tropp, ``An Introduction to Matrix Concentration Inequalities'' (Foundations and Trends, 2015).

\subsection*{Selected citations}
\begin{itemize}[leftmargin=1.5em]
\item Graph sparsification by effective resistances: \url{https://arxiv.org/abs/0803.0929}
\item Interlacing families II (Kadison--Singer machinery): \url{https://arxiv.org/abs/1306.3969}
\item Twice-Ramanujan sparsifiers: \url{https://arxiv.org/abs/0901.2698}
\item Laplacian matrix definition: \url{https://planetmath.org/laplacianmatrix}
\end{itemize}

\clearpage
