\section{Problem 7: Uniform Lattice with 2-Torsion and Rationally Acyclic Universal Cover}

\subsection*{Problem statement}
Can a cocompact lattice $\Gamma$ in a real semisimple Lie group, containing an element of order $2$, be the fundamental group of a closed manifold whose universal cover is $\mathbb Q$-acyclic?

\subsection*{Current status}
Provisionally closed via the rotation route (\(n=7\), congruence setting):
E2 is discharged and the S-branch is supported by the codim-2 surgery theorem
chain documented in the project theorem-number ledger,\footnote{\url{https://github.com/tothedarktowercame/futon6/blob/master/data/first-proof/problem7-g2-theorem-chain.md}} with final
confidence pending independent line-by-line ledger re-check.

\subsection*{Annotated proof structure}
\begin{enumerate}[leftmargin=1.5em]
\item Use Fowler's fixed-point Euler-characteristic criterion to produce groups in $\mathrm{FH}(\mathbb Q)$ from suitable $\mathbb Z/2$-extensions\footnote{Fowler: \url{https://arxiv.org/abs/1204.4667}.}.
\item Construct two arithmetic lattice routes:
  \begin{itemize}[leftmargin=1.5em]
  \item reflection route in even dimension;
  \item codimension-2 rotation route in odd dimension (arithmetically cleaner for surgery).
  \end{itemize}
\item Follow the rotation-route S-branch: codimension-2 setup, obstruction chain, and cap/gluing output to a closed manifold model with rationally acyclic universal cover.
\end{enumerate}

\phantomsection\label{sn:p7}
\statusnote{This monograph treats Problem 7 as provisionally closed via the
rotation route. The construction uses Fowler's \(\mathrm{FH}(\mathbb Q)\)
criterion and codimension-2 equivariant surgery in the \(n=7\)
congruence-lattice setting, with the obstruction-vanishing bridge tracked in
the theorem-number ledger.\footnote{\url{https://github.com/tothedarktowercame/futon6/blob/master/data/first-proof/problem7-g2-theorem-chain.md}} Final confidence is
contingent on independent ledger re-check. Literature anchors:
\url{https://arxiv.org/abs/1204.4667}, \url{https://arxiv.org/abs/1705.10909}.}

\subsection*{Background and prerequisites}
\textbf{Field:} Surgery Theory / Geometric Topology / Arithmetic Groups.\\
\textbf{What you need:} Algebraic topology (homology, fundamental group, covering spaces), surgery theory on manifolds (normal maps, surgery obstructions), and arithmetic lattices in semisimple Lie groups. Research-level; assumes graduate coursework in both topology and Lie theory.\\
\textbf{Way in:} L\"uck, \emph{A Basic Introduction to Surgery Theory} (ICTP lecture notes, 2002) for the topological machinery; Raghunathan, \emph{Discrete Subgroups of Lie Groups} (Springer, 1972) for the arithmetic side.

\subsection*{Selected citations}
\begin{itemize}[leftmargin=1.5em]
\item Fowler criterion: \url{https://arxiv.org/abs/1204.4667}
\item Reflection lattices source used in writeup: \url{https://arxiv.org/abs/2506.23994}
\item Equivariant surgery framework: \url{https://arxiv.org/abs/1705.10909}
\item Davis--Luck obstruction context: \url{https://arxiv.org/abs/2303.15765}
\item G2 theorem-number ledger (project note).\footnote{\url{https://github.com/tothedarktowercame/futon6/blob/master/data/first-proof/problem7-g2-theorem-chain.md}}
\item Arithmetic lattices (Borel--Harish-Chandra): \url{https://www.jstor.org/stable/1970362}
\item Semisimple Lie group definition: \url{https://planetmath.org/semisimpleliegroup}
\item Related MathOverflow discussion: \url{https://mathoverflow.net/questions/33545/equivariant-surgery-problem}
\end{itemize}

\clearpage
