\chapter*{Part III: Prompt Excerpts and Pivot Moments}
\addcontentsline{toc}{chapter}{Part III: Prompt Excerpts and Pivot Moments}

This section records short source excerpts that triggered major proof-state
changes during the sprint. The goal is provenance: each excerpt points to a
specific prompt or exchange that shifted a proof route, status label, or gap
decomposition.

\section*{P4: Jensen Gap Became the Central Obstruction}
\textbf{Source:} \texttt{data/first-proof/problem4-stam-codex-prompts.jsonl:4}
(\texttt{step-4-jensen-gap})

\begin{quote}\small\ttfamily
Address the fundamental obstacle: the inequality we want is about
1/Phi\_n of the EXPECTED POLYNOMIAL, not the EXPECTED 1/Phi\_n.
...
These differ by a Jensen-type gap.
\end{quote}

\textbf{Pivot:} this crystallized why Dyson/Ito style averaging was not enough
by itself and forced explicit control of the expectation-vs-functional order.

\section*{P6: The Conclusion Was Explicitly Reframed as Conditional}
\textbf{Sources:}
\texttt{data/first-proof/problem6-codex-prompts.jsonl:10}
(\texttt{p6-s5}) and
\texttt{data/first-proof/problem6-codex-prompts.jsonl:11}
(\texttt{p6-s6})

\begin{quote}\small\ttfamily
General-graph step is an external dependency in this draft...
this writeup does not rederive that theorem.
...
conditional on the external general theorem (p6-s5),
the full existential answer is yes.
\end{quote}

\textbf{Pivot:} this converted an overstrong global claim into a layered claim:
local framework proved in-text, universal existence deferred to a named
external theorem dependency.

\section*{P7: Dispatch-Level Decomposition of the Remaining Gap}
\textbf{Source:}
\texttt{data/first-proof/problem7-student-dispatch.md:60}

\begin{quote}\small\ttfamily
Step B: integral (torsion) obstruction.
...
This step has not been formalized.
\end{quote}

\textbf{Pivot:} the Problem 7 program was split into Step A (rational
obstruction) and Step B (integral torsion handling), which sharpened review
targets and reduced cross-talk between independent subproblems.

\section*{P7: Gap Triage to G1/G2/G3 with a Candidate Closure Pattern}
\textbf{Source:} \texttt{\string~/.codex/history.jsonl:9224}

\begin{quote}\small\ttfamily
G1 (pi\_1 control) and G3 (rational acyclicity) can be closed once
the cap is homotopy equivalent to F x D2 rel boundary.
...
G2 (Browder-Quinn identification) remains the genuine remaining gap.
\end{quote}

\textbf{Pivot:} this changed the geometry-to-surgery bridge from a monolithic
``open Problem 7'' label to a prioritized closure queue, with G2 isolated as
the main unresolved theorem-chain obligation.

\section*{P7: Non-Fiction Check for the Codim-2 Theorem Chain}
\textbf{Source:}
\texttt{data/first-proof/problem7-g2-theorem-chain.md:11}

\begin{quote}\small\ttfamily
There is no single modern arXiv theorem that states the exact P7 sentence
in one line. The bridge is assembled from standard components.
\end{quote}

\textbf{Pivot:} this stabilized citation strategy by replacing ``single theorem
lookup'' with a theorem-number ledger across Browder-Quinn, Lopez de Medrano,
Cappell-Shaneson, and Ranicki.

\section*{Meta: Narrative and Status Claims Were Prompted for Revision}
\textbf{Sources:}
\texttt{\string~/.codex/history.jsonl:9256} and
\texttt{\string~/.codex/history.jsonl:9261}

\begin{quote}\small\ttfamily
... summary claims about what is closed and open needs to be checked.
...
add a Part III with excerpted prompts that support key pivotal moments.
\end{quote}

\textbf{Pivot:} these prompts directly caused the monograph-level edits to
status language and provenance structure.
