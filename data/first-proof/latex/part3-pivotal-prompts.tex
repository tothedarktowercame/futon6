\chapter*{Part III: The Sprint as Dialogue}
\addcontentsline{toc}{chapter}{Part III: The Sprint as Dialogue}

This chapter records the proof sprint as it actually happened: a
conversation between a human dispatcher (Joe), two AI proof engines
(Claude instances), and a verifier/researcher (Codex), conducted over
roughly 48 hours on February 11--13, 2026. Excerpts are drawn from
\texttt{history.jsonl} (Joe's prompts), Codex output files, and git
commit messages. Editorial annotations mark the pivotal moments where
proof strategy, status, or architecture shifted.

\bigskip

%% ============================================================
\section*{Act I: ``Even if we're a day late''}
%% ============================================================

\begin{joebox}
Claude + Joe Solve ``First Proof'' using Rob's Superpod
\url{https://arxiv.org/html/2602.05192v1}
\end{joebox}

\begin{joebox}
Even if we're a day late and a dollar short at least we can try
\end{joebox}

\begin{joebox}
Let's see if we can do one of them right now using PlanetMath and nlab
and off-the-shelf Claude
\end{joebox}

\pivot{The sprint begins. Problem~10 is chosen as a warm-up.
  Within 30~minutes it is solved, and the pace accelerates.}

\begin{joebox}
OK, should we solve another First Proof problem? The last one wasn't
so hard apparently!
\end{joebox}

%% ============================================================
\section*{Act II: Scaling the Machinery}
%% ============================================================

\begin{joebox}
OK, I have two Codex research assistants working for you as well as
the validator working on p6
\end{joebox}

\begin{joebox}
OK, we have 3~verifiers running now, they are making progress but not
committing yet
\end{joebox}

\begin{joebox}
Codex is \emph{still} after 53~minutes hard at work on Problem~6, but
it is down to two gaps now. I think 3~verifiers is about all my
laptop can handle, noticing considerable slow down.
\end{joebox}

\begin{joebox}
I think I will give it five more minutes and then say ``pencils down''
and let you look at \#6
\end{joebox}

\begin{joebox}
Codex is pushing \#6, I OOMed on 3 and 5, I will get them back up
once 6~lands
\end{joebox}

\pivot{Physical resource limits (RAM, thermal throttling) become a
  genuine constraint on mathematical exploration. The multi-agent
  topology stabilises at three concurrent processes.}

\begin{joebox}
It's slow going over here---are any of the proofs sufficiently
validated that we could start a ``write up''? If the previous
example is representative it can take about an hour for one of
these to complete.
\end{joebox}

%% ============================================================
\section*{Act III: Proofs and Refutations}
%% ============================================================

\begin{joebox}
I have engaged Codex and another Claude in a reviewer +
response-to-reviewer cycle. Codex found some pretty significant
problems; the other Claude is making what look like solid fixes, but
it looks like tough going.
\end{joebox}

\begin{codexbox}[title={Codex --- Triage verdict}]
P8: most recoverable. [\ldots]
P2: recoverable but harder. Needs one serious
representation-theoretic theorem/lemma that is currently hand-waved.
[\ldots]
P7: least recoverable as currently framed. The obstruction-vanishing
step is the main risk; unless they replace it with a precise known
theorem tailored to this lattice family, this will likely remain
conditional/existential rather than a full proof.
\end{codexbox}

\pivot{Codex's triage creates a severity ordering across problems,
  distinguishing ``recoverable'' from ``structural'' gaps.
  The review cycle is now explicitly adversarial: prover vs.\ critic.}

\begin{joebox}
This is interesting in comparison to my work in futon5 on
Meta-Meta-Cellular Automata. There also, generation is cheap---just
bit flipping---but doing things like ``detecting edge of chaos
behaviour'' is a challenge. [\ldots]
However, I think we really haven't done so bad.
We have our Claude-Codex ``Proofs and Refutations'' system up and
running right now, we use both collaborative (verification + library
research) and argumentative (prover vs critic) modes. So
notwithstanding the mathematical issues you raised---at least we
aren't shipping failed proofs! This is verification happening at a
``glacial'' scale in futon3c terminology---but at least it is
happening!
\end{joebox}

\begin{joebox}
Furthermore, the wiring diagram decomposition is probably the only
thing holding this together---it's allowing us to be very targeted
with our reviews and fixes. Imagine trying to do what we're doing
with\ldots\ ChatGPT. It would be hell.
\end{joebox}

\begin{joebox}
Also, another remark---making mistakes isn't a bad thing.
As my grandfather said, that's how we learn.
\end{joebox}

\pivot{A methodological self-assessment. The two key claims:
  (1)~the wiring-diagram decomposition enables targeted review at
  scale; (2)~the collaborative/adversarial agent modes prevent
  ``shipping failed proofs.'' Both claims are tested in
  Acts~IV--V.}

%% ============================================================
\section*{Act IV: Formalization and Futon Theory}
%% ============================================================

\begin{joebox}
This does suggest a possible but probably crazy and stupid idea---which
hasn't stopped hundreds of people trying it. What about formalizing as
we go? Like QED style. I mean, it hurts me to say this. I briefly
had Bob Boyer as a Master's advisor. He and I could not see eye to
eye. Granted, that was 2002 and things were a little different back
then.
\end{joebox}

\begin{joebox}
Yeah, I mean, we needed in 2002 \emph{what we are doing now}. The
only problem is it wasn't possible then.
\end{joebox}

\begin{claudebox}
So in a sense, the 2002 vision was twenty-four years early rather
than wrong.
\end{claudebox}

\begin{joebox}
Better than twenty-four years late! Imagine if I proposed WIFI now.
Great idea right?
\end{joebox}

\pivot{A historical aside connecting the sprint to the 2002-era
  theorem-proving community. The implication: AI-assisted informal
  proof is the approach that was missing, not formalization per~se.}

\begin{joebox}
OK, so the naive idea: could we write a letter to Codex talking about
why Problem~4 is hard? It would use the John McPhee idea\ldots
\end{joebox}

\begin{joebox}
OK, now a non-naive idea (which may be less useful than the naive one).
I invented something called ``futon theory''---it's not quite what it
sounds like. Futons (in my theory) are atoms of the future which are
perceivable in the present. [\ldots]
It's a bit different in mathematics because in principle we don't know
if the result is actually true. But what we could do is design the
search process, and then we would at least know that the search
process is working the way we hoped it would. It's the somewhat more
technical version of the ``Dear Codex'' letter you just
wrote---building a process that you have utter confidence in that
will shed light on the domain even if it doesn't reveal the exact
answer.
\end{joebox}

\pivot{Two ideas collide: McPhee's ``letter to yourself'' technique
  for writer's block, and futon theory's ``atoms of the future.''
  Both converge on the same insight: process design is tractable even
  when the mathematical outcome is uncertain.}

\begin{joebox}
Great, I think you have your work cut out for you. It's cool that
this involved all the strategies, including the ``least likely''
Strategy~C that surprised us by being particularly flavourful
\end{joebox}

%% ============================================================
\section*{Act V: Crisis --- P6 and P7}
%% ============================================================

\begin{joebox}
I thought \#6 and \#7 were proved so I'm glad I asked!
\end{joebox}

\begin{joebox}
Well I am a bit perplexed b/c at this time yesterday I had been
convinced that Problem~7 was done. Look at Problem~7 related git
logs to try and figure out what's been going on with this journey.
\end{joebox}

\begin{joebox}
Well, I have another Codex program mining Arxiv right now for wiring
diagrams that we might use. But this presupposes that we are actually
looking at gaps and not a decomposing corpse.
\end{joebox}

\pivot{The most dramatic moment in the sprint. A problem believed
  closed reopens with structural gaps. The ``decomposing corpse''
  quip captures the uncertainty: is the proof salvageable, or
  fundamentally dead?}

\begin{joebox}
Hm, claim~1 sounds a bit like replacing an overclaim with weasel
words; \#2~similarly. I think we need to restructure the proof?
\end{joebox}

\begin{claudebox}[title={Claude --- Gap triage}]
G1 (fundamental group control) and G3 (rational acyclicity) can be
closed once the cap is homotopy equivalent to $F \times D^2$
rel~boundary. [\ldots] G2 (Browder--Quinn identification) remains
the genuine remaining gap.
\end{claudebox}

\pivot{The monolithic ``open Problem~7'' label is decomposed into a
  prioritised closure queue: G1, G2, G3. Within hours, G1 and G3
  are resolved, isolating G2 as the sole remaining obligation.}

\begin{joebox}
Could we cite secondary literature for G2 that is on Arxiv?
Mathematics is full of folklore
\end{joebox}

\begin{claudebox}[title={Claude --- Citation search result}]
There is no single modern arXiv theorem that states the exact
identification in one line. The bridge is assembled from standard
components: Browder~(1968), L\'opez de Medrano~(1971),
Cappell--Shaneson~(1974), Ranicki~(1998).
\end{claudebox}

\pivot{Citation strategy stabilises: replace ``single theorem lookup''
  with a theorem-number ledger across four classical sources.}

\begin{joebox}
It really sounds to me like the proof of Problem~6 is hanging by a
thread
\end{joebox}

%% ============================================================
\section*{Act VI: Student Dispatch and Agent Teaming}
%% ============================================================

\begin{joebox}
Yeah, I guess the model is rather than treating Codex as a junior
developer as we sometimes do in coding applications, we would treat
them as a ``grad student.''
\end{joebox}

\begin{joebox}
Maybe we should call it student-dispatch to include undergrads, bright
high schoolers, and precocious five year olds like Codex
\end{joebox}

\pivot{The \texttt{agent/student-dispatch} pattern is named. It
  reframes the Codex interaction from ``run this script'' to
  ``here is context, dead ends, directions---explore and report.''}

\begin{joebox}
So I wonder what we could do to help. I've almost run out of ideas.
I guess one pattern I've noticed is that we can support the
``reduction'' pattern by verifying everything that is claimed except
the gap. Closing gaps (as with Problem~4) can sometimes turn out to
be a focused computational task. Another approach is Math Overflow
search or Arxiv search to create wiring diagrams that might help or
might provide analogous reasoning structures. That last part is a bit
woolly.
\end{joebox}

\begin{joebox}
So, if we were able to turn SE into wiring diagrams on the superpod,
then we could use tensor math to search them!
\end{joebox}

\begin{joebox}
Ah, and for ``artificial stack exchange'' what we could do is
\emph{induce} new wiring diagrams, notice where they sit relative to
other known diagrams (this corresponds to ``asking a question'')
\end{joebox}

\pivot{A forward-looking connection: the proof sprint's
  wiring-diagram methodology could be mechanised via tensor search
  over StackExchange-derived argument structures. This motivates
  the Stage~7 thread-wiring work in the futon6 pipeline.}

%% ============================================================
\section*{Act VII: The Coaching Breakthrough (Problem 6)}
%% ============================================================

\begin{joebox}
Well, and I want to talk with you about Problem~6. I think that's the
only one outstanding. But I'm realizing that the ``other Claude'' has
been sending me on continual gap-filling missions without giving me
a valid confidence signal. It's like Gemini saying ``I'll fix your
code right away sir!'' [\ldots] Problem~6 has not brought out the best
in Claude.
\end{joebox}

\begin{joebox}
I think a good approach might be: ``Can you find any other
reductions?'' That would start to build some intelligence around the
pattern rather than just a TryHarder strategy.
\end{joebox}

\pivot{The failure mode is diagnosed: dispatching closure attempts
  without layer-level calibration. The ``reductions'' framing shifts
  from ``close the gap'' to ``map the landscape.''}

\begin{joebox}
I think you should think out of the box a bit. What kind of problem is
this? What kind of proof applies to this kind of problem? How would
you teach it to an undergraduate? How would you teach it to a grad
student? Where, in reality, do people learn about this kind of
problem? What kind of person finds this kind of problem easy? Are
there ``tricks'' (symmetries) that would make some of your Zeno's
Paradoxes go away? Like one from my thesis in the resources directory
is an example from Gowers in which binomial expansion terms have
opposite signs and so cancel. Like, OK, this problem was an ``open
problem'' not a textbook problem but come on---it's on the research
frontier, we're not asking you to solve alien mathematics from the
year~5000AD. This is totally within your capability space. Why aren't
you finding the answer in 30~minutes?
\end{joebox}

\pivot{The coaching intervention. Five questions, each forcing a
  different reframing: layer enumeration (``what kind of problem?''),
  pedagogical reduction (``teach it to an undergrad''), sociological
  targeting (``who finds this easy?''), structural cancellation
  (``tricks / symmetries''), and honest calibration (``why not
  30~minutes?'').}

\begin{codexbox}[title={Codex --- 15 minutes later}]
The proof for $K_n$ is complete. $\bar{d} = 2t/(n\varepsilon) \le 2/3$
at $T = \varepsilon n/3$ steps. Pigeonhole gives $\exists v$ with
$\operatorname{tr}(Y) \le 2/3$, and $\|Y\| \le \operatorname{tr}(Y)$
for PSD matrices. The barrier is maintained at every step.
Result: $|S| = \varepsilon n/3$, $c = 1/3$.
\end{codexbox}

\begin{codexbox}[title={Codex --- The irony}]
The irony: this entire proof is elementary---no interlacing families,
no Borcea--Br\"and\'en, no MSS theorem needed. Just PSD trace bound
$+$ pigeonhole $+$ Tur\'an.
\end{codexbox}

\pivot{The layer switch lands. Eight hours of sparsification-layer
  attacks are bypassed by an elementary combinatorial proof discovered
  in 15~minutes. The proof uses three undergraduate-level facts:
  Tur\'an's theorem, the PSD trace bound, and the pigeonhole
  principle. The coaching did not supply mathematical content---it
  supplied \emph{questions} that forced layer enumeration.}

\begin{joebox}
It's so cool that this is the same basic solution we have used to
overcome continual shoring up of bad design in futon3, and also the
same solution I've used with Codex just now by getting it to spot
syntax errors rather than depending on me to do it.
\end{joebox}

\begin{claudebox}[title={Claude --- The meta-pattern}]
The diagnostic question is always the same: ``who finds this easy?''
If the answer is ``someone other than the entity currently
struggling,'' you're in the wrong layer.
\end{claudebox}

\pivot{The layer-switching pattern is recognized as fractal: it
  appears in proof strategy (Problem~6), in software design (futon3),
  and in task allocation (Codex as syntax checker). The common
  diagnostic: ask who would find the task trivial.}

%% ============================================================
\section*{Act VIII: The Monograph}
%% ============================================================

\begin{joebox}
In addition to the PAR I think it would be good to look through the
git log on futon6 and create a ``making of'' summary note.
\end{joebox}

\begin{joebox}
Also in terms of ``what we've learned''---in the making-of I think we
could add something about the relevance of this pattern to agent
teaming (further work in futon3c). Maybe this even goes back to the
work in futon2 ants---i.e.\ we are exploring a domain, bringing back
relevant information.
\end{joebox}

\begin{joebox}
I think the key thing you could help with now is looking through
\texttt{history.jsonl} for good quotes for Part~III.
\end{joebox}

\pivot{The sprint becomes its own document. The recursive
  structure---a proof sprint whose Part~III records the prompts that
  shaped Part~II---closes the loop between process and product.}
