\section{Problem 1: Equivalence of \(\Phi^4_3\) Measure Under Smooth Shifts}

\subsection{Problem Statement}

Let \(T^3\) be the 3D unit torus. Let \(\mu\) be the \(\Phi^4_3\) measure on
\(\mathcal D'(T^3)\). Let \(\psi: T^3 \to \mathbb R\) be smooth and nonzero, and
let \(T_\psi(u)=u+\psi\). Are \(\mu\) and \(T_\psi^*\mu\) equivalent
(same null sets)?

\subsection{Answer}

\textbf{Yes.} The measures \(\mu\) and \(T_\psi^*\mu\) are equivalent.

\subsection{Solution}

\subsubsection{1. The \(\Phi^4_3\) measure}

Formally,
\[
d\mu(\phi)=Z^{-1}e^{-V(\phi)}\,d\mu_0(\phi),
\]
where \(\mu_0\) is the Gaussian free field measure with covariance
\((m^2-\Delta)^{-1}\) on \(T^3\), and
\[
V(\phi)=\int_{T^3}\bigl(:\phi^4:-C:\phi^2:\bigr)\,dx.
\]
Here \( :\phi^k: \) denotes Wick powers and \(C\) is the mass counterterm.

\subsubsection{2. Absolute continuity with respect to the GFF}

The \(\Phi^4_3\) measure is equivalent to \(\mu_0\):
\[
\mu \sim \mu_0.
\]
In the variational construction (Barashkov--Gubinelli, Theorem 1.1),
\(\mathbb E_{\mu_0}[e^{-V}]<\infty\), so \(\mu\ll\mu_0\) with strictly positive
density \(Z^{-1}e^{-V(\phi)}\), and therefore \(\mu_0\ll\mu\) as well.

\subsubsection{3. Cameron--Martin theory for the GFF}

For covariance \((m^2-\Delta)^{-1}\), the Cameron--Martin space is
\[
H=H^1(T^3).
\]
Since \(\psi\in C^\infty(T^3)\subset H^1(T^3)\), the Cameron--Martin theorem
applies:
\[
\frac{d\,T_h^*\mu_0}{d\mu_0}(\phi)=\exp\!\left(\ell_h(\phi)-\frac{\|h\|_H^2}{2}\right),
\qquad h\in H.
\]
Hence \(T_\psi^*\mu_0\sim\mu_0\).

\subsubsection{4. Shift of the interacting measure}

The shifted measure has density
\[
\frac{d\,T_\psi^*\mu}{d\mu_0}(\phi)=
Z^{-1}e^{-V(\phi-\psi)}
\exp\!\left(\ell_\psi(\phi)-\frac{\|\psi\|_H^2}{2}\right).
\]
Expanding \(V(\phi-\psi)\) (with Wick ordering relative to \(\mu_0\)):
\begin{align*}
V(\phi-\psi)=\;&V(\phi)-4\!\int\!\psi:\phi^3:\,dx
+6\!\int\!\psi^2:\phi^2:\,dx \\
&-4\!\int\!\psi^3:\phi:\,dx
+\int\!\psi^4\,dx
-C\!\left(\int\!:\phi^2:\,dx-2\!\int\!\psi:\phi:\,dx+\int\!\psi^2\,dx\right) \\
&+\text{(renormalization corrections)}.
\end{align*}
The shift introduces the standard additional renormalization contribution in
the mass term; after renormalization,
\(
V(\phi-\psi)-V(\phi)
\)
is well-defined.

\subsubsection{5. Integrability and equivalence}

The Radon--Nikodym factor between \(T_\psi^*\mu\) and \(\mu\) is
\[
R(\phi)=\exp\!\bigl(-(V(\phi-\psi)-V(\phi))+\ell_\psi(\phi)-\text{const}\bigr).
\]
It is strictly positive almost surely. The nontrivial point is integrability.
The leading fluctuation term is \(\int \psi:\phi^3:\,dx\), and the
log-Sobolev/coercivity estimates in the \(\Phi^4_3\) construction imply local
exponential moments, of the form
\[
\mathbb E_\mu\!\left[e^{\,t|\int \psi:\phi^3:\,dx|}\right]<\infty
\quad\text{for }|t|<t_0.
\]
This is enough to place \(R\) in \(L^1(\mu)\), and similarly \(R^{-1}\) in
\(L^1(T_\psi^*\mu)\). Therefore
\[
T_\psi^*\mu \sim \mu.
\]

\subsubsection{6. Variational viewpoint}

In the Bou\'e--Dupuis variational representation used by Barashkov--Gubinelli,
\[
-\log Z=\inf_{u}\;
\mathbb E\!\left[V\!\left(\phi+\int_0^1 u_s\,ds\right)+
\frac12\int_0^1\|u_s\|^2\,ds\right].
\]
Shifting by \(\psi\) corresponds to a finite change in the variational problem,
compatible with equivalence of measures.

\subsubsection{7. Summary}

The proof chain is:
\begin{enumerate}
\item \(\mu\sim\mu_0\) (strictly positive Gibbs density relative to \(\mu_0\)).
\item \(T_\psi^*\mu_0\sim\mu_0\) (Cameron--Martin, since \(\psi\in H^1\)).
\item The renormalized difference \(V(\phi-\psi)-V(\phi)\) is well-defined.
\item The corresponding exponential factor is integrable.
\end{enumerate}
Hence \(T_\psi^*\mu\sim\mu\).

\subsection{References}

\begin{itemize}
\item N. Barashkov, M. Gubinelli, \emph{A variational method for \(\Phi^4_3\)},
  Duke Math. J. 169 (2020), 3339--3415.
\item M. Hairer, \emph{A theory of regularity structures},
  Invent. Math. 198 (2014), 269--504.
\item M. Gubinelli, P. Imkeller, N. Perkowski,
  \emph{Paracontrolled distributions and singular PDEs},
  Forum Math. Pi 3 (2015).
\end{itemize}

\subsection{Key References from futon6 corpus}

\begin{itemize}
\item PlanetMath: distribution
\item PlanetMath: Sobolev space
\item PlanetMath: Gaussian measure
\item PlanetMath: Radon--Nikodym theorem
\end{itemize}
