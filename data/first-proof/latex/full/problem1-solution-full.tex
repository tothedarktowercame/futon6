\hypertarget{problem-1-equivalence-of-phi4_3-measure-under-smooth-shifts}{%
\section{\texorpdfstring{Problem 1: Equivalence of \(\Phi^\mNumber{4}_\mNumber{3}\) Measure Under Smooth Shifts}{Problem 1: Equivalence of \textbackslash Phi\^{}\textbackslash mNumber\{4\}\_\textbackslash mNumber\{3\} Measure Under Smooth Shifts}}\label{problem-1-equivalence-of-phi4_3-measure-under-smooth-shifts}}

\hypertarget{problem-statement}{%
\subsection{Problem Statement}\label{problem-statement}}

Let \(T^\mNumber{3}\) be the 3D unit torus. Let \(\mu\) be the \(\Phi^\mNumber{4}_\mNumber{3}\) measure on \(D'(T^\mNumber{3})\).
Let \(\psi\): \(T^\mNumber{3}\) -\textgreater{} R be a smooth nonzero function and \(T_{\psi}(u) = u \mBridgeOperator{+} \psi\) the
shift map. Are \(\mu\) and \(T_{\psi}^{\mDualStar} \mu\) equivalent (same null sets)?

\hypertarget{answer}{%
\subsection{Answer}\label{answer}}

\textbf{Yes.} The measures \(\mu\) and \(T_{\psi}^{\mDualStar} \mu\) are equivalent.

\textbf{Confidence: Medium-high.} The argument combines Cameron-Martin theory with
the absolute continuity structure of the \(\Phi^\mNumber{4}_\mNumber{3}\) construction (Barashkov-Gubinelli
2020, Theorem 1.1). The integrability bound is stated for a neighborhood of the
required exponent, not for all t, matching the available log-Sobolev technology.

\hypertarget{solution}{%
\subsection{Solution}\label{solution}}

\hypertarget{1-the-phi4_3-measure}{%
\subsubsection{\texorpdfstring{1. The \(\Phi^\mNumber{4}_\mNumber{3}\) measure}{1. The \textbackslash Phi\^{}\textbackslash mNumber\{4\}\_\textbackslash mNumber\{3\} measure}}\label{1-the-phi4_3-measure}}

The \(\Phi^\mNumber{4}_\mNumber{3}\) measure \(\mu\) on \(D'(T^\mNumber{3})\) is the probability measure formally
written as:

\[d\mu(\phi) = Z^{-\mNumber{1}} \exp(-V(\phi)) d\mu_\mNumber{0}(\phi)\]

where \(\mu_\mNumber{0}\) is the Gaussian free field (GFF) measure with covariance
(\(m^\mNumber{2}\) - Delta)\^{}\{-1\} on \(T^\mNumber{3}\), and the interaction is:

\[V(\phi) = \Integral_{T^\mNumber{3}} (:\phi^\mNumber{4}: - C :\phi^\mNumber{2}:) \mathrm{d}x\]

Here \(:\phi^{k}\): denotes the k-th Wick power (renormalized product), and C is a
mass counterterm that diverges under regularization removal.

The rigorous construction (Hairer 2014, Gubinelli-Imkeller-Perkowski 2015,
Barashkov-Gubinelli 2020) produces \(\mu\) as a well-defined probability measure
on distributions of regularity \(C^{-\mNumber{1}/\mNumber{2}-\epsilon}(T^{\mNumber{3}}\)).

\hypertarget{2-absolute-continuity-with-respect-to-the-gff}{%
\subsubsection{2. Absolute continuity with respect to the GFF}\label{2-absolute-continuity-with-respect-to-the-gff}}

\textbf{Key fact:} The \(\Phi^\mNumber{4}_\mNumber{3}\) measure \(\mu\) is equivalent to (has the same null
sets as) the base Gaussian free field measure \(\mu_\mNumber{0}\):

\[\mu ~ \mu_\mNumber{0}\]

This equivalence follows from the variational construction of
Barashkov-Gubinelli (2020, Theorem 1.1), which establishes
\(E_{\mu_\mNumber{0}}[\exp(-V)]\) \textless{} infinity and hence \(\mu << \mu_{\mNumber{0}}\) with strictly positive
density exp(-V(phi))/Z. Since exp(-V(phi)) \textgreater{} 0 a.s. (exponential is always
positive), the reverse absolute continuity \(\mu_{\mNumber{0}} << \mu\) also holds.

\hypertarget{3-cameron-martin-theory-for-the-gff}{%
\subsubsection{3. Cameron-Martin theory for the GFF}\label{3-cameron-martin-theory-for-the-gff}}

For the Gaussian free field \(\mu_\mNumber{0}\) on \(T^\mNumber{3}\) with covariance C = (\(m^\mNumber{2}\) - Delta)\^{}\{-1\},
the Cameron-Martin space is:

\[H = H^\mNumber{1}(T^\mNumber{3}) (\text{Sobolev} space of order \mNumber{1})\]

Since \(\psi\) is smooth, \(\psi \in C^{\infty}(T^\mNumber{3})\) subset \(H^\mNumber{1}(T^\mNumber{3})\), so \(\psi\) is in the
Cameron-Martin space.

\textbf{Cameron-Martin theorem:} For any \(h \in H\), the shifted Gaussian measure
\(T_{h}^{\mDualStar} \mu_{\mNumber{0}}\) is equivalent to \(\mu_\mNumber{0}\), with Radon-Nikodym derivative:

\[\mathrm{d}T_{h}^{\mDualStar} \mu_\mNumber{0} / d\mu_\mNumber{0} (\phi) = \exp(l_{h}(\phi) - ||h||_H^\mNumber{2} / \mNumber{2})\]

where \(l_{h}\) is the linear functional associated to h. In particular:

\[T_{\psi}^{\mDualStar} \mu_\mNumber{0} ~ \mu_\mNumber{0}\]

\hypertarget{4-shift-of-the-interacting-measure}{%
\subsubsection{4. Shift of the interacting measure}\label{4-shift-of-the-interacting-measure}}

To compute \(T_{\psi}^{\mDualStar} \mu,\) we need the density of the shifted interacting measure:

\[\mathrm{d}T_{\psi}^{\mDualStar} \mu / d\mu_\mNumber{0} (\phi) = Z^{-\mNumber{1}} \exp(-V(\phi - \psi)) \ast \exp(l_{\psi}(\phi) - ||\psi||_H^\mNumber{2}/\mNumber{2})\]

The shifted interaction V(phi - psi) expands (using Wick ordering relative to \(\mu_\mNumber{0}\)):

\[\begin{aligned}
V(\phi - \psi) = V(\phi) - \mNumber{4} int \psi :\phi^\mNumber{3}: \mathrm{d}x + \mNumber{6} int \psi^\mNumber{2} :\phi^\mNumber{2}: \mathrm{d}x \\
- \mNumber{4} int \psi^\mNumber{3} :\phi: \mathrm{d}x + int \psi^\mNumber{4} \mathrm{d}x \\
- C(int :\phi^\mNumber{2}: \mathrm{d}x - \mNumber{2} int \psi :\phi: \mathrm{d}x + int \psi^\mNumber{2} \mathrm{d}x) + (renorm. corrections)
\end{aligned}\]

\textbf{Renormalization under shift:} The term 6 int \(\psi^{\mNumber{2}} :\phi^{\mNumber{2}}\): dx generates an
additional logarithmic divergence (from \(\psi^\mNumber{2}\) multiplying the Wick square).
This is absorbed by shifting the mass counterterm:

\[C \to C + \mNumber{6} ||\psi||_{L^\mNumber{2}}^\mNumber{2} \ast (log N correction)\]

The precise counterterm shift is determined by the regularization scheme;
see Hairer (2014, Section 9) or Gubinelli-Imkeller-Perkowski (2015,
Proposition 6.3) for the explicit formula. For the present argument, only
the finiteness of the renormalized difference matters.

After renormalization, V(phi - psi) - V(phi) is a well-defined random variable
under \(\mu_\mNumber{0}\) (and under mu). The dominant fluctuation term is 4 int \(\psi :\phi^{\mNumber{3}}\): dx,
which has the right regularity:

\begin{itemize}
\tightlist
\item
  \(:\phi^\mNumber{3}\): in \(C^{-\mNumber{3}/\mNumber{2}-\epsilon}(T^{\mNumber{3}})\) (as a distribution)
\item
  \(\psi \in C^{\infty}(T^\mNumber{3})\) (smooth)
\item
  int \(\psi :\phi^{\mNumber{3}}\): dx is well-defined (pairing of smooth test function with distribution)
\end{itemize}

\hypertarget{5-integrability-and-equivalence}{%
\subsubsection{5. Integrability and equivalence}\label{5-integrability-and-equivalence}}

The Radon-Nikodym derivative \(\mathrm{d}T_{\psi}^{\mDualStar} \mu / d\mu\) involves:

\[R(\phi) = \exp(-(V(\phi-\psi) - V(\phi)) + l_{\psi}(\phi) - const)\]

We need \(R \in L^{\mNumber{1}}(\mu)\) and \(R > \mNumber{0}\) a.s.

\textbf{Positivity:} \(R > \mNumber{0}\) a.s. because it\textquotesingle s an exponential. ✓

\textbf{Integrability:} The critical term is exp(4 int \(\psi :\phi^{\mNumber{3}}\): dx) under the
\(\Phi^\mNumber{4}_\mNumber{3}\) measure. By the log-Sobolev inequality for \(\Phi^\mNumber{4}_\mNumber{3}\) (Barashkov-Gubinelli
2020), the measure has strong concentration: the tails of int \(\psi :\phi^{\mNumber{3}}\): dx
are controlled by the quartic interaction. Specifically:

\[E_{\mu}[\exp(t |int \psi :\phi^\mNumber{3}: \mathrm{d}x|)] < \infty \text{for} |t| < t_\mNumber{0}\]

where \(t_\mNumber{0} > \mNumber{0}\) depends on \(\|\psi\|_{C^\mNumber{0}}\) and the coupling constant.
This exponential integrability follows from the coercivity of the \(\phi^\mNumber{4}\)
interaction: the quartic potential dominates the cubic perturbation
(Barashkov-Gubinelli 2020, Section 4, exponential integrability from the
Polchinski flow). The bound suffices for \(R \in L^{\mNumber{1}}(\mu)\) since the exponent
in the Radon-Nikodym derivative is bounded by 4 \(\|\psi\|_{C^{\mNumber{0}}} |\Integral :\phi^{\mNumber{3}}:\,\mathrm{d}x|,\)
and \(t_\mNumber{0}\) can be chosen to exceed this coefficient.

Therefore \(R \in L^{\mNumber{1}}(\mu)\) and 1/\(R \in L\)\^{}1(\(T_{\psi}\)\^{}* mu), giving:

\[T_{\psi}^{\mDualStar} \mu ~ \mu (equivalent measures)\]

\hypertarget{6-alternative-argument-via-the-variational-approach}{%
\subsubsection{6. Alternative argument via the variational approach}\label{6-alternative-argument-via-the-variational-approach}}

Barashkov-Gubinelli (2020) construct the \(\Phi^\mNumber{4}_\mNumber{3}\) measure via the Boué-Dupuis
variational formula, which represents:

\[-log Z = inf_{u} E[V(\phi + \int_\mNumber{0}^\mNumber{1} u_{s} \mathrm{d}s) + \mNumber{1}/\mNumber{2} \int_\mNumber{0}^\mNumber{1} ||u_{s}||^\mNumber{2} \mathrm{d}s]\]

In this framework, shifting by \(\psi\) is equivalent to modifying the variational
problem by a shift in the drift, which produces an equivalent measure (the
infimum shifts by a finite amount, preserving absolute continuity).

\hypertarget{7-summary}{%
\subsubsection{7. Summary}\label{7-summary}}

The measures are equivalent because:

\begin{enumerate}
\def\labelenumi{\arabic{enumi}.}
\tightlist
\item
  \(\mu\) \textasciitilde{} \(\mu_\mNumber{0}\) (interacting measure equivalent to Gaussian, by positivity of density)
\item
  \(T_{\psi}^{\mDualStar} \mu_{\mNumber{0}}\) \textasciitilde{} \(\mu_\mNumber{0}\) (Cameron-Martin theorem, since \(\psi \in H^\mNumber{1}\))
\item
  V(phi - psi) - V(phi) is well-defined after renormalization
\item
  The exponential of the cubic perturbation is integrable (log-Sobolev / coercivity)
\item
  Therefore \(T_{\psi}^{\mDualStar} \mu\) \textasciitilde{} \(\mu\)
\end{enumerate}

\hypertarget{references}{%
\subsection{References}\label{references}}

\begin{itemize}
\tightlist
\item
  N. Barashkov, M. Gubinelli, "A variational method for \(\Phi^\mNumber{4}_\mNumber{3},"\) Duke Math J.
  169 (2020), 3339-3415. {[}Theorem 1.1: construction and integrability; Section 4:
  exponential integrability from Polchinski flow{]}
\item
  M. Hairer, "A theory of regularity structures," Inventiones Math. 198 (2014),
  269-504. {[}Section 9: renormalization counterterms{]}
\item
  M. Gubinelli, P. Imkeller, N. Perkowski, "Paracontrolled distributions and
  singular PDEs," Forum Math. \(\Pi\) 3 (2015). {[}Proposition 6.3: explicit counterterm formula{]}
\end{itemize}

\hypertarget{key-references-from-futon6-corpus}{%
\subsection{Key References from futon6 corpus}\label{key-references-from-futon6-corpus}}

\begin{itemize}
\tightlist
\item
  PlanetMath: "distribution" --- distributions on manifolds
\item
  PlanetMath: "Sobolev space" --- Cameron-Martin space is \(H^\mNumber{1}\)
\item
  PlanetMath: "Gaussian measure" --- reference measure \(\mu_\mNumber{0}\)
\item
  PlanetMath: "Radon-Nikodym theorem" --- absolute continuity and RN derivatives
\end{itemize}
