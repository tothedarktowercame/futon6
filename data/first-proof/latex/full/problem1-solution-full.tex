\section{Problem 1: Equivalence of Phi\^{}4\_3 Measure Under Smooth Shifts}

\subsection{Problem Statement}

Let T\^{}3 be the 3D unit torus. Let mu be the Phi\^{}4\_3 measure on D\textquotesingle(T\^{}3).
Let psi: T\^{}3 -\textgreater{} R be a smooth nonzero function and T\_psi(u) = u + psi the
shift map. Are mu and T\_psi\^{}* mu equivalent (same null sets)?

\subsection{Answer}

\textbf{Yes.} The measures mu and T\_psi\^{}* mu are equivalent.

\textbf{Confidence: Medium-high.} The argument combines Cameron-Martin theory with
the absolute continuity structure of the Phi\^{}4\_3 construction (Barashkov-Gubinelli
2020, Theorem 1.1). The integrability bound is stated for a neighborhood of the
required exponent, not for all t, matching the available log-Sobolev technology.

\subsection{Solution}

\subsubsection{1. The Phi\^{}4\_3 measure}

The Phi\^{}4\_3 measure mu on D\textquotesingle(T\^{}3) is the probability measure formally
written as:

\begin{verbatim}
dmu(phi) = Z^{-1} exp(-V(phi)) dmu_0(phi)
\end{verbatim}

where mu\_0 is the Gaussian free field (GFF) measure with covariance
(m\^{}2 - Delta)\^{}\{-1\} on T\^{}3, and the interaction is:

\begin{verbatim}
V(phi) = integral_{T^3} (:phi^4: - C :phi^2:) dx
\end{verbatim}

Here :phi\^{}k: denotes the k-th Wick power (renormalized product), and C is a
mass counterterm that diverges under regularization removal.

The rigorous construction (Hairer 2014, Gubinelli-Imkeller-Perkowski 2015,
Barashkov-Gubinelli 2020) produces mu as a well-defined probability measure
on distributions of regularity C\^{}\{-1/2-epsilon\}(T\^{}3).

\subsubsection{2. Absolute continuity with respect to the GFF}

\textbf{Key fact:} The Phi\^{}4\_3 measure mu is equivalent to (has the same null
sets as) the base Gaussian free field measure mu\_0:

\begin{verbatim}
mu ~ mu_0
\end{verbatim}

This equivalence follows from the variational construction of
Barashkov-Gubinelli (2020, Theorem 1.1), which establishes
E\_\{mu\_0\}{[}exp(-V){]} \textless{} infinity and hence mu \textless\textless{} mu\_0 with strictly positive
density exp(-V(phi))/Z. Since exp(-V(phi)) \textgreater{} 0 a.s. (exponential is always
positive), the reverse absolute continuity mu\_0 \textless\textless{} mu also holds.

\subsubsection{3. Cameron-Martin theory for the GFF}

For the Gaussian free field mu\_0 on T\^{}3 with covariance C = (m\^{}2 - Delta)\^{}\{-1\},
the Cameron-Martin space is:

\begin{verbatim}
H = H^1(T^3)  (Sobolev space of order 1)
\end{verbatim}

Since psi is smooth, psi in C\^{}infinity(T\^{}3) subset H\^{}1(T\^{}3), so psi is in the
Cameron-Martin space.

\textbf{Cameron-Martin theorem:} For any h in H, the shifted Gaussian measure
T\_h\^{}* mu\_0 is equivalent to mu\_0, with Radon-Nikodym derivative:

\begin{verbatim}
dT_h^* mu_0 / dmu_0 (phi) = exp(l_h(phi) - ||h||_H^2 / 2)
\end{verbatim}

where l\_h is the linear functional associated to h. In particular:

\begin{verbatim}
T_psi^* mu_0 ~ mu_0
\end{verbatim}

\subsubsection{4. Shift of the interacting measure}

To compute T\_psi\^{}* mu, we need the density of the shifted interacting measure:

\begin{verbatim}
dT_psi^* mu / dmu_0 (phi) = Z^{-1} exp(-V(phi - psi)) * exp(l_psi(phi) - ||psi||_H^2/2)
\end{verbatim}

The shifted interaction V(phi - psi) expands (using Wick ordering relative to mu\_0):

\begin{verbatim}
V(phi - psi) = V(phi) - 4 int psi :phi^3: dx + 6 int psi^2 :phi^2: dx
               - 4 int psi^3 :phi: dx + int psi^4 dx
               - C(int :phi^2: dx - 2 int psi :phi: dx + int psi^2 dx) + (renorm. corrections)
\end{verbatim}

\textbf{Renormalization under shift:} The term 6 int psi\^{}2 :phi\^{}2: dx generates an
additional logarithmic divergence (from psi\^{}2 multiplying the Wick square).
This is absorbed by shifting the mass counterterm:

\begin{verbatim}
C -> C + 6 ||psi||_{L^2}^2 * (log N correction)
\end{verbatim}

The precise counterterm shift is determined by the regularization scheme;
see Hairer (2014, Section 9) or Gubinelli-Imkeller-Perkowski (2015,
Proposition 6.3) for the explicit formula. For the present argument, only
the finiteness of the renormalized difference matters.

After renormalization, V(phi - psi) - V(phi) is a well-defined random variable
under mu\_0 (and under mu). The dominant fluctuation term is 4 int psi :phi\^{}3: dx,
which has the right regularity:

\begin{itemize}
\tightlist
\item
  :phi\^{}3: in C\^{}\{-3/2-epsilon\}(T\^{}3) (as a distribution)
\item
  psi in C\^{}infinity(T\^{}3) (smooth)
\item
  int psi :phi\^{}3: dx is well-defined (pairing of smooth test function with distribution)
\end{itemize}

\subsubsection{5. Integrability and equivalence}

The Radon-Nikodym derivative dT\_psi\^{}* mu / dmu involves:

\begin{verbatim}
R(phi) = exp(-(V(phi-psi) - V(phi)) + l_psi(phi) - const)
\end{verbatim}

We need R in L\^{}1(mu) and R \textgreater{} 0 a.s.

\textbf{Positivity:} R \textgreater{} 0 a.s. because it\textquotesingle s an exponential. ✓

\textbf{Integrability:} The critical term is exp(4 int psi :phi\^{}3: dx) under the
Phi\^{}4\_3 measure. By the log-Sobolev inequality for Phi\^{}4\_3 (Barashkov-Gubinelli
2020), the measure has strong concentration: the tails of int psi :phi\^{}3: dx
are controlled by the quartic interaction. Specifically:

\begin{verbatim}
E_mu[exp(t |int psi :phi^3: dx|)] < infinity  for |t| < t_0
\end{verbatim}

where t\_0 \textgreater{} 0 depends on \textbar\textbar psi\textbar\textbar{}\emph{\{C\^{}0\} and the coupling constant.
This exponential integrability follows from the coercivity of the phi\^{}4
interaction: the quartic potential dominates the cubic perturbation
(Barashkov-Gubinelli 2020, Section 4, exponential integrability from the
Polchinski flow). The bound suffices for R in L\^{}1(mu) since the exponent
in the Radon-Nikodym derivative is bounded by 4 \textbar\textbar psi\textbar\textbar{}}\{C\^{}0\} \textbar int :phi\^{}3: dx\textbar,
and t\_0 can be chosen to exceed this coefficient.

Therefore R in L\^{}1(mu) and 1/R in L\^{}1(T\_psi\^{}* mu), giving:

\begin{verbatim}
T_psi^* mu ~ mu  (equivalent measures)
\end{verbatim}

\subsubsection{6. Alternative argument via the variational approach}

Barashkov-Gubinelli (2020) construct the Phi\^{}4\_3 measure via the Boué-Dupuis
variational formula, which represents:

\begin{verbatim}
-log Z = inf_{u} E[V(phi + int_0^1 u_s ds) + 1/2 int_0^1 ||u_s||^2 ds]
\end{verbatim}

In this framework, shifting by psi is equivalent to modifying the variational
problem by a shift in the drift, which produces an equivalent measure (the
infimum shifts by a finite amount, preserving absolute continuity).

\subsubsection{7. Summary}

The measures are equivalent because:

\begin{enumerate}
\def\labelenumi{\arabic{enumi}.}
\tightlist
\item
  mu \textasciitilde{} mu\_0 (interacting measure equivalent to Gaussian, by positivity of density)
\item
  T\_psi\^{}* mu\_0 \textasciitilde{} mu\_0 (Cameron-Martin theorem, since psi in H\^{}1)
\item
  V(phi - psi) - V(phi) is well-defined after renormalization
\item
  The exponential of the cubic perturbation is integrable (log-Sobolev / coercivity)
\item
  Therefore T\_psi\^{}* mu \textasciitilde{} mu
\end{enumerate}

\subsection{References}

\begin{itemize}
\tightlist
\item
  N. Barashkov, M. Gubinelli, "A variational method for Phi\^{}4\_3," Duke Math J.
  169 (2020), 3339-3415. {[}Theorem 1.1: construction and integrability; Section 4:
  exponential integrability from Polchinski flow{]}
\item
  M. Hairer, "A theory of regularity structures," Inventiones Math. 198 (2014),
  269-504. {[}Section 9: renormalization counterterms{]}
\item
  M. Gubinelli, P. Imkeller, N. Perkowski, "Paracontrolled distributions and
  singular PDEs," Forum Math. Pi 3 (2015). {[}Proposition 6.3: explicit counterterm formula{]}
\end{itemize}

\subsection{Key References from futon6 corpus}

\begin{itemize}
\tightlist
\item
  PlanetMath: "distribution" --- distributions on manifolds
\item
  PlanetMath: "Sobolev space" --- Cameron-Martin space is H\^{}1
\item
  PlanetMath: "Gaussian measure" --- reference measure mu\_0
\item
  PlanetMath: "Radon-Nikodym theorem" --- absolute continuity and RN derivatives
\end{itemize}
