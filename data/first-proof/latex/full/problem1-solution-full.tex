\hypertarget{problem-1-equivalence-of-phi4_3-measure-under-smooth-shifts}{%
\section{\texorpdfstring{Problem 1: Equivalence of \(\Phi^4_3\) Measure Under Smooth Shifts}{Problem 1: Equivalence of \textbackslash Phi\^{}4\_3 Measure Under Smooth Shifts}}\label{problem-1-equivalence-of-phi4_3-measure-under-smooth-shifts}}

\hypertarget{problem-statement}{%
\subsection{Problem Statement}\label{problem-statement}}

Let \(T^3\) be the 3D unit torus. Let \(\mu\) be the \(\Phi^4_3\) measure on D\textquotesingle(\(T^3\)).
Let \(\psi\): \(T^3\) -\textgreater{} R be a smooth nonzero function and \(T_{\psi}(u) = u\) + \(\psi\) the
shift map. Are \(\mu\) and \(T_{\psi}^*\) \(\mu\) equivalent (same null sets)?

\hypertarget{answer}{%
\subsection{Answer}\label{answer}}

\textbf{Yes.} The measures \(\mu\) and \(T_{\psi}^*\) \(\mu\) are equivalent.

\textbf{Confidence: Medium-high.} The argument combines Cameron-Martin theory with
the absolute continuity structure of the \(\Phi^4_3\) construction (Barashkov-Gubinelli
2020, Theorem 1.1). The integrability bound is stated for a neighborhood of the
required exponent, not for all t, matching the available log-Sobolev technology.

\hypertarget{solution}{%
\subsection{Solution}\label{solution}}

\hypertarget{1-the-phi4_3-measure}{%
\subsubsection{\texorpdfstring{1. The \(\Phi^4_3\) measure}{1. The \textbackslash Phi\^{}4\_3 measure}}\label{1-the-phi4_3-measure}}

The \(\Phi^4_3\) measure \(\mu\) on D\textquotesingle(\(T^3\)) is the probability measure formally
written as:

\[dmu(\phi) = Z^{-1} exp(-V(\phi)) dmu_0(\phi)\]

where \(\mu_0\) is the Gaussian free field (GFF) measure with covariance
(\(m^2\) - Delta)\^{}\{-1\} on \(T^3,\) and the interaction is:

\[V(\phi) = integral_{T^3} (:\phi^4: - C :\phi^2:) dx\]

Here \(:\phi^{k}:\) denotes the k-th Wick power (renormalized product), and C is a
mass counterterm that diverges under regularization removal.

The rigorous construction (Hairer 2014, Gubinelli-Imkeller-Perkowski 2015,
Barashkov-Gubinelli 2020) produces \(\mu\) as a well-defined probability measure
on distributions of regularity \(C^{-1/2-\epsilon}\)(\(T^3\)).

\hypertarget{2-absolute-continuity-with-respect-to-the-gff}{%
\subsubsection{2. Absolute continuity with respect to the GFF}\label{2-absolute-continuity-with-respect-to-the-gff}}

\textbf{Key fact:} The \(\Phi^4_3\) measure \(\mu\) is equivalent to (has the same null
sets as) the base Gaussian free field measure \(\mu_0:\)

\[\mu ~ \mu_0\]

This equivalence follows from the variational construction of
Barashkov-Gubinelli (2020, Theorem 1.1), which establishes
\(E_{\mu_0}\){[}exp(-V){]} \textless{} infinity and hence \(\mu\) \textless\textless{} \(\mu_0\) with strictly positive
density exp(-V(phi))/Z. Since exp(-V(phi)) \textgreater{} 0 a.s. (exponential is always
positive), the reverse absolute continuity \(\mu_0\) \textless\textless{} \(\mu\) also holds.

\hypertarget{3-cameron-martin-theory-for-the-gff}{%
\subsubsection{3. Cameron-Martin theory for the GFF}\label{3-cameron-martin-theory-for-the-gff}}

For the Gaussian free field \(\mu_0\) on \(T^3\) with covariance C = (\(m^2\) - Delta)\^{}\{-1\},
the Cameron-Martin space is:

\[H = H^1(T^3) (Sobolev space of order 1)\]

Since \(\psi\) is smooth, \(\psi \in C^{infinity}(T^3)\) subset \(H^1(T^3),\) so \(\psi\) is in the
Cameron-Martin space.

\textbf{Cameron-Martin theorem:} For any \(h \in H\), the shifted Gaussian measure
\(T_{h}^*\) \(\mu_0\) is equivalent to \(\mu_0,\) with Radon-Nikodym derivative:

\[dT_{h}^ \ast \mu_0 / dmu_0 (\phi) = exp(l_{h}(\phi) - ||h||_H^2 / 2)\]

where \(l_{h}\) is the linear functional associated to h. In particular:

\[T_{\psi}^ \ast \mu_0 ~ \mu_0\]

\hypertarget{4-shift-of-the-interacting-measure}{%
\subsubsection{4. Shift of the interacting measure}\label{4-shift-of-the-interacting-measure}}

To compute \(T_{\psi}^*\) \(\mu\), we need the density of the shifted interacting measure:

\[dT_{\psi}^ \ast \mu / dmu_0 (\phi) = Z^{-1} exp(-V(\phi - \psi)) \ast exp(l_{\psi}(\phi) - ||\psi||_H^2/2)\]

The shifted interaction V(phi - psi) expands (using Wick ordering relative to \(\mu_0\)):

\[\begin{aligned}
V(\phi - \psi) = V(\phi) - 4 int \psi :\phi^3: dx + 6 int \psi^2 :\phi^2: dx \\
- 4 int \psi^3 :\phi: dx + int \psi^4 dx \\
- C(int :\phi^2: dx - 2 int \psi :\phi: dx + int \psi^2 dx) + (renorm. corrections)
\end{aligned}\]

\textbf{Renormalization under shift:} The term 6 int \(\psi^2\) \(:\phi^2:\) dx generates an
additional logarithmic divergence (from \(\psi^2\) multiplying the Wick square).
This is absorbed by shifting the mass counterterm:

\[C \to C + 6 ||\psi||_{L^2}^2 \ast (log N correction)\]

The precise counterterm shift is determined by the regularization scheme;
see Hairer (2014, Section 9) or Gubinelli-Imkeller-Perkowski (2015,
Proposition 6.3) for the explicit formula. For the present argument, only
the finiteness of the renormalized difference matters.

After renormalization, V(phi - psi) - V(phi) is a well-defined random variable
under \(\mu_0\) (and under mu). The dominant fluctuation term is 4 int \(\psi\) \(:\phi^3:\) dx,
which has the right regularity:

\begin{itemize}
\tightlist
\item
  \(:\phi^3: \in C^{-3/2-\epsilon}\)(\(T^3\)) (as a distribution)
\item
  \(\psi \in C^{infinity}(T^3)\) (smooth)
\item
  int \(\psi\) \(:\phi^3:\) dx is well-defined (pairing of smooth test function with distribution)
\end{itemize}

\hypertarget{5-integrability-and-equivalence}{%
\subsubsection{5. Integrability and equivalence}\label{5-integrability-and-equivalence}}

The Radon-Nikodym derivative \(dT_{\psi}^*\) \(\mu\) / dmu involves:

\[R(\phi) = exp(-(V(\phi-\psi) - V(\phi)) + l_{\psi}(\phi) - const)\]

We need \(R \in L\)\(^1(\mu)\) and R \textgreater{} 0 a.s.

\textbf{Positivity:} R \textgreater{} 0 a.s. because it\textquotesingle s an exponential. ✓

\textbf{Integrability:} The critical term is exp(4 int \(\psi\) \(:\phi^3:\) dx) under the
\(\Phi^4_3\) measure. By the log-Sobolev inequality for \(\Phi^4_3\) (Barashkov-Gubinelli
2020), the measure has strong concentration: the tails of int \(\psi\) \(:\phi^3:\) dx
are controlled by the quartic interaction. Specifically:

\[E_{\mu}[exp(t |int \psi :\phi^3: dx|)] < infinity for |t| < t_0\]

where \(t_0\) \textgreater{} 0 depends on \textbar\textbar psi\textbar\textbar{}\emph{\({C^0}\) and the coupling constant.
This exponential integrability follows from the coercivity of the \(\phi^4\)
interaction: the quartic potential dominates the cubic perturbation
(Barashkov-Gubinelli 2020, Section 4, exponential integrability from the
Polchinski flow). The bound suffices for \(R \in L\)\(^1(\mu)\) since the exponent
in the Radon-Nikodym derivative is bounded by 4 \textbar\textbar psi\textbar\textbar{}}\({C^0}\) \textbar int \(:\phi^3:\) dx\textbar,
and \(t_0\) can be chosen to exceed this coefficient.

Therefore \(R \in L\)\(^1(\mu)\) and 1/\(R \in L\)\^{}1(\(T_{\psi}\)\^{}* mu), giving:

\[T_{\psi}^ \ast \mu ~ \mu (equivalent measures)\]

\hypertarget{6-alternative-argument-via-the-variational-approach}{%
\subsubsection{6. Alternative argument via the variational approach}\label{6-alternative-argument-via-the-variational-approach}}

Barashkov-Gubinelli (2020) construct the \(\Phi^4_3\) measure via the Boué-Dupuis
variational formula, which represents:

\[-log Z = inf_{u} E[V(\phi + int_0^1 u_{s} ds) + 1/2 int_0^1 ||u_{s}||^2 ds]\]

In this framework, shifting by \(\psi\) is equivalent to modifying the variational
problem by a shift in the drift, which produces an equivalent measure (the
infimum shifts by a finite amount, preserving absolute continuity).

\hypertarget{7-summary}{%
\subsubsection{7. Summary}\label{7-summary}}

The measures are equivalent because:

\begin{enumerate}
\def\labelenumi{\arabic{enumi}.}
\tightlist
\item
  \(\mu\) \textasciitilde{} \(\mu_0\) (interacting measure equivalent to Gaussian, by positivity of density)
\item
  \(T_{\psi}^*\) \(\mu_0\) \textasciitilde{} \(\mu_0\) (Cameron-Martin theorem, since \(\psi \in H^1\))
\item
  V(phi - psi) - V(phi) is well-defined after renormalization
\item
  The exponential of the cubic perturbation is integrable (log-Sobolev / coercivity)
\item
  Therefore \(T_{\psi}^*\) \(\mu\) \textasciitilde{} \(\mu\)
\end{enumerate}

\hypertarget{references}{%
\subsection{References}\label{references}}

\begin{itemize}
\tightlist
\item
  N. Barashkov, M. Gubinelli, "A variational method for \(\Phi^4_3\)," Duke Math J.
  169 (2020), 3339-3415. {[}Theorem 1.1: construction and integrability; Section 4:
  exponential integrability from Polchinski flow{]}
\item
  M. Hairer, "A theory of regularity structures," Inventiones Math. 198 (2014),
  269-504. {[}Section 9: renormalization counterterms{]}
\item
  M. Gubinelli, P. Imkeller, N. Perkowski, "Paracontrolled distributions and
  singular PDEs," Forum Math. \(\Pi\) 3 (2015). {[}Proposition 6.3: explicit counterterm formula{]}
\end{itemize}

\hypertarget{key-references-from-futon6-corpus}{%
\subsection{Key References from futon6 corpus}\label{key-references-from-futon6-corpus}}

\begin{itemize}
\tightlist
\item
  PlanetMath: "distribution" --- distributions on manifolds
\item
  PlanetMath: "Sobolev space" --- Cameron-Martin space is \(H^1\)
\item
  PlanetMath: "Gaussian measure" --- reference measure \(\mu_0\)
\item
  PlanetMath: "Radon-Nikodym theorem" --- absolute continuity and RN derivatives
\end{itemize}
