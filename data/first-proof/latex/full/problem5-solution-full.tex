\hypertarget{problem-5-o-slice-connectivity-via-geometric-fixed-points}{%
\section{Problem 5: O-Slice Connectivity via Geometric Fixed Points}\label{problem-5-o-slice-connectivity-via-geometric-fixed-points}}

\hypertarget{problem-statement}{%
\subsection{Problem Statement}\label{problem-statement}}

Fix a finite group \texttt{G}. Let \texttt{O} denote an incomplete transfer system
associated to an \(N_{infinity}\) operad. Define a slice filtration on \(SH^{G}\)
adapted to \texttt{O} and characterize \texttt{O}-slice connectivity of a connective
\texttt{G}-spectrum using geometric fixed points.

\hypertarget{short-answer}{%
\subsection{Short Answer}\label{short-answer}}

Use the classical regular-slice cells, but restrict subgroup indexing to the
family \(F_{O}\) of subgroups allowed by the transfer system. For \(F_{O}\)-local
connective \texttt{G}-spectra (those with geometric isotropy in \(F_{O}\)), the
connectivity test is the same geometric-fixed-point criterion as Hill-Yarnall,
but quantified only over \(H \in F_{O}\).

\hypertarget{confidence}{%
\subsection{Confidence}\label{confidence}}

Medium. The corrected formulation is structurally consistent with known slice
and geometric-fixed-point machinery, and avoids the nonstandard
\(\rho_{H}^{O}\) definition from the previous draft.

\hypertarget{important-correction}{%
\subsection{Important Correction}\label{important-correction}}

The previous draft used an \texttt{O}-regular representation \(\rho_{H}^{O}\). That is not
standard in the transfer-system / \(N_{infinity}\) literature. Transfer/indexing
systems control admissible transfers (and admissible finite \texttt{H}-sets), not a
replacement for the regular representation in slice-cell definitions.

\hypertarget{working-definition-transfer-restricted-regular-slice}{%
\subsection{Working Definition (transfer-restricted regular slice)}\label{working-definition-transfer-restricted-regular-slice}}

Let \(T_{O}\) be the transfer system associated to \texttt{O}. Define the admissible
subgroup family

\[F_{O} = { H \le G : the transfer e \to H is allowed \in T_{O} }.\]

\(F_{O}\) is a family of subgroups in the standard sense: it is closed under
conjugation and passage to subgroups (both inherited from the indexing-system
axioms of Blumberg-Hill, specifically the conditions on admissible \texttt{H}-sets
in their Theorem 1.5 / Section 7).

Define \(\tau_{\ge n}^{O}\) as the localizing subcategory of \(SH^{G}\) generated by
regular slice cells

\[G_+ wedge_{H} S^{k \rho_{H}}\]

with \(H \in F_{O}\) and \(k|H| \ge n\).

This is the same generating template as regular slice filtration, with subgroup
indexing restricted by \texttt{O}.

\textbf{Filtration verification.} Monotonicity (\(\tau_{\ge n+\mNumber{1}}^{O} subseteq \tau_{\ge n}^{O}\))
is immediate: the generating set for \(n+\mNumber{1}\) is a subset of that for \texttt{n}
(if \(k|H| \ge n+\mNumber{1}\) then \(k|H| \ge n\)). Exhaustiveness
(\(bigcup_{n} \tau_{\ge n}^{O} = SH^{G}_{F_{O}-local}\)) holds because every \(F_{O}\)-local
object is built from cells with finite \texttt{k\textbar{}H\textbar{}}. Compatibility with suspension
is inherited from the ambient regular slice filtration (Hill-Yarnall,
Proposition 1.3).

\textbf{\(F_{O}\)-locality convention.} Say a \texttt{G}-spectrum \texttt{X} is \emph{\(F_{O}\)-local} if
\(\Phi^{K} x = \mNumber{0}\) for all \texttt{K} not in \(F_{O}\) (equivalently, \texttt{X} has geometric
isotropy contained in \(F_{O}\)). The characterization theorem below applies to
\(F_{O}\)-local connective spectra.

Reference anchor:

\begin{itemize}
\tightlist
\item
  Regular slice generators: Hill-Yarnall, Definition 1.1 (arXiv:1703.10526).
\item
  Family/indexing-system closure: Blumberg-Hill, Theorem 1.5 (arXiv:1309.1750).
\end{itemize}

\hypertarget{characterization-theorem-subgroup-family-level}{%
\subsection{Characterization Theorem (subgroup-family level)}\label{characterization-theorem-subgroup-family-level}}

Let \texttt{X} be a connective, \(F_{O}\)-local \texttt{G}-spectrum. Then

\[X \in \tau_{\ge n}^{O}\]

iff for every \(H \in F_{O}\),

\[\Phi^{H} x is (ceil(n/|H|) - \mNumber{1})-connected.\]

Equivalent homotopy-group form:

\[\pi_{i}(\Phi^{H} X) = \mNumber{0} for all i < ceil(n/|H|), for all H \in F_{O}.\]

The \(F_{O}\)-locality hypothesis is essential: for \texttt{K} not in \(F_{O}\), the condition
\(\Phi^{K} x = \mNumber{0}\) is assumed (not tested), ensuring that the restricted generators
span exactly the right localizing subcategory.

Reference anchor:

\begin{itemize}
\tightlist
\item
  The geometric-fixed-point characterization for the full regular slice is
  Hill-Yarnall Theorem A / Theorem 2.5 (arXiv:1703.10526).
\item
  Family localization in equivariant stable homotopy: see e.g. the isotropy
  separation sequence in HHR, Section 4.
\end{itemize}

\hypertarget{proof-sketch}{%
\subsection{Proof Sketch}\label{proof-sketch}}

\textbf{Forward direction} (\(X \in \tau_{\ge n}^{O}\) implies connectivity bounds):

\begin{enumerate}
\def\labelenumi{\arabic{enumi}.}
\tightlist
\item
  By construction, \(\tau_{\ge n}^{O}\) is the localizing subcategory generated by
  cells \(G_+ wedge_{H} S^{k \rho_{H}}\) with \(H \in F_{O}\), \(k|H| \ge n\).
\item
  For each \(H \in F_{O}\), the geometric fixed points of a generating cell satisfy
  \(\Phi^{H}(G_+ wedge_{H} S^{k \rho_{H}}) ~= W_{G}(H)_+ wedge S^{k}\) (a finite wedge of
  \(S^{k}\)\textquotesingle s indexed by the Weyl group \(W_{G}(H) = N_{G}(H)/H\)).
\item
  Maps from these generators into \texttt{X} detect \(\pi_{k}(\Phi^{H} X)\): the Weyl-group
  multiplicity does not affect vanishing (a map from a finite wedge of \(S^{k}\)\textquotesingle s
  is zero iff each component is zero iff \(\pi_{k} = \mNumber{0}\)).
\item
  Vanishing for all generators with \(k|H| < n\) gives \(\pi_{i}(\Phi^{H} X) = \mNumber{0}\)
  for \(i < ceil(n/|H|)\), for each \(H \in F_{O}\).
\end{enumerate}

\textbf{Reverse direction} (connectivity bounds imply \(X \in \tau_{\ge n}^{O}\)):

\begin{enumerate}
\def\labelenumi{\arabic{enumi}.}
\setcounter{enumi}{4}
\tightlist
\item
  Since \texttt{X} is \(F_{O}\)-local, \(\Phi^{K} x = \mNumber{0}\) for \texttt{K} not in \(F_{O}\). By the
  isotropy separation sequence, the geometric isotropy of \texttt{X} is contained in
  \(F_{O}\), so \texttt{X} is built from cells indexed by subgroups in \(F_{O}\).
\item
  The connectivity bounds \(\pi_{i}(\Phi^{H} X) = \mNumber{0}\) for \(i < ceil(n/|H|)\) and
  all \(H \in F_{O}\), combined with \(F_{O}\)-locality, place \texttt{X} in the localizing
  subcategory generated by the \(F_{O}\)-restricted regular slice cells with
  \(k|H| \ge n\). This is the family-localized analogue of the Hill-Yarnall
  Theorem 2.5 reverse direction.
\end{enumerate}

\textbf{Reduction lemma for step 6.} The Hill-Yarnall reverse argument (Section 2
of arXiv:1703.10526) constructs Postnikov sections using the slice cells as
building blocks and the geometric fixed-point detection criterion to control
connectivity at each stage. In the \(F_{O}\)-local setting:

(1) The building blocks are the \(F_{O}\)-restricted slice cells (same
construction, restricted indexing to \(H \in F_{O}\)).
(2) The detection criterion is the same \(\Phi^{H}\) test for \(H \in F_{O}\)
(with \(\Phi^{K} x = \mNumber{0}\) for \(K \notin F_{O}\) by the locality assumption).
(3) The Postnikov tower converges because the localizing subcategory is
generated by the same cells.

The only input from the ambient theory is the isotropy separation sequence
(HHR, Section 4), which holds in any localizing subcategory of \(SH^{G}\).
This establishes that the reverse direction transfers from the full regular
slice setting to the \(F_{O}\)-local setting.

\textbf{Note:} Without \(F_{O}\)-locality, the reverse direction fails. For example,
with \(F_{O} = {e}\) (trivial transfer), an arbitrary \texttt{G}-spectrum \texttt{X} with
\(\Phi^{e} X\) highly connected need not lie in \(\tau_{\ge n}^{O}\) if it has
nontrivial geometric isotropy at proper subgroups.

\hypertarget{special-cases}{%
\subsection{Special Cases}\label{special-cases}}

\begin{enumerate}
\def\labelenumi{\arabic{enumi}.}
\tightlist
\item
  \textbf{Complete transfer system} (\(F_{O} = all subgroups\)): every \texttt{G}-spectrum is
  \(F_{O}\)-local, so the hypothesis is vacuous. Recovers the usual Hill-Yarnall
  regular slice criterion.
\item
  \textbf{Trivial transfer system} (\(F_{O} = {e}\)): \(F_{O}\)-local means \(\Phi^{H} x = \mNumber{0}\)
  for all \(H \neq e\), i.e., \texttt{X} has trivial geometric isotropy. For such \texttt{X},
  the criterion reduces to ordinary Postnikov connectivity of the underlying
  nonequivariant spectrum \(\Phi^{e} X\).
\item
  \textbf{Intermediate transfer systems}: for \(F_{O}\)-local \texttt{X}, the criterion
  interpolates by checking \(\Phi^{H}\) connectivity only on the allowed family.
\end{enumerate}

\hypertarget{scope--caveat}{%
\subsection{Scope / Caveat}\label{scope--caveat}}

\textbf{Important limitation.} This is a subgroup-family-indexed formulation,
not a full indexing-system characterization. The full indexing-system
formalism of Blumberg-Hill involves admissible finite \texttt{H}-sets, not just
admissible subgroups \texttt{H}. Our \(F_{O} = {H : e \to H admissible}\) extracts the
subgroup-level data; it does not use the finer \(K \to H\) admissibility
structure for \(K \neq e\). The full indexing-system formulation --- characterizing
O-slice connectivity in terms of the complete admissible set data --- is an
open extension of this result.

The \(F_{O}\)-locality hypothesis was added to close the reverse direction of the
characterization theorem. This is analogous to how classical slice theorems
work in full \(SH^{G}\) (where every spectrum is \texttt{F}-local for \texttt{F} = all
subgroups). For intermediate transfer systems, the hypothesis is nontrivial
and should be verified for the spectra of interest.

Literature positioning:

\begin{itemize}
\tightlist
\item
  \(N_{infinity}\) operads and admissibility data: Blumberg-Hill, Def. 1.1,
  Thm. 1.2, Thm. 1.5 / Sec. 7 (arXiv:1309.1750).
\item
  Regular slice generators and detection: Hill-Yarnall, Def. 1.1, Thm. 2.5
  (arXiv:1703.10526).
\item
  Isotropy separation and family localization: HHR, Section 4.
\end{itemize}

\hypertarget{references}{%
\subsection{References}\label{references}}

\begin{itemize}
\tightlist
\item
  Michael A. Hill, Michael J. Hopkins, Douglas C. Ravenel, \emph{Equivariant stable
  homotopy theory and the Kervaire invariant problem} (slice framework).
\item
  Andrew J. Blumberg, Michael A. Hill, \emph{Operadic multiplications in equivariant
  spectra, norms, and transfers}, arXiv:1309.1750:
  Definition 1.1, Theorem 1.2.
\item
  Michael A. Hill, Carolyn Yarnall, \emph{A new formulation of the equivariant slice
  filtration with applications to \(C_{p}\)-slices}, arXiv:1703.10526:
  Definition 1.1, Theorem A, Theorem 2.5.
\item
  \texttt{data/first-proof/library-research-findings.md}
\end{itemize}
