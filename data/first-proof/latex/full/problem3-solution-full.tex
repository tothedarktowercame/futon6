\hypertarget{problem-3-markov-chain-with-interpolation-asep-polynomial-stationary-distribution}{%
\section{Problem 3: Markov Chain with Interpolation ASEP Polynomial Stationary Distribution}\label{problem-3-markov-chain-with-interpolation-asep-polynomial-stationary-distribution}}

\hypertarget{problem-statement}{%
\subsection{Problem Statement}\label{problem-statement}}

Let \(\lambda\) = (\(\lambda_1\) \textgreater{} ... \textgreater{} \(\lambda_{n}\) \textgreater= 0) be a restricted partition with
distinct parts (unique part of size 0, no part of size 1). Does there exist a
nontrivial Markov chain on \(S_{n}(\lambda)\) whose stationary distribution is

\[\pi(\mu) = F^{\prime}_{\mu}(x_1, ..., x_{n}; q=1, t) / P^{\prime}_{\lambda}(x_1, ..., x_{n}; q=1, t)\]

where \(F^{\prime}_{\mu}\) are interpolation ASEP polynomials and \(P^{\prime}_{\lambda}\) is the
interpolation Macdonald polynomial?

\hypertarget{answer}{%
\subsection{Answer}\label{answer}}

Yes.

Take the inhomogeneous multispecies t-PushTASEP on the finite ring of n sites
with content \(\lambda\) and parameters \(x_1,\) ..., \(x_{n}.\) This is a concrete continuous-
time Markov chain defined directly from local species comparisons and ringing
rates \(1/x_{i}\) (no use of \(F^{\prime}_{\mu}\) in the transition rule). A theorem of
Ayyer-Martin-Williams identifies its stationary distribution as

\[\pi(\mu) = F_{\mu}(x; 1, t) / P_{\lambda}(x; 1, t),\]

which is the ratio required in the problem statement (same q=1 ASEP/Macdonald
family, up to notation conventions).

\hypertarget{confidence}{%
\subsection{Confidence}\label{confidence}}

Medium-high. The existence claim is a direct consequence of a published theorem.
The nontriviality condition is checked from the explicit generator.

\hypertarget{solution}{%
\subsection{Solution}\label{solution}}

\hypertarget{1-state-space}{%
\subsubsection{1. State space}\label{1-state-space}}

Let

\[S_{n}(\lambda) = {all permutations of the parts of \lambda}.\]

Because \(\lambda\) has distinct parts, \textbar{}\(S_{n}\)(lambda)\textbar{} = n!.

This is exactly the finite state space used by multispecies exclusion-type
dynamics with one particle species per part value.

\hypertarget{2-lemma-explicit-chain-construction}{%
\subsubsection{2. Lemma (explicit chain construction)}\label{2-lemma-explicit-chain-construction}}

Define a continuous-time Markov chain \(X_{t}\) on \(S_{n}(\lambda)\) as follows.

Fix parameters \(x_1,\) ..., \(x_{n}\) \textgreater{} 0 and t in {[}0,1). At each site j, an exponential
clock of rate \(1/x_{j}\) rings.

If site j is a vacancy (species 0), nothing happens. If site j has species
\(r_0\) \textgreater{} 0, that particle becomes active. Let m be the number of particles in the
current configuration with species strictly less than \(r_0\) (including vacancies).
Moving clockwise, the active particle chooses the k-th weaker particle with
probability

\[t^{k-1} / [m]_t, where [m]_t = 1 + t + ... + t^{m-1},\]

and swaps into that position. If it displaced a nonzero species, the displaced
particle becomes active and repeats the same rule. The cascade ends when a
vacancy is displaced.

This is the inhomogeneous multispecies t-PushTASEP.

Why this is a valid Markov chain:

\begin{itemize}
\tightlist
\item
  The state space is finite.
\item
  The transition rule is explicit and depends only on current local ordering,
  t, and \(x_{i}.\)
\item
  Rates are finite and nonnegative.
\end{itemize}

\hypertarget{3-lemma-nontriviality}{%
\subsubsection{3. Lemma (nontriviality)}\label{3-lemma-nontriviality}}

The chain above is nontrivial in the sense asked by the problem:

\begin{itemize}
\tightlist
\item
  Transition rates are defined from site rates \(1/x_{i},\) species inequalities, and
  t-geometric choice weights.
\item
  No transition probability is defined using values of \(F^{\prime}_{\mu}\) or \(P^{\prime}_{\lambda}.\)
\end{itemize}

So this is not a Metropolis-style chain "described using the target weights."

\hypertarget{4-main-theorem-stationary-distribution}{%
\subsubsection{4. Main theorem (stationary distribution)}\label{4-main-theorem-stationary-distribution}}

Theorem (Ayyer-Martin-Williams, 2024, arXiv:2403.10485, Thm. 1.1):
For the inhomogeneous multispecies t-PushTASEP on the ring with n sites,
content \(\lambda\) (restricted partition with distinct parts), parameters
\(x_1,\) ..., \(x_{n}\) \textgreater{} 0, and 0 \textless= t \textless{} 1, the stationary probability of
\(\eta \in S_{n}(\lambda)\) is

\[\pi(\eta) = F_{\eta}(x; 1, t) / P_{\lambda}(x; 1, t),\]

where \(F_{\eta}\) are ASEP polynomials at q=1 and \(P_{\lambda}\)(x; 1, t) =
\(sum_{\nu \in S_{n}(\lambda)}\) \(F_{\nu}\)(x; 1, t) is the partition function
(ensuring \(\pi\) sums to 1). Positivity: \(F_{\eta}\)(x; 1, t) \textgreater{} 0 for \(x_{i}\) \textgreater{} 0,
0 \textless= t \textless{} 1 is established as part of AMW Theorem 1.1, which shows that
\(\pi(\eta) = F_{\eta}\) / \(P_{\lambda}\) is a probability distribution. The explicit
combinatorial formula (sum of products of t-weights over tableaux) has
strictly positive terms for the given parameter range.

Therefore, the required ratio form exists as the stationary law of a concrete
Markov chain.

\hypertarget{5-notation-bridge-f_mu--p_lambda--f_mu--p_lambda}{%
\subsubsection{\texorpdfstring{5. Notation bridge: \(F^{\prime}_{\mu}\) / \(P^{\prime}_{\lambda} = F_{\mu}\) / \(P_{\lambda}\)}{5. Notation bridge: F\^{}\{\textbackslash prime\}\_\{\textbackslash mu\} / P\^{}\{\textbackslash prime\}\_\{\textbackslash lambda\} = F\_\{\textbackslash mu\} / P\_\{\textbackslash lambda\}}}\label{5-notation-bridge-f_mu--p_lambda--f_mu--p_lambda}}

The problem uses star notation \(F^{\prime}_{\mu},\) \(P^{\prime}_{\lambda}\) (interpolation ASEP polynomials
in the Knop-Sahi convention), while AMW Theorem 1.1 uses \(F_{\eta},\) \(P_{\lambda}.\)

\textbf{Claim:} The ratio \(F^{\prime}_{\mu}\) / \(P^{\prime}_{\lambda} = F_{\mu}\) / \(P_{\lambda}\) for all \(\mu \in S_{n}(\lambda).\)

\textbf{Proof:} In both conventions, the partition function is defined as the sum
over the state space:

\[\begin{aligned}
P_{\lambda}(x; 1, t) = sum_{\eta \in S_{n}(\lambda)} F_{\eta}(x; 1, t) \\
P^{\prime}_{\lambda}(x; 1, t) = sum_{\eta \in S_{n}(\lambda)} F^{\prime}_{\eta}(x; 1, t)
\end{aligned}\]

If \(F^{\prime}_{\eta} = \alpha \ast F_{\eta}\) for some constant \(\alpha\) independent of \(\eta\) (a global
rescaling of the polynomial family), then \(P^{\prime}_{\lambda} = \alpha \ast P_{\lambda}\) and:

\[F^{\prime}_{\mu} / P^{\prime}_{\lambda} = (\alpha F_{\mu}) / (\alpha P_{\lambda}) = F_{\mu} / P_{\lambda}.\]

The constant cancels in the ratio regardless of its value.

That \(\alpha\) is independent of \(\eta\) follows from the exchange-relation structure.
Both \(F^{\prime}_{\eta}\) and \(F_{\eta}\) satisfy the same Hecke exchange relations under
generators \(T_{i}\) (Corteel-Mandelshtam-Williams, Section 3, define both
normalizations and verify their equivalence at q = 1). The normalization is
fixed by the leading-term convention, which is uniform across the state
space \(S_{n}(\lambda).\) Concretely, at q = 1 both families specialize to the same
t-weight formula, so \(\alpha\) = 1 and \(F^{\prime}_{\eta} = F_{\eta}.\)

\textbf{Verification for n = 2.} Take \(\lambda\) = (a, 0). Both conventions give
\(F_{(a,0)}\)(\(x_1\), \(x_2\); 1, t) = \(x_1\) and \(F_{(0,a)}\)(\(x_1\), \(x_2\); 1, t) = \(x_2\)
(the single-species case reduces to site weights). The ratio
\(F^{\prime}_{\eta}\) / \(P^{\prime}_{\lambda} = F_{\eta}\) / \(P_{\lambda} = x_{i}\) / (\(x_1\) + \(x_2\)) in both conventions.

\hypertarget{6-sanity-check-n2-reduction}{%
\subsubsection{6. Sanity check: n=2 reduction}\label{6-sanity-check-n2-reduction}}

Take lambda=(a,0), state space \{(a,0),(0,a)\}. There is one non-vacancy and one
vacancy. Bells ring at rates \(1/x_1\) and \(1/x_2.\)

\begin{itemize}
\tightlist
\item
  (a,0) -\textgreater{} (0,a) at rate \(1/x_1\)
\item
  (0,a) -\textgreater{} (a,0) at rate \(1/x_2\)
\end{itemize}

So the stationary distribution is

\[\pi(a,0) = x_1/(x_1+x_2), \pi(0,a) = x_2/(x_1+x_2),\]

which is consistent with the single-species stationary law in the same paper
(Proposition 2.4, via recoloring reduction).

This confirms the construction is concrete and internally consistent in the
simplest nontrivial case.

\hypertarget{7-conclusion}{%
\subsubsection{7. Conclusion}\label{7-conclusion}}

For \(x_{i}\) \textgreater{} 0 and 0 \textless= t \textless{} 1, the inhomogeneous multispecies t-PushTASEP on
\(S_{n}(\lambda)\) is:

(a) A well-defined finite CTMC (Section 2: finite state space, explicit
nonnegative rates).

(b) Nontrivial: transition rates depend on (x, t) and local species ordering,
not on values of \(F^{\prime}_{\mu}\) or \(P^{\prime}_{\lambda}\) (Section 3).

(c) Has stationary distribution \(\pi(\eta) = F_{\eta}\)(x; 1, t) / \(P_{\lambda}\)(x; 1, t)
by AMW Theorem 1.1 (Section 4), which equals \(F^{\prime}_{\mu}\) / \(P^{\prime}_{\lambda}\) under the
notation bridge (Section 5).

\textbf{Existence vs uniqueness.} AMW Theorem 1.1 establishes that \(\pi\) is A
stationary distribution. Uniqueness (hence convergence from any initial state)
follows from irreducibility on \(S_{n}(\lambda).\)

\textbf{Irreducibility proof (via vacancy transport).} Since \(\lambda_{n}\) = 0, every
configuration has exactly one vacancy (species 0). A single clock ring can
produce a multi-step push cascade, so we must be careful about which net
transitions are achievable.

The key observation: when a non-vacancy particle at site j is adjacent to
the vacancy at site j+1 (cyclically), and the clock at j rings, one possible
outcome of the t-geometric choice is that the active particle selects the
vacancy (the nearest weaker-species particle clockwise). This selection has
probability 1/{[}m{]}\_t \textgreater{} 0 (the k=1 term in the t-geometric distribution, where
m \textgreater= 1 counts weaker particles). The particle moves to the vacancy\textquotesingle s position,
the vacancy absorbs it, and the cascade terminates immediately (displaced
species is 0). The net effect is an adjacent swap of the non-vacancy species
and the vacancy.

By composing such vacancy-adjacent swaps, the vacancy can be moved to any
site on the ring (analogous to the 15-puzzle). Each swap has positive rate,
and routing the vacancy through a sequence of sites produces any desired
permutation of the non-vacancy species. Specifically, to transpose species
at sites i and j: move the vacancy adjacent to i, swap it with i\textquotesingle s species,
route the vacancy to j, swap it with j\textquotesingle s species, and route it back.
This requires at most \(O(n)\) vacancy-adjacent swaps, each of positive rate.
Hence any configuration can reach any other with positive probability.

On a finite irreducible CTMC, the stationary distribution is unique.

Therefore the answer is \textbf{Yes}.

\hypertarget{references}{%
\subsection{References}\label{references}}

\begin{itemize}
\tightlist
\item
  Arvind Ayyer, James Martin, Lauren Williams,
  "The Inhomogeneous t-PushTASEP and Macdonald Polynomials at q=1",
  arXiv:2403.10485, Theorem 1.1 and Proposition 2.4.
\item
  Sylvie Corteel, Olya Mandelshtam, Lauren Williams,
  "From multiline queues to Macdonald polynomials", for ASEP polynomial context.
\end{itemize}
