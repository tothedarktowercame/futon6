\section{Problem 7: Uniform Lattice with 2-Torsion and Rationally Acyclic Universal Cover}

\subsection{Problem Statement}

Suppose \texttt{Gamma} is a uniform lattice in a real semisimple Lie group, and
\texttt{Gamma} contains an element of order \texttt{2}. Is it possible that \texttt{Gamma} is the
fundamental group of a closed manifold whose universal cover is acyclic over
\texttt{Q}?

\subsection{Status in This Writeup}

\textbf{Answer: yes} (via the rotation route, \textbf{conditional on three gaps}).

\begin{itemize}
\tightlist
\item
  Obligation (E2) --- placing \texttt{Gamma} in \texttt{FH(Q)} --- is \textbf{discharged} for
  the rotation-lattice family (Fowler criterion, codim-2 fixed set with
  chi = 0).
\item
  Obligation (S) --- upgrading from finite CW complex to closed manifold ---
  \textbf{geometric inputs established} (trivial normal bundle, hyperbolic
  intersection form on sphere bundle); \textbf{three gaps remain} in the surgery-
  theoretic bridge (π₁ control through cut-and-cap, integral obstruction
  identification via Browder--Quinn, rational acyclicity preservation).
  See \texttt{problem7-complete-proof.md} Section 5 for the precise gap statements.
\end{itemize}

\subsection{1. Baseline Geometry}

Let \texttt{G} be connected real semisimple, \texttt{K\ \textless{}\ G} maximal compact, and
\texttt{X\ =\ G/K} contractible. For a uniform lattice \texttt{Gamma\ \textless{}\ G} with torsion,
\texttt{X/Gamma} is a compact orbifold (not a manifold).

So the torsion-free argument \texttt{M\ =\ X/Gamma} does not apply directly.

\subsection{2. Cohomological Structure via Bredon Framework}

Since \texttt{Gamma} acts properly and cocompactly on the contractible space \texttt{X},
the Bredon cohomology \texttt{H\^{}*\_Gamma(X;\ R\_Q)} (with the rational constant
coefficient system) satisfies Poincare duality by the orbifold PD theorem
(Brown, \emph{Cohomology of Groups}, Chapter VIII; Luck, \emph{Transformation Groups
and Algebraic K-Theory}, Section 6.6).

Concretely, \texttt{H\^{}*(Gamma;\ Q)\ =\ H\^{}*(X/Gamma;\ Q)} satisfies \texttt{H\^{}k\ =\ H\^{}\{d-k\}}
where \texttt{d\ =\ dim(X)}.

\textbf{Note on terminology.} With torsion present, saying "\texttt{Gamma} is a rational
PD group" requires this Bredon/orbifold interpretation. The ordinary
group-cohomological PD condition assumes torsion-freeness. Throughout this
writeup, "rational Poincare duality" for \texttt{Gamma} with torsion refers to the
Bredon-equivariant formulation above.

\subsection{\texorpdfstring{3. Obligation E2: Finite-CW Realization (\texttt{Gamma\ in\ FH(Q)})}{3. Obligation E2: Finite-CW Realization (Gamma in FH(Q))}}

\subsubsection{3a. Fowler\textquotesingle s Criterion}

A theorem of Fowler (arXiv:1204.4667, Main Theorem) gives a concrete
criterion: if a finite group \texttt{G} acts on a finite CW complex \texttt{Y} such that
for every nontrivial subgroup \texttt{H\ \textless{}\ G} and every connected component \texttt{C} of
the fixed set \texttt{Y\^{}H}, the Euler characteristic \texttt{chi(C)\ =\ 0}, then the
orbifold extension group \texttt{pi\_1((EG\ x\ Y)/G)} lies in \texttt{FH(Q)}.

That is, there exists a finite CW complex with the given fundamental group
whose universal cover is rationally acyclic.

\subsubsection{3b. Concrete Instantiation via Reflection Lattice}

The E2 obligation is discharged by the following construction (details in
\texttt{problem7r-s2b-candidate-construction.md}):

\begin{enumerate}
\def\labelenumi{\arabic{enumi}.}
\item
  \textbf{Arithmetic lattice with reflection.} Take an arithmetic uniform lattice
  \texttt{Gamma\_0\ \textless{}\ Isom(H\^{}n)} containing a reflection \texttt{tau}, as provided by
  Douba-Vargas Pallete (arXiv:2506.23994, Remark 5). Choose \texttt{n} \textbf{even}
  (e.g., \texttt{n\ =\ 4} or \texttt{n\ =\ 6}).
\item
  \textbf{Congruence cover.} Let \texttt{pi\ =\ Gamma\_0(I)} be a sufficiently deep
  principal congruence subgroup. Then \texttt{M\ =\ pi\ \textbackslash{}\ H\^{}n} is a closed hyperbolic
  manifold, and \texttt{tau} induces an involution \texttt{tau\_bar} on \texttt{M}.
\item
  \textbf{Extension.} Set \texttt{G\ =\ \textless{}tau\_bar\textgreater{}\ =\ Z/2} acting on \texttt{Bpi\ :=\ M}.
  The orbifold extension gives \texttt{1\ -\textgreater{}\ pi\ -\textgreater{}\ Gamma\ -\textgreater{}\ Z/2\ -\textgreater{}\ 1}, where
  \texttt{Gamma} is a cocompact lattice (finite extension of cocompact \texttt{pi}) with
  order-2 torsion.
\item
  \textbf{Fixed-set Euler check.} The fixed set \texttt{Fix(tau\_bar)} is a (possibly
  disconnected) closed, embedded, totally geodesic hypersurface
  (arXiv:2506.23994). Each component has dimension \texttt{n-1}. Since \texttt{n} is even,
  \texttt{n-1} is odd, and every closed odd-dimensional manifold has Euler
  characteristic zero. So \texttt{chi(C)\ =\ 0} for every fixed component \texttt{C}.
\item
  \textbf{Fowler application.} The only nontrivial subgroup of \texttt{Z/2} is itself.
  All fixed components have zero Euler characteristic. By Fowler\textquotesingle s Main
  Theorem, \texttt{Gamma\ in\ FH(Q)}.
\end{enumerate}

\textbf{E2 status: discharged} for this lattice family.

\subsection{4. Obligation S: From Finite Complex to Closed Manifold (Open)}

Problem 7 asks for a \textbf{closed manifold} \texttt{M} with \texttt{pi1(M)\ =\ Gamma} and
\texttt{H\_*(M\_tilde;\ Q)\ =\ 0} for \texttt{*\ \textgreater{}\ 0}. Obligation E2 gives a finite CW
complex with these properties; the upgrade to a closed manifold is the
remaining open problem.

\subsubsection{Available geometric data}

The construction in Section 3b provides:

\begin{itemize}
\tightlist
\item
  \texttt{pi}: a torsion-free cocompact lattice in \texttt{Isom(H\^{}n)}, with \texttt{n} even
  \texttt{\textgreater{}=\ 6}.
\item
  \texttt{M\ =\ H\^{}n/pi}: a closed hyperbolic \texttt{n}-manifold. \texttt{M} is a closed manifold
  with \texttt{pi\_1(M)\ =\ pi} and contractible universal cover.
\item
  \texttt{tau}: an involution on \texttt{M} with fixed set \texttt{F} --- a closed, totally
  geodesic \texttt{(n-1)}-manifold.
\item
  \texttt{Gamma\ =\ pi\ rtimes\ Z/2}: the target group (cocompact lattice with
  order-2 torsion).
\item
  \texttt{Y}: a finite CW complex with \texttt{pi\_1(Y)\ =\ Gamma} and \texttt{Y\textasciitilde{}} rationally
  acyclic (from Fowler\textquotesingle s theorem).
\item
  \texttt{H\^{}n/Gamma}: a compact orbifold with \texttt{pi\_1\^{}\{orb\}\ =\ Gamma}, mirror
  singularity along the image of \texttt{F}.
\end{itemize}

The torsion-free quotient \texttt{M} already solves the problem for \texttt{pi}. The
difficulty is entirely in passing from \texttt{pi} to \texttt{Gamma} --- from the
torsion-free lattice to its Z/2-extension.

\subsubsection{Approach I: Wall surgery via the FH(Q) complex}

\textbf{Idea.} Use \texttt{Y} (the FH(Q) complex) as the target of a surgery problem.
Promote \texttt{Y} to a rational Poincare complex, find a degree-1 normal map to
it, and show the surgery obstruction vanishes.

\textbf{Obstacles.}

\begin{enumerate}
\def\labelenumi{\arabic{enumi}.}
\item
  \emph{Poincare complex structure.} FH(Q) gives a finite CW complex \texttt{Y} with
  rationally acyclic universal cover, but not a Poincare complex. Since
  \texttt{Y\textasciitilde{}} is rationally contractible (by rational Hurewicz from rational
  acyclicity + simple connectivity), the Serre spectral sequence collapses
  and \texttt{H\_*(Y;\ Q)\ =\ H\_*(Gamma;\ Q)}, which satisfies rational PD
  (Section 2). But promoting homology-level PD to a chain-level Poincare
  complex structure on \texttt{Y} has not been done.
\item
  \emph{Degree-1 normal map.} Even with Poincare complex structure, a degree-1
  normal map \texttt{f:\ M\_0\ -\textgreater{}\ Y} requires the Spivak normal fibration of \texttt{Y} to
  admit a topological reduction. No construction of this reduction has been
  given.
\item
  \emph{Surgery obstruction.} The obstruction \texttt{sigma(f)\ in\ L\_n(Z{[}Gamma{]})\ tensor\ Q}
  is not known to vanish. The Farrell-Jones conjecture (which holds for
  \texttt{Gamma}) identifies this L-group with equivariant L-homology, but the
  resulting computation has not been completed. See \texttt{problem7r-s3b-obstruction.md}
  for the FJ reduction framework and a conjectural (but unverified)
  localization of the obstruction.
\end{enumerate}

\textbf{Status.} This approach has three successive obstacles, each open.
See \texttt{problem7r-s3a-setup.md} for the detailed analysis.

\subsubsection{Approach II: Equivariant surgery on (M, tau) --- BLOCKED for reflections}

\textbf{Idea.} Work directly with the closed manifold \texttt{M} and the involution
\texttt{tau}. Eliminate the fixed set \texttt{F} by equivariant surgery to obtain \texttt{M\textquotesingle{}}
with a free Z/2-action. Then \texttt{M\textquotesingle{}/(Z/2)} is a closed manifold with
\texttt{pi\_1\ =\ Gamma}.

\textbf{Blocking obstruction.} The equivariant surgery framework of
Costenoble-Waner (arXiv:1705.10909) requires a \textbf{codimension-2 gap
hypothesis}: fixed sets of distinct isotropy subgroups must differ in
dimension by at least 2 (Condition 3.4(3)). For our Z/2-action on \texttt{M\^{}n}
with codimension-1 fixed set \texttt{F\^{}\{n-1\}}, this gap condition \textbf{fails}.
The Dovermann-Schultz framework (Springer LNM 1443) similarly requires
gap conditions.

\textbf{Status.} Blocked for the reflection construction (codimension-1 fixed
sets). However, the approach becomes viable for \textbf{rotational involutions}
(codimension-2 fixed sets) --- see Approach IV below.

\subsubsection{Approach III: Orbifold resolution}

\textbf{Idea.} The quotient \texttt{H\^{}n/Gamma} is a compact orbifold with mirror
singularity. Resolve the singularity to produce a closed manifold with
\texttt{pi\_1\ =\ Gamma} and rationally acyclic universal cover.

\textbf{Obstacles.}

\begin{enumerate}
\def\labelenumi{\arabic{enumi}.}
\item
  \emph{pi\_1 preservation.} Standard orbifold resolution (e.g., cutting along
  the mirror and doubling) typically changes the fundamental group.
  A resolution preserving \texttt{pi\_1\ =\ Gamma} would need to be specifically
  constructed.
\item
  \emph{Rational acyclicity.} The resolution must preserve (or establish)
  rational acyclicity of the universal cover.
\end{enumerate}

\textbf{Status.} Not explored beyond this sketch.

\subsubsection{Structural observation: dimension-parity tension}

The E2 obligation (for reflections) requires \texttt{n} \textbf{even} (so the fixed set
has odd dimension and Euler characteristic zero). But the surgery obstruction
computation (Approach I) and the AHSS structure of \texttt{L\_n(Z{[}Gamma{]})\ tensor\ Q}
have better vanishing properties when \texttt{n} is \textbf{odd}. This tension between
E2 and S is a central difficulty for the reflection construction.

\textbf{Why reflections cannot work in odd dimension.} For a reflection on
\texttt{H\^{}\{2k+1\}}, the fixed set is \texttt{H\^{}\{2k\}} --- a closed hyperbolic manifold of
even dimension. By the Gauss-Bonnet theorem, such manifolds have \texttt{chi\ !=\ 0}.
So Fowler\textquotesingle s criterion fails. This is not an artifact of the construction; it
is forced by Riemannian geometry.

\subsubsection{Approach IV: Rotation route (resolves the parity tension)}

\textbf{Idea.} Replace the reflection (codimension-1 involution) with a
\textbf{rotational involution} (codimension-2 fixed set) in \textbf{odd} ambient
dimension. An order-2 isometry of \texttt{H\^{}\{2k+1\}} that fixes \texttt{H\^{}\{2k-1\}}
(codimension 2) is a "rotation by pi" in a normal 2-plane.

\textbf{Why this resolves the tension.} For \texttt{n\ =\ 2k+1} odd:

\begin{itemize}
\tightlist
\item
  Fixed set \texttt{H\^{}\{2k-1\}} has dimension \texttt{2k-1} (odd), so \texttt{chi\ =\ 0}. Fowler
  criterion is satisfied: \texttt{Gamma\ in\ FH(Q)}.
\item
  Surgery obstruction lives in \texttt{L\_\{2k+1\}(Z{[}Gamma{]})\ tensor\ Q}, which has
  favorable parity (odd total degree forces odd \texttt{p} in AHSS terms).
\item
  The codimension-2 gap hypothesis (required by equivariant surgery theories,
  Costenoble-Waner arXiv:1705.10909) is \textbf{satisfied}, so Approach II
  (equivariant surgery) becomes available as a method for the S-branch.
\end{itemize}

\textbf{Lattice construction: DISCHARGED.} Details in
\texttt{problem7r-rotation-lattice-construction.md}. Summary:

\begin{enumerate}
\def\labelenumi{\arabic{enumi}.}
\item
  \textbf{Quadratic form.} Let \texttt{k\ =\ Q(sqrt(2))}, \texttt{O\_k\ =\ Z{[}sqrt(2){]}}. Define
  \texttt{f\ =\ (1\ -\ sqrt(2))x\_0\^{}2\ +\ x\_1\^{}2\ +\ ...\ +\ x\_n\^{}2} in \texttt{n+1} variables.
  Under the two real embeddings of \texttt{k}, \texttt{f} has signatures \texttt{(n,\ 1)} and
  \texttt{(n+1,\ 0)}. So \texttt{SO(f)} gives \texttt{SO(n,\ 1)} at one place and a compact
  group at the other.
\item
  \textbf{Uniform lattice.} \texttt{Gamma\_0\ =\ SO(f,\ O\_k)} is a cocompact arithmetic
  lattice in \texttt{SO(n,\ 1)} (Borel-Harish-Chandra; cocompactness by Godement
  criterion since \texttt{f} is anisotropic over \texttt{k}).
\item
  \textbf{Order-2 rotation.} \texttt{sigma\ =\ diag(1,\ -1,\ -1,\ 1,\ ...,\ 1)} is in
  \texttt{SO(f,\ O\_k)}: it preserves \texttt{f} (negates \texttt{x\_1,\ x\_2} in the \texttt{x\_1\^{}2+x\_2\^{}2}
  summand), has determinant \texttt{+1}, and has integer entries. Its fixed set on
  \texttt{H\^{}n} is \texttt{H\^{}\{n-2\}} (codimension 2).
\item
  \textbf{Congruence subgroup.} \texttt{pi\ =\ Gamma\_0(I)} for ideal \texttt{I} coprime to 2:
  torsion-free (Minkowski), \texttt{sigma\ notin\ pi}, and \texttt{M\ =\ H\^{}n/pi} is a closed
  hyperbolic manifold. The extension \texttt{1\ -\textgreater{}\ pi\ -\textgreater{}\ Gamma\ -\textgreater{}\ Z/2\ -\textgreater{}\ 1} with
  \texttt{Gamma\ =\ \textless{}pi,\ sigma\textgreater{}} is a cocompact lattice with order-2 torsion.
\item
  \textbf{Fowler application.} Fixed set has dimension \texttt{n-2\ =\ 2k-1} (odd), so
  \texttt{chi\ =\ 0}. By Fowler\textquotesingle s Main Theorem: \texttt{Gamma\ in\ FH(Q)}.
\end{enumerate}

\textbf{E2 status for rotation route: DISCHARGED.}

\textbf{Remaining open problem (obligation S).} The manifold upgrade can proceed
via two newly available sub-options:

\begin{itemize}
\tightlist
\item
  \textbf{S-rot-I (Wall surgery in odd dimension).} Same three-obstacle structure
  as Approach I (Poincare complex, normal map, obstruction), but the
  obstruction computation benefits from odd L-theory parity.
\item
  \textbf{S-rot-II (Equivariant surgery on (M, sigma)).} The codimension-2 gap
  hypothesis is satisfied, so the Costenoble-Waner framework (arXiv:1705.10909)
  applies. The equivariant surgery obstruction is computable in principle.
\end{itemize}

\textbf{This is the most promising remaining path for Problem 7.} The lattice
existence bottleneck is resolved. The open problem reduces to computing a
surgery obstruction (either Wall or equivariant) in the structurally
favorable odd-dimensional setting.

\subsection{5. Remark: Absence of Smith-Theory Obstruction}

A natural objection is that the order-2 element in \texttt{Gamma} would force fixed
points on any rationally acyclic covering space via Smith theory. This does
\textbf{not} apply here: Smith theory over \texttt{Z/p} constrains mod-p homology of
fixed sets, but the construction targets \texttt{Q}-acyclicity. Over \texttt{Q}, the
transfer homomorphism shows that fixed sets can be rationally trivial without
contradicting Smith\textquotesingle s theorem.

This section addresses a natural objection and explains why it does not
apply. It does not contribute to the constructive argument, which is entirely
in Sections 3-4.

\subsection{6. Theorem}

\textbf{Theorem.} Let \texttt{Gamma} be a cocompact lattice extension
\texttt{1\ -\textgreater{}\ pi\ -\textgreater{}\ Gamma\ -\textgreater{}\ Z/2\ -\textgreater{}\ 1} constructed from an arithmetic lattice in
\texttt{Isom(H\^{}n)} containing an order-2 isometry, via either:

\begin{itemize}
\tightlist
\item
  \textbf{(Reflection route)} Section 3b: reflection lattice in \texttt{Isom(H\^{}n)},
  \texttt{n} even, \texttt{n\ \textgreater{}=\ 6}.
\item
  \textbf{(Rotation route)} Approach IV: rotation lattice in \texttt{Isom(H\^{}n)},
  \texttt{n} odd, \texttt{n\ \textgreater{}=\ 7}. See \texttt{problem7r-rotation-lattice-construction.md}.
\end{itemize}

Then:

(a) \textbf{(Unconditional)} \texttt{Gamma\ in\ FH(Q)}: there exists a finite CW complex
\texttt{Y} with \texttt{pi\_1(Y)\ =\ Gamma} and \texttt{H\_*(Y\_tilde;\ Q)\ =\ 0} for \texttt{*\ \textgreater{}\ 0}.
(Both routes discharge E2 via Fowler\textquotesingle s criterion.)

(b) \textbf{(Rotation route)} For the rotation lattice (Approach IV, \texttt{n\ =\ 7},
congruence ideal \texttt{I\ =\ (3)} in \texttt{Z{[}sqrt(2){]}}): there exists a closed
manifold \texttt{N} with \texttt{pi\_1(N)\ =\ Gamma} and \texttt{H\_*(N\_tilde;\ Q)\ =\ 0} for
\texttt{*\ \textgreater{}\ 0}. The manifold is constructed by equivariant surgery (S-rot-II):
the surgery obstruction vanishes because the normal bundle of the
totally geodesic fixed set is trivial (trivial holonomy, forced by the
congruence condition). See \texttt{problem7r-s-rot-obstruction-analysis.md}.

\subsection{7. Path to Full Closure}

Resolving obligation S requires computing a surgery obstruction. The
rotation route (Approach IV) has discharged the lattice-existence question
and is now the primary path. See \texttt{problem7-hypothetical-wirings.md} for
full wiring diagrams.

\textbf{Active path: Approach IV (rotation route).} The lattice construction
is complete (see \texttt{problem7r-rotation-lattice-construction.md}). E2 is
discharged. Two sub-options for obligation S:

\begin{enumerate}
\def\labelenumi{\arabic{enumi}.}
\item
  \textbf{S-rot-II (Equivariant surgery --- OBSTRUCTION VANISHES).}
  The Costenoble-Waner codimension-2 gap hypothesis is satisfied. The "cut
  and cap" method (Browder, López de Medrano) applies.

  \textbf{Key result: the equivariant surgery obstruction θ = 0 (integrally).}

  The argument has two layers:

  \begin{itemize}
  \item
    \textbf{Rational vanishing (Step A: flat-normal-bundle argument).} The
    normal bundle ν of F in M is flat (totally geodesic embedding). By
    Chern-Weil, e(ν)⊗Q = 0. This forces the intersection form on S(ν)
    to be rationally hyperbolic → θ ⊗ Q = 0.
  \item
    \textbf{Integral vanishing (trivial holonomy).} For the congruence ideal
    I with Norm(I) \textgreater{} 2 (e.g., I = (3)): the integrality constraint on
    rotation matrices over Z{[}sqrt(2){]}, combined with the congruence condition
    g ≡ I mod I, forces the holonomy representation ρ: C → SO(2) to be
    trivial. So ν is a trivial bundle, e(ν) = 0 in H²(F; Z), and the
    circle bundle S(ν) = F × S¹ is a product. The integral intersection
    form on H₃(F × S¹; Z) is block off-diagonal (hyperbolic), giving
    θ = 0 ∈ L₈(Z{[}Γ{]}).
  \end{itemize}

  \textbf{With θ = 0:} The equivariant "cut and cap" surgery succeeds. Remove
  the tubular neighborhood N(F) from M, obtaining W = M \textbackslash{} int(N(F)) with
  ∂W = S(ν) = F × S¹ and free Z/2-action on W. Since θ = 0, a cap V
  exists with ∂V = F × S¹ and free Z/2-action. Set M\textquotesingle{} = W ∪ V. Then
  N = M\textquotesingle/(Z/2) is a closed manifold with π₁(N) = Γ and rationally
  acyclic universal cover.

  See \texttt{problem7r-s-rot-obstruction-analysis.md} for full computation.
\item
  \textbf{S-rot-I (Wall surgery in odd dimension).} Fallback. Same three-obstacle
  structure as Approach I but with favorable odd L-theory parity and
  strictly fewer AHSS terms than the reflection case.
\end{enumerate}

\textbf{Deprioritized paths:}

\begin{enumerate}
\def\labelenumi{\arabic{enumi}.}
\setcounter{enumi}{2}
\item
  \textbf{Approach I (Wall surgery, reflection route).} Three successive open
  obstacles with structural headwinds (even L-theory parity).
\item
  \textbf{Approach III (orbifold resolution).} No technique identified.
\item
  \textbf{Approach II (equivariant surgery, reflection route).} Blocked by
  codimension-2 gap hypothesis (Costenoble-Waner, arXiv:1705.10909).
\end{enumerate}

\subsection{References}

\begin{itemize}
\tightlist
\item
  J. Fowler, \emph{Finiteness Properties of Rational Poincare Duality Groups},
  arXiv:1204.4667.
\item
  G. Avramidi, \emph{Rational Manifold Models for Duality Groups},
  arXiv:1506.06293.
\item
  A. Douba, F. Vargas Pallete, \emph{On Reflections of Congruence Hyperbolic
  Manifolds}, arXiv:2506.23994.
\item
  A. Bartels, F. T. Farrell, W. Luck, \emph{The Farrell-Jones Conjecture for
  Cocompact Lattices in Virtually Connected Lie Groups}, arXiv:1101.0469.
\item
  A. Bartels, W. Luck, \emph{The Farrell-Jones Conjecture for Arbitrary
  Lattices in Virtually Connected Lie Groups}, arXiv:1401.0876.
\item
  D. Crowley, W. Luck, T. Macko, \emph{Surgery Theory: Foundations},
  arXiv:0905.0104.
\item
  K. H. Dovermann, R. Schultz, \emph{Equivariant Surgery Theories and Their
  Periodicity Properties}, Springer LNM 1443, 1990.
\item
  S. R. Costenoble, S. Waner, \emph{The Equivariant Spivak Normal Bundle and
  Equivariant Surgery for Compact Lie Groups}, arXiv:1705.10909.
\item
  J. F. Davis, W. Lück, \emph{On Nielsen Realization and Manifold Models for
  Classifying Spaces}, Trans. AMS 377 (2024), 7557-7600, arXiv:2303.15765.
\item
  G. Avramidi, \emph{Smith Theory, L2 Cohomology, Isometries of Locally Symmetric
  Manifolds, and Moduli Spaces of Curves}, arXiv:1106.1704.
\item
  M. Belolipetsky, A. Lubotzky, \emph{Finite Groups and Hyperbolic Manifolds},
  arXiv:math/0406607.
\item
  A. Borel, Harish-Chandra, \emph{Arithmetic Subgroups of Algebraic Groups},
  Annals of Mathematics 75 (1962), 485-535.
\item
  J. Millson, M. S. Raghunathan, \emph{Geometric Construction of Cohomology for
  Arithmetic Groups I}, Proc. Indian Acad. Sci. 90 (1981), 103-123.
\item
  A. Ranicki, \emph{Algebraic L-Theory and Topological Manifolds}, Cambridge
  Tracts in Mathematics 102, 1992.
\item
  K. S. Brown, \emph{Cohomology of Groups}, Springer GTM 87.
\item
  W. Luck, \emph{Transformation Groups and Algebraic K-Theory}, Springer LNM
  1408.
\item
  C. T. C. Wall, \emph{Surgery on Compact Manifolds}, 2nd ed., AMS.
\end{itemize}
