\section{Problem 8: Lagrangian Smoothing of Polyhedral Surfaces with 4-Valent Vertices}

\subsection{Problem Statement}

A \textbf{polyhedral Lagrangian surface} K in R\^{}4 is a finite polyhedral complex all
of whose faces are Lagrangian 2-planes, which is a topological submanifold of
R\^{}4. A \textbf{Lagrangian smoothing} of K is a Hamiltonian isotopy K\_t of smooth
Lagrangian submanifolds for t in (0,1{]}, extending to a topological isotopy on
{[}0,1{]}, with K\_0 = K.

\textbf{Question:} If K has exactly 4 faces meeting at every vertex, does K
necessarily have a Lagrangian smoothing?

\subsection{Answer}

\textbf{Yes.} A polyhedral Lagrangian surface with 4 faces per vertex (with
distinct adjacent faces) always admits a Lagrangian smoothing. The vertex
spanning property (\{e\_1,...,e\_4\} spans R\^{}4) is automatic --- it follows from
the Lagrangian and distinct-face conditions alone (Section 3), with no
additional hypotheses needed.

\textbf{Confidence: Medium-high.} The v2 argument (symplectic direct sum
decomposition at each vertex) is verified numerically: 998/998 random valid
4-valent configurations give Maslov index exactly 0. The decomposition proof
is algebraic (not just heuristic). Vertex smoothing uses the product structure
(corners in symplectic factors); edge smoothing uses generating functions.
The Hamiltonian isotopy property is established via the Weinstein neighborhood
theorem (edges) and the vanishing flux in simply-connected R\^{}2 (vertices).

\subsection{Solution}

\subsubsection{1. Setup: Lagrangian planes in R\^{}4}

Identify R\^{}4 = C\^{}2 with symplectic form omega = dx\_1 ∧ dy\_1 + dx\_2 ∧ dy\_2.
A Lagrangian plane is a 2-dimensional subspace L where omega\textbar\_L = 0. The space
of Lagrangian planes (the Lagrangian Grassmannian) is:

\begin{verbatim}
Lambda(2) = U(2)/O(2)
\end{verbatim}

This is a manifold of dimension 3 with pi\_1(Lambda(2)) = Z (the Maslov class).

Each face of K lies in a Lagrangian plane. At each edge, two Lagrangian faces
meet at a dihedral angle. At each vertex, 4 Lagrangian faces meet.

\subsubsection{2. Local structure at a 4-valent vertex}

At a vertex v, the 4 faces L\_1, L\_2, L\_3, L\_4 (in cyclic order around v) are
Lagrangian half-planes meeting along edges:

\begin{verbatim}
e_{12} = L_1 ∩ L_2,  e_{23} = L_2 ∩ L_3,  e_{34} = L_3 ∩ L_4,  e_{41} = L_4 ∩ L_1
\end{verbatim}

These 4 edges are rays from v. Each face is spanned by its two boundary edges:

\begin{verbatim}
L_i = span(e_{i-1,i}, e_{i,i+1})
\end{verbatim}

The Lagrangian condition omega\textbar\_\{L\_i\} = 0 requires:

\begin{verbatim}
omega(e_{i-1,i}, e_{i,i+1}) = 0  for each i (mod 4)
\end{verbatim}

\subsubsection{3. The symplectic direct sum decomposition (key new argument)}

\textbf{Theorem (v2).} The 4 edge vectors e\_1 = e\_\{12\}, e\_2 = e\_\{23\}, e\_3 = e\_\{34\},
e\_4 = e\_\{41\} satisfy omega(e\_i, e\_\{i+1\}) = 0 for all i (mod 4). This forces
a symplectic direct sum decomposition:

\begin{verbatim}
R^4 = V_1 ⊕ V_2
\end{verbatim}

where V\_1 = span(e\_1, e\_3) and V\_2 = span(e\_2, e\_4) are symplectic 2-planes
with omega\textbar\_\{V\_1 x V\_2\} = 0.

\textbf{Proof.} Write the omega matrix in the basis (e\_1, e\_2, e\_3, e\_4). The
conditions omega(e\_i, e\_\{i+1\}) = 0 kill 4 of the 6 independent entries,
leaving only:

\begin{verbatim}
a = omega(e_1, e_3) ≠ 0,    b = omega(e_2, e_4) ≠ 0
\end{verbatim}

(Nonzero because omega is non-degenerate and \{e\_1, e\_2, e\_3, e\_4\} is a
basis --- see the nondegeneracy hypothesis below.)

\textbf{Lemma (vertex spanning).} At every vertex of a polyhedral Lagrangian
surface with 4 faces per vertex (with distinct adjacent faces sharing
1-dimensional edges), the 4 edge vectors \{e\_1, e\_2, e\_3, e\_4\} span R\^{}4.

\emph{Proof.} Suppose for contradiction that the 4 edge vectors lie in a
3-dimensional subspace H ⊂ R\^{}4. Then all 4 faces L\_i = span(e\_\{i-1,i\},
e\_\{i,i+1\}) ⊂ H.

\textbf{Step 1.} The restricted form omega\textbar\_H has rank 2: since omega is
non-degenerate on R\^{}4, its restriction to a codimension-1 subspace has a
1-dimensional kernel ell = ker(omega\textbar\_H) ⊂ H.

\textbf{Step 2.} Every Lagrangian 2-plane L ⊂ H contains ell. Proof: L is
2-dimensional with omega\textbar\_L = 0. If L were transverse to ell (i.e.,
L ∩ ell = \{0\}), then L would project isomorphically onto a 2-plane in
H/ell ≅ R\^{}2. But omega\textbar\_H descends to a non-degenerate 2-form on H/ell
(since ell = ker(omega\textbar\_H)), and a 2-plane in R\^{}2 on which a
non-degenerate 2-form vanishes must be \{0\}. Contradiction. So ell ⊂ L.

\textbf{Step 3 (key).} Each edge e\_i is the intersection of two adjacent faces:
e\_1 = L\_1 ∩ L\_2, e\_2 = L\_2 ∩ L\_3, etc. Since L\_i and L\_\{i+1\} are
distinct 2-planes (distinct faces), their intersection is exactly
1-dimensional: dim(L\_i ∩ L\_\{i+1\}) = 1, so L\_i ∩ L\_\{i+1\} = span(e\_i).

By Step 2, ell ⊂ L\_i and ell ⊂ L\_\{i+1\}, so ell ⊂ L\_i ∩ L\_\{i+1\} =
span(e\_i). Since both ell and span(e\_i) are 1-dimensional subspaces and
one contains the other, ell = span(e\_i).

Applying this to all four edges: span(e\_1) = ell = span(e\_2) = span(e\_3) =
span(e\_4). That is, all four edge vectors are proportional.

But then L\_1 = span(e\_4, e\_1) = span(e\_1) is 1-dimensional, contradicting
the fact that L\_1 is a 2-plane. ∎

(Note: this argument is purely algebraic --- it uses only that adjacent faces
are distinct Lagrangian 2-planes sharing a 1-dimensional edge. No
topological submanifold condition is needed.)

In the reordered basis (e\_1, e\_3, e\_2, e\_4), the omega matrix is:

\begin{verbatim}
[[0, a, 0, 0], [-a, 0, 0, 0], [0, 0, 0, b], [0, 0, -b, 0]]
\end{verbatim}

This is block diagonal: (V\_1, a) ⊕ (V\_2, b), a symplectic direct sum. ∎

\textbf{Consequence:} Each Lagrangian face decomposes as a direct sum of lines:

\begin{verbatim}
L_1 = span(e_4) ⊕ span(e_1)  ⊂  V_2 ⊕ V_1
L_2 = span(e_1) ⊕ span(e_2)  ⊂  V_1 ⊕ V_2
L_3 = span(e_2) ⊕ span(e_3)  ⊂  V_2 ⊕ V_1
L_4 = span(e_3) ⊕ span(e_4)  ⊂  V_1 ⊕ V_2
\end{verbatim}

Each L\_i takes one line from V\_1 and one from V\_2, which is automatically
Lagrangian in V\_1 ⊕ V\_2 (omega restricted to a line is zero).

\subsubsection{4. Maslov index vanishes exactly}

The Maslov index of the vertex loop L\_1 → L\_2 → L\_3 → L\_4 → L\_1 in
Lambda(2) decomposes via the direct sum:

\begin{verbatim}
mu = mu_1 + mu_2
\end{verbatim}

where mu\_j is the winding number of the component loop in V\_j.

\textbf{In V\_1 = span(e\_1, e\_3):} The loop traces
span(e\_1) → span(e\_1) → span(e\_3) → span(e\_3) → span(e\_1)

This is a back-and-forth path (e\_1 → e\_1 → e\_3 → e\_3 → e\_1), not a
winding loop. The winding number is 0.

\textbf{In V\_2 = span(e\_2, e\_4):} Similarly:
span(e\_4) → span(e\_2) → span(e\_2) → span(e\_4) → span(e\_4)

Winding number 0.

\textbf{Total Maslov index: mu = 0 + 0 = 0.}

\textbf{Numerical verification:} 998/998 random valid 4-valent configurations
(with edge-sharing enforced) give Maslov index exactly 0. In contrast,
random quadruples without edge-sharing give nonzero Maslov index \textasciitilde45\% of
the time (see scripts/verify-p8-maslov-v2.py).

\subsubsection{5. Why 4 is special: the 3-face obstruction}

For a 3-face vertex, the 3 edge vectors e\_1, e\_2, e\_3 must satisfy:

\begin{verbatim}
omega(e_1, e_2) = omega(e_2, e_3) = omega(e_3, e_1) = 0
\end{verbatim}

This means omega vanishes on ALL pairs --- the span is an isotropic subspace.
But in (R\^{}4, omega), the maximum isotropic dimension is 2 (= half the
dimension). Three independent isotropic vectors cannot exist.

\textbf{Therefore: a non-degenerate 3-face Lagrangian vertex is impossible in R\^{}4.}

The 4-face condition is precisely the right valence: it gives enough edges to
span R\^{}4 while the cyclic omega-orthogonality creates the symplectic direct
sum that forces Maslov index 0.

For 5 or more faces: the omega-orthogonality conditions are over-determined
and generically have no solution with the edges spanning R\^{}4. The 4-face case
is the generic sweet spot.

\subsubsection{5a. Vertex smoothing via the product structure}

The key observation: the V\_1 ⊕ V\_2 decomposition (Section 3) gives K near
each vertex v a product structure, enabling a direct smoothing that bypasses
both crease smoothing and Lagrangian surgery.

\textbf{Product decomposition of K near v.} Choose Darboux coordinates adapted to
V\_1 ⊕ V\_2: let V\_1 have coordinates (x, y) with omega\_1 = dx ∧ dy, and V\_2
have coordinates (u, v) with omega\_2 = du ∧ dv (so omega = omega\_1 + omega\_2).
Orient the edge vectors so that:

\begin{verbatim}
e_1 = (1,0,0,0), e_2 = (0,0,1,0), e_3 = (0,1,0,0), e_4 = (0,0,0,1)
\end{verbatim}

(Here e\_1, e\_3 ∈ V\_1 and e\_2, e\_4 ∈ V\_2, consistent with the decomposition
in Section 3. This can always be arranged by a linear symplectomorphism.)

The 4 faces in these coordinates are:

\begin{verbatim}
L_1 = span(e_4, e_1) = {y = 0, u = 0}, sector x > 0, v > 0
L_2 = span(e_1, e_2) = {y = 0, v = 0}, sector x > 0, u > 0
L_3 = span(e_2, e_3) = {x = 0, v = 0}, sector y > 0, u > 0
L_4 = span(e_3, e_4) = {x = 0, u = 0}, sector y > 0, v > 0
\end{verbatim}

Define the "corner" curves in each factor:

\begin{verbatim}
C_1 = {(x, 0) : x ≥ 0} ∪ {(0, y) : y ≥ 0}  ⊂ V_1  (positive x and y axes)
C_2 = {(u, 0) : u ≥ 0} ∪ {(0, v) : v ≥ 0}  ⊂ V_2  (positive u and v axes)
\end{verbatim}

\textbf{Claim:} K ∩ B(v, r) = C\_1 × C\_2 (product of the two corners) for small r.

\emph{Verification:} A point (p, q) with p ∈ C\_1 and q ∈ C\_2 has p = (x, 0) or
(0, y) and q = (u, 0) or (0, v), giving 4 cases --- exactly the 4 faces
L\_1, ..., L\_4 listed above. ∎

\textbf{Smoothing.} Replace each corner C\_j with a smooth curve C\_j\^{}\{sm\} ⊂ V\_j
that agrees with C\_j outside a ball of radius delta around the origin:

\begin{verbatim}
C_1^{sm}: smooth curve in V_1, = {(x, 0)} for x > delta, = {(0, y)} for y > delta,
          smooth through the origin (e.g., the curve (cos theta, sin theta)
          reparameterized to match the axes outside the transition)
C_2^{sm}: analogous in V_2
\end{verbatim}

Explicitly: parameterize C\_1\^{}\{sm\} as gamma\_1(t) for t ∈ R, where
gamma\_1(t) = (t, 0) for t ≥ delta, gamma\_1(t) = (0, -t) for t ≤ -delta,
and gamma\_1 is a smooth embedded curve for t ∈ {[}-delta, delta{]}. (Such a
curve exists: it is a smooth rounding of the right angle at the origin.)
Define gamma\_2(t) analogously for C\_2\^{}\{sm\}.

\textbf{Lagrangian property of the product:} The smoothed surface
K\^{}\{sm\} = C\_1\^{}\{sm\} × C\_2\^{}\{sm\} is a product of smooth curves in the
symplectic factors. A curve in a 2-dimensional symplectic manifold is
always Lagrangian (its dimension is 1 = half of 2, and omega restricted
to a 1-submanifold is zero for dimensional reasons). The product of
Lagrangian submanifolds in (V\_1, omega\_1) × (V\_2, omega\_2) is Lagrangian
in (V\_1 ⊕ V\_2, omega\_1 + omega\_2): for tangent vectors (u\_1, u\_2) and
(w\_1, w\_2) to C\_1\^{}\{sm\} × C\_2\^{}\{sm\}, we have

\begin{verbatim}
omega((u_1, u_2), (w_1, w_2)) = omega_1(u_1, w_1) + omega_2(u_2, w_2) = 0 + 0 = 0
\end{verbatim}

since u\_1, w\_1 are tangent to C\_1\^{}\{sm\} (1-dimensional in V\_1) and u\_2, w\_2
are tangent to C\_2\^{}\{sm\} (1-dimensional in V\_2). ∎

\textbf{Smoothness:} K\^{}\{sm\} = C\_1\^{}\{sm\} × C\_2\^{}\{sm\} is the image of the smooth
map (t\_1, t\_2) → (gamma\_1(t\_1), gamma\_2(t\_2)), which is a smooth immersion
(the tangent vectors d gamma\_1/dt\_1 and d gamma\_2/dt\_2 are nonzero and lie
in complementary subspaces V\_1 and V\_2). So K\^{}\{sm\} is a smooth Lagrangian
surface, including at v = gamma\_1(0) × gamma\_2(0). ∎

\textbf{Agreement:} K\^{}\{sm\} = C\_1 × C\_2 = K outside the region where either
factor was modified (\textbar p\textbar{} ≤ delta in V\_1 or \textbar q\textbar{} ≤ delta in V\_2), so K\^{}\{sm\}
agrees with K outside a neighborhood of v of radius O(delta).

\subsubsection{6. Edge smoothing (crease smoothing along edges between vertices)}

After resolving all vertices (Section 5a), the remaining singularities of K
are edge creases: compact arcs connecting the boundaries of resolved vertex
neighborhoods. Along each edge arc, two adjacent Lagrangian faces meet at a
dihedral angle. These creases are smoothed by the standard generating-function
interpolation:

\textbf{Lemma (edge crease smoothing).} Let L\_1, L\_2 be two Lagrangian half-planes
meeting along a compact edge arc e (a segment of a common boundary ray, away
from any vertex). Then there exists a smooth Lagrangian surface agreeing with
L\_1 on one side and L\_2 on the other.

\emph{Proof.} Choose Darboux coordinates (x\_1, y\_1, x\_2, y\_2) with e along the
x\_1-axis. Each L\_i is locally the graph y = grad S\_i(x) for quadratic S\_i.
Define S\_eps(x) = chi(x\_1/eps) S\_1(x) + (1 - chi(x\_1/eps)) S\_2(x) for a
smooth cutoff chi. The graph y = grad S\_eps is Lagrangian (graph of the
exact 1-form dS\_eps). For any fixed eps \textgreater{} 0, this gives a smooth Lagrangian
surface replacing the crease along e. ∎

(The C\^{}1 control issues noted in earlier drafts do not arise here: the
crease smoothing is applied only along edge arcs that are INTERIOR to edges
(between vertex neighborhoods), not at vertices. Since the edge arcs are at
positive distance from all vertices, and the smoothing width eps can be
chosen small relative to this distance, the edge smoothings are localized
in thin tubular neighborhoods of the edge arcs.)

\subsubsection{7. Global smoothing}

\textbf{Step 7a: Resolve vertices.} For each 4-valent vertex v\_i, choose a ball
B\_i of radius r\_i centered at v\_i, where r\_i is small enough that:

\begin{itemize}
\tightlist
\item
  B\_i contains no other vertex v\_j (j != i), and
\item
  B\_i intersects only the edges and faces incident to v\_i.
\end{itemize}

Such radii exist because the vertex set is finite and discrete in R\^{}4.
Within each B\_i, apply the product smoothing (Section 5a) using the
V\_1 ⊕ V\_2 decomposition from Section 3. This replaces K ∩ B\_i = C\_1 × C\_2
with the smooth Lagrangian surface C\_1\^{}\{sm\} × C\_2\^{}\{sm\}.

\textbf{Commutativity:} Since the balls \{B\_i\} are pairwise disjoint, the
vertex smoothings have disjoint support and commute.

\textbf{Step 7b: Smooth edges.} After all vertex resolutions, the remaining
singularities are edge creases --- compact arcs connecting the boundaries
of resolved vertex neighborhoods. Each edge arc lies along the intersection
of two Lagrangian faces and is a compact 1-manifold (with boundary on
the spheres ∂B\_i). These creases are resolved by the generating-function
interpolation of Section 6.

The edge smoothings have support in tubular neighborhoods of the edge arcs.
These neighborhoods can be chosen to be disjoint from each other (since the
edges are disjoint away from vertices, which have already been resolved) and
from the vertex balls (since the edge arcs start at ∂B\_i, outside B\_i).
Therefore the edge smoothings commute with each other and with the vertex
smoothings.

\textbf{Compatibility at ∂B\_i.} The vertex smoothing (Section 5a) agrees with
the original polyhedral K outside a neighborhood of v\_i of radius delta \textless{} r\_i.
So on ∂B\_i, the surface is still polyhedral (two flat faces meeting at an
edge). The edge crease smoothing (Section 6) begins at ∂B\_i, where the
surface is already flat. Since both the vertex smoothing (product of smooth
curves) and the edge smoothing (graph of exact 1-form) produce Lagrangian
surfaces, and they agree on the overlap (both equal the original flat faces
on the annular region delta \textless{} \textbar x\textbar{} \textless{} r\_i), they glue to a globally smooth
Lagrangian surface.

\textbf{Step 7c: Global Hamiltonian isotopy.} Each smoothing is parameterized by
a width parameter t (delta for vertices, eps for edges). As t → 0, the
smoothing region shrinks and K\_t → K = K\_0 in C\^{}0, giving the topological
isotopy on {[}0, 1{]}. It remains to show K\_t is a \emph{Hamiltonian} isotopy
for t \textgreater{} 0.

\textbf{Edge smoothings are Hamiltonian.} The family of generating functions
S\_t(x) = chi(x\_1/t) S\_1(x) + (1 - chi(x\_1/t)) S\_2(x) defines a smooth
1-parameter family of Lagrangian graphs y = grad S\_t(x) in T\^{}*R\^{}2. By
the Weinstein Lagrangian neighborhood theorem (Weinstein 1971, Theorem 6.1),
a smooth family of exact Lagrangian submanifolds in an exact symplectic
manifold (here T\^{}*R\^{}2 with lambda = y dx) is generated by a Hamiltonian:
the 1-form i\_\{d/dt\}(omega)\textbar{}\emph{\{K\_t\} is exact (it equals d(S\_t\textbar{}}\{K\_t\})),
so the isotopy is Hamiltonian with generating function H\_t = dS\_t/dt
evaluated on the Lagrangian.

\textbf{Vertex smoothings are Hamiltonian.} The vertex smoothing is a product:
K\_t\^{}\{vertex\} = C\_1\^{}\{sm\}(t) × C\_2\^{}\{sm\}(t) in V\_1 × V\_2. Each C\_j\^{}\{sm\}(t)
is a smooth 1-parameter family of curves in (V\_j, omega\_j) ≅ (R\^{}2, dx∧dy).
In a 2-dimensional exact symplectic manifold, EVERY isotopy of compact
(or compactly supported) Lagrangian submanifolds is Hamiltonian. This is
because V\_j = R\^{}2 is simply connected, so the flux homomorphism
Flux: pi\_1(Symp) → H\^{}1(L; R) is trivial (no non-Hamiltonian symplectic
isotopies exist). Concretely: the velocity field d/dt of the isotopy
C\_j\^{}\{sm\}(t) is a symplectic vector field along C\_j\^{}\{sm\}(t); contracting
with omega\_j gives a closed 1-form on C\_j\^{}\{sm\}(t), which is exact because
H\^{}1(C\_j\^{}\{sm\}(t); R) = 0 (each C\_j\^{}\{sm\}(t) is a noncompact embedded arc
in R\^{}2 --- the smoothed corner --- hence contractible). The primitive
is the Hamiltonian H\_j(t). The product isotopy on V\_1 × V\_2 is Hamiltonian
with generating function H\_1(t) + H\_2(t) (see McDuff-Salamon, 3rd ed.,
Exercise 3.18: product of Hamiltonian isotopies is Hamiltonian).

\textbf{Composition.} The vertex smoothings (in disjoint balls B\_i) and edge
smoothings (in disjoint tubular neighborhoods) have pairwise disjoint
compact support. Each is a compactly supported Hamiltonian isotopy. Their
composition is a compactly supported Hamiltonian isotopy (McDuff-Salamon,
\emph{Introduction to Symplectic Topology}, 3rd ed., Proposition 3.17: the
group Ham\_c(M, omega) of compactly supported Hamiltonian diffeomorphisms
is a group under composition).

\subsubsection{8. Topological constraints}

For K to be a compact topological submanifold with 4 faces per vertex,
Euler\textquotesingle s formula constrains the topology. For quadrilateral faces:
V - E + F = chi(K), with 4F = 2E and 4V = 2E, giving chi = 0.
So K is topologically a torus or Klein bottle.

For non-quadrilateral faces with 4 per vertex, other topologies are possible
but chi \textless= 0 (higher genus surfaces).

\subsubsection{9. Summary}

The smoothing exists because:

\begin{enumerate}
\def\labelenumi{\arabic{enumi}.}
\tightlist
\item
  The 4-face + Lagrangian condition forces a symplectic direct sum R\^{}4 = V\_1 ⊕ V\_2
  at each vertex (vertex spanning is proved algebraically; opposite edges span
  complementary symplectic 2-planes)
\item
  This decomposition forces Maslov index exactly 0 (algebraic, not just generic)
\item
  Near each vertex, K = C\_1 × C\_2 (product of corners). Smoothing each corner
  gives K\^{}\{sm\} = C\_1\^{}\{sm\} × C\_2\^{}\{sm\}, a smooth Lagrangian surface (product of
  curves in symplectic factors). No Lagrangian surgery needed at vertices.
\item
  Edge creases (between vertices) are smoothed by generating function interpolation
\item
  All smoothings are Hamiltonian isotopies, composable to give global K\_t
\item
  The 3-face case is impossible (isotropic dimension bound), explaining
  why 4 is the right valence
\end{enumerate}

\subsection{Corrections from v1}

\begin{itemize}
\tightlist
\item
  \textbf{v1 claimed} Maslov index vanishes by "alternating orientations relative
  to J." \textbf{Replaced} with the precise symplectic direct sum argument: the
  edge-sharing constraint forces omega to be block diagonal in the
  (e\_1, e\_3, e\_2, e\_4) basis, giving mu = 0 algebraically.
\item
  \textbf{v1 stated} the 3-face obstruction as "nonzero Maslov index."
  \textbf{Corrected:} 3-face Lagrangian vertices don\textquotesingle t have nonzero Maslov index ---
  they\textquotesingle re impossible (3 isotropic vectors can\textquotesingle t span 3D in R\^{}4).
\item
  \textbf{v1 lacked} numerical verification. \textbf{Added} via verify-p8-maslov-v2.py:
  998/998 valid configurations give mu = 0; comparison with non-edge-sharing
  quadruples (55\% have mu = 0) confirms the constraint is essential.
\end{itemize}

\subsection{References}

\begin{itemize}
\tightlist
\item
  D. McDuff, D. Salamon, \emph{Introduction to Symplectic Topology}, 3rd ed.,
  Oxford University Press, 2017, Proposition 3.17. {[}Composition of compactly
  supported Hamiltonian isotopies{]}
\end{itemize}

\subsection{Key References from futon6 corpus}

\begin{itemize}
\tightlist
\item
  PlanetMath: "symplectic manifold" (SymplecticManifold) --- symplectic structures
\item
  PlanetMath: "Darboux\textquotesingle s theorem" (DarbouxsTheoremSymplecticGeometry) --- local coordinates
\item
  PlanetMath: "concepts in symplectic geometry" (ConceptsInSymplecticGeometry) --- overview
\item
  PlanetMath: "symplectic complement" / "coisotropic subspace" --- Lagrangian subspaces
\item
  PlanetMath: "Kähler manifold is symplectic" (AKahlerManifoldIsSymplectic) --- complex structure
\end{itemize}
