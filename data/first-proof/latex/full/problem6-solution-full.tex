\hypertarget{problem-6-epsilon-light-subsets-of-graphs}{%
\section{Problem 6: Epsilon-Light Subsets of Graphs}\label{problem-6-epsilon-light-subsets-of-graphs}}

\hypertarget{problem-statement}{%
\subsection{Problem Statement}\label{problem-statement}}

Let G=(V,E,w) be a finite undirected graph with nonnegative edge weights and
n=\textbar V\textbar. Its Laplacian is

\[L = \sum_{e={u,v} \in E} w_{e} (e_{u}-e_{v})(e_{u}-e_{v})^{T}.\]

For S subseteq V, define the induced-subgraph Laplacian embedded to \(R^{VxV}\):

\[L_{S} = \sum_{e={u,v} \in E, u,v \in S} w_{e} (e_{u}-e_{v})(e_{u}-e_{v})^{T},\]

with zeros outside S rows/columns.

S is \(\epsilon\)-light if

\[L_{S} \le \epsilon L\]

in Loewner order.

Question: does there exist universal \(c > \mNumber{0}\) such that for every G and every
\(\epsilon\) in (0,1), there exists S with \textbar S\textbar{} \textgreater= c\emph{\(\epsilon\)}n and \(L_{S} \le \epsilon\) L?

\hypertarget{status-of-this-writeup}{%
\subsection{Status of this writeup}\label{status-of-this-writeup}}

This draft closes the internal algebraic gaps and separates proved facts from
external dependencies:

\begin{enumerate}
\def\labelenumi{\arabic{enumi}.}
\tightlist
\item
  Proved in-text: exact formulation, \(K_{n}\) upper bound c\textless=1, sampling identities,
  and a correct matrix-concentration setup.
\item
  External dependency: the universal lower bound \(c\mNumber{0} > \mNumber{0}\) for all graphs is not
  rederived here; it is treated as an imported theorem assumption.
\end{enumerate}

So the final existential answer is \textbf{conditional/open in this writeup}:
without Assumption V, the universal-c claim is unresolved here.

\hypertarget{1-exact-reformulation}{%
\subsection{1. Exact reformulation}\label{1-exact-reformulation}}

The PSD condition is equivalent to the quadratic form inequality

\begin{verbatim}
for all x in R^V: x^T L_S x <= epsilon x^T L x.
\end{verbatim}

On im(L), with \(L^+\) the Moore-Penrose pseudoinverse:

\[L_{S} \le \epsilon L \Leftrightarrow || L^{+/\mNumber{2}} L_{S} L^{+/\mNumber{2}} || \le \epsilon.\]

\hypertarget{2-complete-graph-upper-bound-rigorous}{%
\subsection{2. Complete graph upper bound (rigorous)}\label{2-complete-graph-upper-bound-rigorous}}

For \(G=K_{n}\) and S of size s, choose x supported on S with \(\\sum_{i \in S} x_{i} = \mNumber{0}\).
Then

\[\begin{aligned}
x^{T} L_{K_{n}} x = n ||x||^\mNumber{2}, \\
x^{T} L_{S} x = s ||x||^\mNumber{2}.
\end{aligned}\]

Hence \(L_{S} \le \epsilon L_{K_{n}}\) implies \(s \le \epsilon\) n.
Therefore any universal constant must satisfy

\[c \le \mNumber{1}.\]

This is an upper bound only.

\hypertarget{3-random-sampling-identities-expectation-level}{%
\subsection{3. Random sampling identities (expectation level)}\label{3-random-sampling-identities-expectation-level}}

Let \(Z_{v}\) \textasciitilde{} Bernoulli(p) independently and S=\{v: \(Z_{v}\)=1\}.

\begin{enumerate}
\def\labelenumi{\arabic{enumi}.}
\item
  Size:

  E\textbar S\textbar{} = pn,
  Pr{[}\textbar S\textbar{} \textless{} pn/2{]} \textless= exp(-pn/8) (Chernoff).
\item
  Spectral expectation:

  \(E[L_{S}] = p^\mNumber{2}\) L,
\end{enumerate}

since each edge survives with probability \(p^\mNumber{2}\).

Thus

\[E[\epsilon L - L_{S}] = (\epsilon - p^\mNumber{2})L.\]

Setting p=epsilon gives \(E[L_{S}]=\epsilon^{\mNumber{2}} L \le \epsilon\) L. This is not yet a
realization-level guarantee.

\hypertarget{4-concentration-setup-gap-fixed-formulation}{%
\subsection{4. Concentration setup (gap-fixed formulation)}\label{4-concentration-setup-gap-fixed-formulation}}

Define edge-normalized PSD matrices

\[X_{e} = L^{+/\mNumber{2}} w_{e} b_{e} b_{e}^{T} L^{+/\mNumber{2}}, b_{e} = e_{u}-e_{v},\]

and leverage scores

\[\begin{aligned}
\tau_{e} = \mOpName{tr}(X_{e}) = w_{e} b_{e}^{T} L^+ b_{e}, \\
\sum_{e} \tau_{e} = n - k
\end{aligned}\]

(k = number of connected components).

\hypertarget{4a-star-domination-with-correct-counting}{%
\subsubsection{4a. Star domination with correct counting}\label{4a-star-domination-with-correct-counting}}

Using \(Z_{u} Z_{v} \le Z_{u}\) and \(Z_{u} Z_{v} \le Z_{v}\) for each edge \(\{u,v\}\),

\[\begin{aligned}
L_{S} = \sum_{uv \in E} Z_{u} Z_{v} L_{uv} \\
\le (\mNumber{1}/\mNumber{2}) \sum_{v} Z_{v} \sum_{u~v} L_{uv}.
\end{aligned}\]

So in normalized coordinates

\[\begin{aligned}
L^{+/\mNumber{2}} L_{S} L^{+/\mNumber{2}} \le \sum_{v} Z_{v} A_{v}, \\
A_{v} := (\mNumber{1}/\mNumber{2}) \sum_{u~v} X_{uv} \ge \mNumber{0}.
\end{aligned}\]

Because \(Z_{v}\) are independent Bernoulli variables, the random matrices \(Z_{v} A_{v}\)
are independent PSD summands.

\hypertarget{4b-freedmanbernstein-martingale-parameters}{%
\subsubsection{4b. Freedman/Bernstein martingale parameters}\label{4b-freedmanbernstein-martingale-parameters}}

Let \(A_{i}\) be a fixed ordering of \(\{A_{v}\}, p_{i}=E[Z_{i}],\) and define the centered sum

\[X = \sum_{i} (Z_{i} - p_{i}) A_{i}.\]

With filtration \(F_{i} = \sigma(Z_\mNumber{1},...,Z_{i})\), Doob martingale

\[\begin{aligned}
Y_{i} = E[X | F_{i}], \\
\Delta_{i} = Y_{i} - Y_{i-\mNumber{1}}
\end{aligned}\]

has self-adjoint differences. For independent Bernoulli sampling,

\[\begin{aligned}
\Delta_{i} = (Z_{i} - p_{i}) A_{i}, \\
||\Delta_{i}|| \le ||A_{i}|| \le R_{\mDualStar} (R_{\mDualStar} = max_{i} ||A_{i}||),
\end{aligned}\]

and predictable quadratic variation

\[W_{n} = \sum_{i} E[\Delta_{i}^\mNumber{2} | F_{i-\mNumber{1}}] = \sum_{i} p_{i}(\mNumber{1}-p_{i}) A_{i}^\mNumber{2}.\]

Matrix Freedman (or matrix Bernstein in independent form) applies once bounds
on \(R_{\mDualStar}\) and \(||W_{n}||\) are supplied.

\textbf{What graph-dependent bounds are needed.} To obtain a self-contained
concentration bound, one would need \(R_{\mDualStar} \le C_\mNumber{1} \ast \epsilon\) and
\(||W_{n}|| \le C_\mNumber{2} \ast \epsilon^\mNumber{2}\) for graph-dependent constants \(C_{\mNumber{1}}, C_{\mNumber{2}}\). Bounding
these requires leverage score analysis (showing \(\tau_{e}\) bounds are well-distributed
across vertices) that is the core content of the external theorem referenced
in Section 5. Specifically, the Batson-Spielman-Srivastava barrier-function
method controls both \(R_{\mDualStar}\) and \(||W_{n}||\) simultaneously through a potential
function that tracks the spectral approximation quality.

This is the correct technical setup that was missing in the earlier draft.

\hypertarget{5-external-dependency-for-universal-c0--0}{%
\subsection{\texorpdfstring{5. External dependency for universal \(c\mNumber{0} > \mNumber{0}\)}{5. External dependency for universal c\textbackslash mNumber\{0\} \textgreater{} \textbackslash mNumber\{0\}}}\label{5-external-dependency-for-universal-c0--0}}

To conclude a universal lower bound for all graphs, one needs an additional
theorem (not proved in this writeup) that controls leverage/pruning and proves:

\begin{verbatim}
exists $c0 > 0$ universal such that
for all G, epsilon in (0,1), exists S with |S|>=c0*epsilon*n and L_S<=epsilon L.
\end{verbatim}

We state this explicitly as an assumption:

\[\begin{aligned}
\text{Assumption} V (vertex-light selection theorem): \\
\text{There} \text{exists} c\mNumber{0} > \mNumber{0} \text{such} \text{that} \text{for} \text{every} weighted graph G \text{and} \epsilon \in (\mNumber{0},\mNumber{1}), \\
\text{one} \text{can} choose S subseteq V \text{with} |S| \ge c\mNumber{0} \ast \epsilon \ast n \text{and} L_{S} \le \epsilon L.
\end{aligned}\]

Literature check in this cycle (MO/MSE local corpus plus the standard
sparsification references) did not locate a theorem matching Assumption V in
this exact vertex-induced, no-reweighting form.

Related evidence comes from the twice-Ramanujan sparsification theorem of
Batson-Spielman-Srivastava (2012, "Twice-Ramanujan Sparsifiers," SIAM Review
56(2), 315-334, Theorem 1.1), but that theorem is about edge selection
with reweighting, not induced subgraphs from vertex subsets.

\textbf{Important gap:} BSS is an \emph{edge sparsification} result --- it selects a
subset of edges with reweighting, not a subset of vertices. The problem asks
for a \emph{vertex subset} S with \(L_{S} \le \epsilon\) L. These are different objects:
edge sparsification preserves the vertex set and reweights edges, while the
\(\epsilon\)-light condition restricts to the induced subgraph on a vertex subset.

The star domination decomposition in Section 4a decomposes \(L_{S}\) into
vertex-indexed PSD summands, which is the right algebraic setup for a
vertex-selection proof strategy. But the adaptation from edge sparsification
to vertex-induced sparsification is exactly the missing step.

This writeup therefore treats the universal vertex-subset bound as conditional:
Sections 1-4 are proved in-text, and the existential claim is left open unless
Assumption V is supplied from outside.

\hypertarget{6-final-conclusion-explicitly-conditional}{%
\subsection{6. Final conclusion (explicitly conditional)}\label{6-final-conclusion-explicitly-conditional}}

Unconditional conclusions from this text:

\begin{enumerate}
\def\labelenumi{\arabic{enumi}.}
\tightlist
\item
  The statement is well-posed in Laplacian PSD order.
\item
  \(K_{n}\) implies universal upper bound c\textless=1.
\item
  The concentration machinery is set up correctly with explicit martingale
  increments and variance process.
\end{enumerate}

Conditional/open conclusion:

\begin{itemize}
\tightlist
\item
  Under Assumption V, one obtains a positive existential result (some universal
  \(c\mNumber{0} > \mNumber{0}\) exists).
\item
  Without Assumption V, this writeup does not prove existence of such a
  universal c0; the problem remains open here.
\end{itemize}

\hypertarget{key-identities-used}{%
\subsection{Key identities used}\label{key-identities-used}}

\begin{enumerate}
\def\labelenumi{\arabic{enumi}.}
\tightlist
\item
  \(L = \\sum_{e} w_{e} b_{e} b_{e}^{T}\)
\item
  \(L_{S} = \\sum_{e \text{internal} \text{to} S} w_{e} b_{e} b_{e}^{T}\)
\item
  \(\tau_{e} = w_{e} b_{e}^{T} L^+ b_{e}, \\sum_{e} \tau_{e} =\) n-k
\item
  \(L^{+/\mNumber{2}} L_{S} L^{+/\mNumber{2}} \le \\sum_{v} Z_{v} A_{v}\) with \(A_{v}=(\mNumber{1}/\mNumber{2})\sum_{u~v}X_{uv}\)
\end{enumerate}

\hypertarget{references}{%
\subsection{References}\label{references}}

\begin{itemize}
\tightlist
\item
  Batson, Spielman, Srivastava (2012), "Twice-Ramanujan Sparsifiers," SIAM
  Review 56(2), 315-334. {[}Theorem 1.1: deterministic edge sparsification;
  related but not equivalent to vertex-induced selection{]}
\item
  Tropp (2011), Freedman\textquotesingle s inequality for matrix martingales
\item
  Standard matrix Bernstein inequality for sums of independent self-adjoint
  random matrices
\end{itemize}
