\section{Problem 6: Epsilon-Light Subsets of Graphs}

\subsection{Problem Statement}

Let G=(V,E,w) be a finite undirected graph with nonnegative edge weights and
n=\textbar V\textbar. Its Laplacian is

\begin{verbatim}
L = sum_{e={u,v} in E} w_e (e_u-e_v)(e_u-e_v)^T.
\end{verbatim}

For S subseteq V, define the induced-subgraph Laplacian embedded to R\^{}\{VxV\}:

\begin{verbatim}
L_S = sum_{e={u,v} in E, u,v in S} w_e (e_u-e_v)(e_u-e_v)^T,
\end{verbatim}

with zeros outside S rows/columns.

S is epsilon-light if

\begin{verbatim}
L_S <= epsilon L
\end{verbatim}

in Loewner order.

Question: does there exist universal c\textgreater0 such that for every G and every
epsilon in (0,1), there exists S with \textbar S\textbar{} \textgreater= c\emph{epsilon}n and L\_S \textless= epsilon L?

\subsection{Status of this writeup}

This draft closes the internal algebraic gaps and separates proved facts from
external dependencies:

\begin{enumerate}
\def\labelenumi{\arabic{enumi}.}
\tightlist
\item
  Proved in-text: exact formulation, K\_n upper bound c\textless=1, sampling identities,
  and a correct matrix-concentration setup.
\item
  External dependency: the universal lower bound c0\textgreater0 for all graphs is not
  rederived here; it is treated as an imported theorem assumption.
\end{enumerate}

So the final existential answer is \textbf{conditional/open in this writeup}:
without Assumption V, the universal-c claim is unresolved here.

\subsection{1. Exact reformulation}

The PSD condition is equivalent to the quadratic form inequality

\begin{verbatim}
for all x in R^V: x^T L_S x <= epsilon x^T L x.
\end{verbatim}

On im(L), with L\^{}+ the Moore-Penrose pseudoinverse:

\begin{verbatim}
L_S <= epsilon L  <=>  || L^{+/2} L_S L^{+/2} || <= epsilon.
\end{verbatim}

\subsection{2. Complete graph upper bound (rigorous)}

For G=K\_n and S of size s, choose x supported on S with sum\_\{i in S\} x\_i = 0.
Then

\begin{verbatim}
x^T L_{K_n} x = n ||x||^2,
x^T L_S x = s ||x||^2.
\end{verbatim}

Hence L\_S \textless= epsilon L\_\{K\_n\} implies s \textless= epsilon n.
Therefore any universal constant must satisfy

\begin{verbatim}
c <= 1.
\end{verbatim}

This is an upper bound only.

\subsection{3. Random sampling identities (expectation level)}

Let Z\_v \textasciitilde{} Bernoulli(p) independently and S=\{v: Z\_v=1\}.

\begin{enumerate}
\def\labelenumi{\arabic{enumi}.}
\item
  Size:

  E\textbar S\textbar{} = pn,
  Pr{[}\textbar S\textbar{} \textless{} pn/2{]} \textless= exp(-pn/8) (Chernoff).
\item
  Spectral expectation:

  E{[}L\_S{]} = p\^{}2 L,
\end{enumerate}

since each edge survives with probability p\^{}2.

Thus

\begin{verbatim}
E[epsilon L - L_S] = (epsilon - p^2)L.
\end{verbatim}

Setting p=epsilon gives E{[}L\_S{]}=epsilon\^{}2 L \textless= epsilon L. This is not yet a
realization-level guarantee.

\subsection{4. Concentration setup (gap-fixed formulation)}

Define edge-normalized PSD matrices

\begin{verbatim}
X_e = L^{+/2} w_e b_e b_e^T L^{+/2},   b_e = e_u-e_v,
\end{verbatim}

and leverage scores

\begin{verbatim}
tau_e = tr(X_e) = w_e b_e^T L^+ b_e,
sum_e tau_e = n - k
\end{verbatim}

(k = number of connected components).

\subsubsection{4a. Star domination with correct counting}

Using Z\_u Z\_v \textless= Z\_u and Z\_u Z\_v \textless= Z\_v for each edge \{u,v\},

\begin{verbatim}
L_S = sum_{uv in E} Z_u Z_v L_uv
    <= (1/2) sum_v Z_v sum_{u~v} L_uv.
\end{verbatim}

So in normalized coordinates

\begin{verbatim}
L^{+/2} L_S L^{+/2} <= sum_v Z_v A_v,
A_v := (1/2) sum_{u~v} X_{uv} >= 0.
\end{verbatim}

Because Z\_v are independent Bernoulli variables, the random matrices Z\_v A\_v
are independent PSD summands.

\subsubsection{4b. Freedman/Bernstein martingale parameters}

Let A\_i be a fixed ordering of \{A\_v\}, p\_i=E{[}Z\_i{]}, and define the centered sum

\begin{verbatim}
X = sum_i (Z_i - p_i) A_i.
\end{verbatim}

With filtration F\_i = sigma(Z\_1,...,Z\_i), Doob martingale

\begin{verbatim}
Y_i = E[X | F_i],
Delta_i = Y_i - Y_{i-1}
\end{verbatim}

has self-adjoint differences. For independent Bernoulli sampling,

\begin{verbatim}
Delta_i = (Z_i - p_i) A_i,
||Delta_i|| <= ||A_i|| <= R_*  (R_* = max_i ||A_i||),
\end{verbatim}

and predictable quadratic variation

\begin{verbatim}
W_n = sum_i E[Delta_i^2 | F_{i-1}] = sum_i p_i(1-p_i) A_i^2.
\end{verbatim}

Matrix Freedman (or matrix Bernstein in independent form) applies once bounds
on R\_* and \textbar\textbar W\_n\textbar\textbar{} are supplied.

\textbf{What graph-dependent bounds are needed.} To obtain a self-contained
concentration bound, one would need R\_* \textless= C\_1 * epsilon and
\textbar\textbar W\_n\textbar\textbar{} \textless= C\_2 * epsilon\^{}2 for graph-dependent constants C\_1, C\_2. Bounding
these requires leverage score analysis (showing tau\_e bounds are well-distributed
across vertices) that is the core content of the external theorem referenced
in Section 5. Specifically, the Batson-Spielman-Srivastava barrier-function
method controls both R\_* and \textbar\textbar W\_n\textbar\textbar{} simultaneously through a potential
function that tracks the spectral approximation quality.

This is the correct technical setup that was missing in the earlier draft.

\subsection{5. External dependency for universal c0\textgreater0}

To conclude a universal lower bound for all graphs, one needs an additional
theorem (not proved in this writeup) that controls leverage/pruning and proves:

\begin{verbatim}
exists c0>0 universal such that
for all G, epsilon in (0,1), exists S with |S|>=c0*epsilon*n and L_S<=epsilon L.
\end{verbatim}

We state this explicitly as an assumption:

\begin{verbatim}
Assumption V (vertex-light selection theorem):
There exists c0>0 such that for every weighted graph G and epsilon in (0,1),
one can choose S subseteq V with |S| >= c0*epsilon*n and L_S <= epsilon L.
\end{verbatim}

Literature check in this cycle (MO/MSE local corpus plus the standard
sparsification references) did not locate a theorem matching Assumption V in
this exact vertex-induced, no-reweighting form.

Related evidence comes from the twice-Ramanujan sparsification theorem of
Batson-Spielman-Srivastava (2012, "Twice-Ramanujan Sparsifiers," SIAM Review
56(2), 315-334, Theorem 1.1), but that theorem is about edge selection
with reweighting, not induced subgraphs from vertex subsets.

\textbf{Important gap:} BSS is an \emph{edge sparsification} result --- it selects a
subset of edges with reweighting, not a subset of vertices. The problem asks
for a \emph{vertex subset} S with L\_S \textless= epsilon L. These are different objects:
edge sparsification preserves the vertex set and reweights edges, while the
epsilon-light condition restricts to the induced subgraph on a vertex subset.

The star domination decomposition in Section 4a decomposes L\_S into
vertex-indexed PSD summands, which is the right algebraic setup for a
vertex-selection proof strategy. But the adaptation from edge sparsification
to vertex-induced sparsification is exactly the missing step.

This writeup therefore treats the universal vertex-subset bound as conditional:
Sections 1-4 are proved in-text, and the existential claim is left open unless
Assumption V is supplied from outside.

\subsection{6. Final conclusion (explicitly conditional)}

Unconditional conclusions from this text:

\begin{enumerate}
\def\labelenumi{\arabic{enumi}.}
\tightlist
\item
  The statement is well-posed in Laplacian PSD order.
\item
  K\_n implies universal upper bound c\textless=1.
\item
  The concentration machinery is set up correctly with explicit martingale
  increments and variance process.
\end{enumerate}

Conditional/open conclusion:

\begin{itemize}
\tightlist
\item
  Under Assumption V, one obtains a positive existential result (some universal
  c0\textgreater0 exists).
\item
  Without Assumption V, this writeup does not prove existence of such a
  universal c0; the problem remains open here.
\end{itemize}

\subsection{Key identities used}

\begin{enumerate}
\def\labelenumi{\arabic{enumi}.}
\tightlist
\item
  L = sum\_e w\_e b\_e b\_e\^{}T
\item
  L\_S = sum\_\{e internal to S\} w\_e b\_e b\_e\^{}T
\item
  tau\_e = w\_e b\_e\^{}T L\^{}+ b\_e, sum\_e tau\_e = n-k
\item
  L\^{}\{+/2\} L\_S L\^{}\{+/2\} \textless= sum\_v Z\_v A\_v with A\_v=(1/2)sum\_\{u\textasciitilde v\}X\_\{uv\}
\end{enumerate}

\subsection{References}

\begin{itemize}
\tightlist
\item
  Batson, Spielman, Srivastava (2012), "Twice-Ramanujan Sparsifiers," SIAM
  Review 56(2), 315-334. {[}Theorem 1.1: deterministic edge sparsification;
  related but not equivalent to vertex-induced selection{]}
\item
  Tropp (2011), Freedman\textquotesingle s inequality for matrix martingales
\item
  Standard matrix Bernstein inequality for sums of independent self-adjoint
  random matrices
\end{itemize}
