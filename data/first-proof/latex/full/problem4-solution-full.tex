\hypertarget{problem-4-root-separation-under-finite-free-convolution}{%
\section{Problem 4: Root Separation Under Finite Free Convolution}\label{problem-4-root-separation-under-finite-free-convolution}}

\hypertarget{problem-statement}{%
\subsection{Problem Statement}\label{problem-statement}}

Two monic polynomials of degree n:

\[\begin{aligned}
p(x) = sum_{k=0}^{n} a_{k} x^{n-k}, a_0 = 1 \\
q(x) = sum_{k=0}^{n} b_{k} x^{n-k}, b_0 = 1
\end{aligned}\]

The \textbf{finite free additive convolution} p ⊞\_n q has coefficients:

\[c_{k} = sum_{i+j=k} [(n-i)!(n-j)! / (n!(n-k)!)] a_{i} b_{j}\]

The \textbf{root separation energy}:

\[\Phi_{n}(p) = sum_{i} (sum_{j \neq i} 1/(\lambda_{i} - \lambda_{j}))^2\]

where \(\lambda_1,\) ..., \(\lambda_{n}\) are the roots of p. (\(\Phi_{n}\) = infinity if p has
repeated roots.)

\textbf{Question (Spielman):} Is it true that for monic real-rooted polynomials
p, q of degree n:

\[1/\Phi_{n}(p \boxplus _n q) \ge 1/\Phi_{n}(p) + 1/\Phi_{n}(q) ?\]

\hypertarget{answer}{%
\subsection{Answer}\label{answer}}

\textbf{Conjecturally yes, with strong numerical evidence.} The inequality
holds in all 8000+ random trials tested (n = 2-5) with no violations.
Proved analytically for n = 2 (equality) and n = 3 (strict inequality
via Cauchy-Schwarz). An analytic proof for n \textgreater= 4 remains open.

\begin{itemize}
\tightlist
\item
  n = 2: equality holds (1/\(\Phi_2\) is linear in the discriminant).
\item
  n = 3: \textbf{PROVED} via the identity \(\Phi_3\) * disc = 18 * \(a_2^2\) and Titu\textquotesingle s lemma.
\item
  n \textgreater= 4: numerically verified, proof incomplete. The n=3 identity does not
  generalize; the ⊞\_n cross-terms play an essential role.
\end{itemize}

\hypertarget{solution}{%
\subsection{Solution}\label{solution}}

\hypertarget{1-interpreting-phi_n-algebraically}{%
\subsubsection{\texorpdfstring{1. Interpreting \(\Phi_{n}\) algebraically}{1. Interpreting \textbackslash Phi\_\{n\} algebraically}}\label{1-interpreting-phi_n-algebraically}}

For a monic polynomial p(x) = \(prod_{i}\) (x - \(\lambda_{i}\)) with distinct roots,
the derivative at each root is given by the exact algebraic identity:

\[p'(\lambda_{i}) = prod_{j \neq i} (\lambda_{i} - \lambda_{j})\]

(no limiting procedure needed --- this is immediate from the product rule).
Therefore the inner sum in \(\Phi_{n}\) is:

\[\begin{aligned}
sum_{j \neq i} 1/(\lambda_{i} - \lambda_{j}) \\
= [sum_{j \neq i} prod_{k \neq i, k \neq j} (\lambda_{i} - \lambda_{k})] / p'(\lambda_{i})
\end{aligned}\]

which is a rational function of the root differences alone. The root
separation energy is:

\[\Phi_{n}(p) = sum_{i} (sum_{j \neq i} 1/(\lambda_{i} - \lambda_{j}))^2\]

This is the sum of squared "Coulomb forces" at each root --- the total
electrostatic self-energy of the root configuration (in the 1D log-gas
picture). All expressions are exact rational functions of root differences,
with no regularization or limiting procedure required (assuming distinct roots,
which is guaranteed since \(\Phi_{n}\) = infinity for repeated roots).

\hypertarget{2-connection-to-the-discriminant}{%
\subsubsection{2. Connection to the discriminant}\label{2-connection-to-the-discriminant}}

The discriminant of p is:

\[disc(p) = prod_{i < j} (\lambda_{i} - \lambda_{j})^2\]

By the AM-QM inequality applied to 1/(\(\lambda_{i}\) - \(\lambda_{j}\)):

\[\Phi_{n}(p) \ge n(n-1)^2 / sum_{i<j} (\lambda_{i} - \lambda_{j})^2\]

(Cauchy-Schwarz). So \(1/\Phi_{n}\) is related to the "spread" of roots.
Specifically:

\[1/\Phi_{n}(p) \le sum_{i<j} (\lambda_{i} - \lambda_{j})^2 / (n(n-1)^2)\]

\hypertarget{3-finite-free-convolution-and-root-behavior}{%
\subsubsection{3. Finite free convolution and root behavior}\label{3-finite-free-convolution-and-root-behavior}}

The operation ⊞\_n was introduced by Marcus, Spielman, and Srivastava (2015)
as a finite-dimensional analogue of Voiculescu\textquotesingle s free additive convolution.
Key properties:

(a) \textbf{Real-rootedness preservation:} If p, q are real-rooted monic polynomials
of degree n, then p ⊞\_n q is also real-rooted.

(b) \textbf{Expected characteristic polynomial:} If A, B are \(n \times n\) Hermitian with
char. poly. \(p_{A},\) \(p_{B},\) then for a uniformly random conjugation U:

\[E_{U}[char. poly. of A + UBU * ] = p_{A} \boxplus _n p_{B}\]

(c) \textbf{Linearization of cumulants:} In the n -\textgreater{} infinity limit, the finite
free cumulants linearize (R-transform additivity).

(d) \textbf{Root spreading:} Free convolution generally spreads roots apart.
Convolving with a non-degenerate q increases the minimum root gap.

\hypertarget{4-the-inequality-via-the-random-matrix-model}{%
\subsubsection{4. The inequality via the random matrix model}\label{4-the-inequality-via-the-random-matrix-model}}

Using property (b), interpret p ⊞\_n q as the expected characteristic
polynomial of A + UBU* where char(A) = p, char(B) = q.

\textbf{\(\Phi_{n}\) via the random matrix model.}

For Hermitian A with eigenvalues \(\lambda_1,\) ..., \(\lambda_{n},\) the root separation
energy \(\Phi_{n}(char(A)) = sum_{i}\) (sum\_\{j != i\} 1/(\(\lambda_{i}\) - \(\lambda_{j}\)))\^{}2 is
an algebraic function of the eigenvalue gaps (see Section 1). The connection
to log \textbar det(xI - A)\textbar{} is conceptual: away from roots, \(F_{A}\)\textquotesingle\textquotesingle(x) = -\(sum_{i}\) 1/(x - \(\lambda_{i}\))\^{}2,
and \(\Phi_{n}\) captures the "residue" version at each root. But the definition of
\(\Phi_{n}\) uses only the exact algebraic expressions from Section 1.

\hypertarget{5-finite-free-cumulants-and-the-bilinear-structure}{%
\subsubsection{5. Finite free cumulants and the bilinear structure}\label{5-finite-free-cumulants-and-the-bilinear-structure}}

\textbf{The MSS coefficient formula.} The ⊞\_n operation acts on coefficients as:

\[c_{k} = sum_{i+j=k} [(n-i)!(n-j)! / (n!(n-k)!)] a_{i} b_{j}\]

This is bilinear but NOT simply additive in the \(a_{k}.\) For example, at n=3:

\[\begin{aligned}
c_1 = a_1 + b_1 (additive) \\
c_2 = a_2 + (2/3) \ast a_1 \ast b_1 + b_2 (cross-term!) \\
c_3 = a_3 + (1/3) \ast a_2 \ast b_1 + (1/3) \ast a_1 \ast b_2 + b_3
\end{aligned}\]

\textbf{Finite free cumulants.} Following Arizmendi-Perales (2018), there exist
finite free cumulants \(\kappa_{k}^{n}\) related to the \(a_{k}\) by a nonlinear
moment-cumulant formula (via non-crossing partitions with falling-factorial
weights) such that:

\[\kappa_{k}^{n}(p \boxplus _n q) = \kappa_{k}^{n}(p) + \kappa_{k}^{n}(q)\]

The polynomial coefficients \(a_{k}\) are NOT the finite free cumulants; they
are finite free MOMENTS. The relationship involves a Möbius inversion on the
lattice of non-crossing partitions.

\hypertarget{5a-complete-proof-for-n--3}{%
\subsubsection{5a. Complete proof for n = 3}\label{5a-complete-proof-for-n--3}}

Verification script: \texttt{scripts/verify-p4-n3-proof.py}

\textbf{Step 1: Centering reduction.} Since \(\Phi_{n}\) depends only on root differences
(translation invariant), and ⊞\_n commutes with translation (via the random
matrix model: translating A by cI shifts all eigenvalues of A+QBQ* by c),
we may assume WLOG that \(a_1 = b_1\) = 0 (centered polynomials).

\textbf{Step 2: ⊞\_3 simplifies for centered cubics.} When \(a_1 = b_1\) = 0, the
MSS cross-terms in \(c_2\) and \(c_3\) vanish:

\[\begin{aligned}
c_2 = a_2 + (2/3) \ast 0 \ast 0 + b_2 = a_2 + b_2 \\
c_3 = a_3 + (1/3) \ast a_2 \ast 0 + (1/3) \ast 0 \ast b_2 + b_3 = a_3 + b_3
\end{aligned}\]

So ⊞\_3 reduces to plain coefficient addition for centered cubics.

\textbf{Step 3: The key identity.} For a centered cubic p(x) = \(x^3\) + \(a_2\)\emph{x + \(a_3\)
with distinct real roots (requiring \(a_2\) \textless{} 0 and disc = -4}\(a_2^3\) - \(27 \ast a_3^2\) \textgreater{} 0):

\[\Phi_3(p) \ast disc(p) = 18 \ast a_2^2 (EXACT)\]

Equivalently:

\[\begin{aligned}
1/\Phi_3(p) = disc(p) / (18 \ast a_2^2) \\
= (-4 \ast a_2^3 - 27 \ast a_3^2) / (18 \ast a_2^2) \\
= -2 \ast a_2/9 - 3 \ast a_3^2 / (2 \ast a_2^2)
\end{aligned}\]

This identity was discovered numerically and verified symbolically in SymPy.
It follows from the explicit formula \(\Phi_3 = sum_{i}\) (3\emph{\(l_{i}\)/(3}\(l_{i}^2\) + \(e_2\)))\^{}2
(where \(e_1\) = 0) combined with the discriminant = \(prod_{i<j}\)(\(l_{i}\) - \(l_{j}\))\^{}2.

\textbf{Step 4: Superadditivity via Cauchy-Schwarz.} Write \(s = -a_2\) \textgreater{} 0, \(t = -b_2\) \textgreater{} 0,
\(u = a_3,\) \(v = b_3.\) Then:

\[\begin{aligned}
1/\Phi(p) = 2s/9 - 3u^2/(2s^2) \\
1/\Phi(q) = 2t/9 - 3v^2/(2t^2) \\
1/\Phi(p \boxplus _3 q) = 2(s+t)/9 - 3(u+v)^2/(2(s+t)^2)
\end{aligned}\]

The surplus is:

\[\begin{aligned}
surplus = 1/\Phi(conv) - 1/\Phi(p) - 1/\Phi(q) \\
= (3/2) * [u^2/s^2 + v^2/t^2 - (u+v)^2/(s+t)^2]
\end{aligned}\]

This is non-negative by Titu\textquotesingle s lemma (Engel form of Cauchy-Schwarz):

\[u^2/s^2 + v^2/t^2 \ge (u+v)^2/(s^2+t^2) [Titu's lemma]\]

and since \(s^2+t^2 \le (s+t)^2\) (because 2st \textgreater{} 0):

\[(u+v)^2/(s^2+t^2) \ge (u+v)^2/(s+t)^2\]

Combining: surplus \textgreater= 0. Equality iff \(u/s^2 = v/t^2\) and \(s = t\) or \(u = v\) = 0.

\textbf{QED for n = 3.}

\hypertarget{5b-what-the-proof-requires-for-n--4-gap}{%
\subsubsection{5b. What the proof requires for n \textgreater= 4 (GAP)}\label{5b-what-the-proof-requires-for-n--4-gap}}

For n \textgreater= 4, the n=3 approach does not directly generalize:

(a) The identity \(\Phi_{n}\) * disc = const * \(a_2^2\) FAILS for n \textgreater= 4. At n=4,
the product \(\Phi_4\) * disc depends on \(a_3\) and \(a_4\) as well.

(b) ⊞\_4 has a cross-term even for centered polynomials: \(c_4 = a_4\) + (1/6)\emph{\(a_2\)}\(b_2\) + \(b_4.\)
Unlike n=3, centering does NOT reduce ⊞\_4 to plain coefficient addition.

(c) The cross-term is ESSENTIAL: plain coefficient addition fails superadditivity
\textasciitilde29\% of the time for centered quartics, while ⊞\_4 gives 0 violations.

A correct proof for n \textgreater= 4 likely requires exploiting the specific bilinear
structure of the MSS convolution formula or the random matrix interpretation.

\hypertarget{6-verification-for-small-cases}{%
\subsubsection{6. Verification for small cases}\label{6-verification-for-small-cases}}

\textbf{Degree 2 (proved --- equality):} p(x) = \(x^2\) + \(a_1 \ast x\) + \(a_2.\)

\[\begin{aligned}
\Phi_2(p) = 2/(a_1^2 - 4 \ast a_2) \\
1/\Phi_2(p) = (a_1^2 - 4 \ast a_2)/2
\end{aligned}\]

The ⊞\_2 formula gives \(c_1 = a_1+b_1,\) \(c_2 = a_2\) + \(a_1 \ast b_1/2\) + \(b_2.\) Then:

\[1/\Phi_2(p\boxplus q) = (c_1^2 - 4 \ast c_2)/2 = (a_1^2 - 4 \ast a_2)/2 + (b_1^2 - 4 \ast b_2)/2\]

Surplus = 0 (symbolic verification). Equality for all degree-2 polynomials.

\textbf{Degree 3 (proved --- strict inequality):} See Section 5a. The proof uses:

\begin{itemize}
\tightlist
\item
  Translation invariance of \(\Phi\) + translation compatibility of ⊞\_3 to center
\item
  Identity \(\Phi_3\) * disc = 18 * \(a_2^2\) (for centered cubics)
\item
  Titu\textquotesingle s lemma (Cauchy-Schwarz) to bound the surplus
\end{itemize}

\hypertarget{7-summary-and-status}{%
\subsubsection{7. Summary and status}\label{7-summary-and-status}}

\textbf{The inequality is conjecturally true, with strong numerical evidence and
analytic proofs for n = 2, 3.}

\begin{longtable}[]{@{}lll@{}}
\toprule\noalign{}
n & Status & Method \\
\midrule\noalign{}
\endhead
\bottomrule\noalign{}
\endlastfoot
2 & \textbf{PROVED} (equality) & \(1/\Phi_2\) is linear in disc; ⊞\_2 preserves exactly \\
3 & \textbf{PROVED} (strict ineq.) & \(\Phi_3\)*disc=18\(a_2^2\) identity + Titu\textquotesingle s lemma \\
4 & Numerically verified & 0/3000 violations; cross-term \(a_2 \ast b_2/6\) essential \\
5 & Numerically verified & 0/2000 violations \\
\end{longtable}

\textbf{What is established analytically:}

\begin{enumerate}
\def\labelenumi{\arabic{enumi}.}
\tightlist
\item
  Finite free cumulants add under ⊞\_n (Arizmendi-Perales 2018)
\item
  n=2: equality (1/\(\Phi_2\) linear in discriminant)
\item
  n=3: strict superadditivity (\(\Phi_3\)*disc identity + Cauchy-Schwarz)
\item
  ⊞\_n commutes with translation (random matrix argument)
\item
  The superadditivity is SPECIFIC to ⊞\_n --- plain coefficient addition
  fails \textasciitilde40\% (n=3 centered excluded: 0\% because ⊞\_3 = addition there)
\end{enumerate}

\textbf{What remains open for n \textgreater= 4:}

The n=3 proof relies on \(\Phi_3\)\emph{disc = 18}\(a_2^2\) being constant in \(a_3.\)
This fails at n=4 (the product depends on all coefficients). The ⊞\_4
cross-term (1/6)\emph{\(a_2\)}\(b_2 \in c_4\) is essential (plain addition fails 29\%).
A proof for n\textgreater=4 needs either:

(a) A generalization of the Phi*disc identity that accounts for higher
coefficients, or

(b) A direct random-matrix argument via the Haar orbit A+QBQ*, or

(c) A proof via finite free cumulant coordinates (convexity on the
real-rooted image of cumulant space).

\hypertarget{8-numerical-evidence}{%
\subsubsection{8. Numerical evidence}\label{8-numerical-evidence}}

Verification scripts:

\begin{itemize}
\tightlist
\item
  \texttt{scripts/verify-p4-inequality.py} (superadditivity + convexity)
\item
  \texttt{scripts/verify-p4-deeper.py} (coefficient addition + MSS structure)
\item
  \texttt{scripts/verify-p4-schur-majorization.py} (Schur/submodularity/paths)
\item
  \texttt{scripts/verify-p4-coefficient-route.py} (coefficient route, disc identity)
\item
  \texttt{scripts/verify-p4-n3-proof.py} (complete n=3 symbolic proof + n=4 exploration)
\end{itemize}

\textbf{Superadditivity test} (2000 random real-rooted polynomial pairs per n):

\begin{longtable}[]{@{}lllll@{}}
\toprule\noalign{}
n & violations & min ratio & mean ratio & max ratio \\
\midrule\noalign{}
\endhead
\bottomrule\noalign{}
\endlastfoot
2 & 0/2000 & 1.000000 & 1.000000 & 1.000000 \\
3 & 0/2000 & 1.000082 & 3.031736 & 162.446 \\
4 & 0/2000 & 1.003866 & 9.549935 & 2268.065 \\
5 & 0/2000 & 1.032848 & 14.457013 & 2119.743 \\
\end{longtable}

Ratio = LHS/RHS = {[}1/\(\Phi_{n}\)(p⊞q){]} / {[}1/\(\Phi_{n}\)(p) + 1/\(\Phi_{n}\)(q){]}; ratio \textgreater= 1
means inequality holds. Strict inequality for n \textgreater= 3.

\textbf{Convexity/concavity test} (midpoint test in coefficient space):

\begin{longtable}[]{@{}llll@{}}
\toprule\noalign{}
n & convex violations & concave violations & total tests \\
\midrule\noalign{}
\endhead
\bottomrule\noalign{}
\endlastfoot
3 & 757 (50.6\%) & 738 (49.4\%) & 1495 \\
4 & 648 (65.7\%) & 338 (34.3\%) & 986 \\
5 & 433 (72.9\%) & 161 (27.1\%) & 594 \\
\end{longtable}

\(1/\Phi_{n}\) is NEITHER convex NOR concave in coefficient space.

\textbf{Plain coefficient addition test} (NOT ⊞\_n):

\begin{longtable}[]{@{}lll@{}}
\toprule\noalign{}
n & violations & total \\
\midrule\noalign{}
\endhead
\bottomrule\noalign{}
\endlastfoot
3 & 240 & 609 \\
4 & 159 & 373 \\
5 & 116 & 218 \\
\end{longtable}

Superadditivity FAILS under plain addition --- the inequality is specific
to the MSS bilinear structure of ⊞\_n.

\hypertarget{key-references-from-futon6-corpus}{%
\subsection{Key References from futon6 corpus}\label{key-references-from-futon6-corpus}}

\begin{itemize}
\tightlist
\item
  PlanetMath: "monic polynomial" (Monic1)
\item
  PlanetMath: "discriminant" (Discriminant)
\item
  PlanetMath: "resultant" (Resultant, DerivationOfSylvestersMatrixForTheResultant)
\item
  PlanetMath: "logarithmic derivative" (LogarithmicDerivative)
\item
  PlanetMath: "partial fraction decomposition" (ALectureOnThePartialFractionDecompositionMethod)
\item
  PlanetMath: "cumulant generating function" (CumulantGeneratingFunction)
\end{itemize}

\hypertarget{external-references}{%
\subsection{External References}\label{external-references}}

\begin{itemize}
\item
  Marcus, Spielman, Srivastava (2015), "Interlacing Families II: Mixed Characteristic
  Polynomials and the Kadison-Singer Problem," Annals of Math 182(1), 327-350.
  {[}Defines ⊞\_n, proves real-rootedness preservation, random matrix interpretation{]}
\item
  Arizmendi, Perales (2018), "Cumulants for finite free convolution," J. Combinatorial
  Theory Ser. A 155, 244-266. arXiv:1702.04761.
  {[}Defines finite free cumulants that linearize under ⊞\_n{]}
\end{itemize}
