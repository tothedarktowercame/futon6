\hypertarget{problem-4-root-separation-under-finite-free-convolution}{%
\section{Problem 4: Root Separation Under Finite Free Convolution}\label{problem-4-root-separation-under-finite-free-convolution}}

\hypertarget{problem-statement}{%
\subsection{Problem Statement}\label{problem-statement}}

Two monic polynomials of degree n:

\[\begin{aligned}
p(x) = \sum_{k=\mNumber{0}}^{n} a_{k} x^{n-k}, a_\mNumber{0} = \mNumber{1} \\
q(x) = \sum_{k=\mNumber{0}}^{n} b_{k} x^{n-k}, b_\mNumber{0} = \mNumber{1}
\end{aligned}\]

The \textbf{finite free additive convolution} p \(\boxplus_{n}\) q has coefficients:

\[c_{k} = \sum_{i+j=k} [(n-i)!(n-j)! / (n!(n-k)!)] a_{i} b_{j}\]

The \textbf{root separation energy}:

\[\Phi_{n}(p) = \sum_{i} (\sum_{j \neq i} \mNumber{1}/(\lambda_{i} - \lambda_{j}))^\mNumber{2}\]

where \(\lambda_{\mNumber{1}}, ..., \lambda_{n}\) are the roots of p. (\(\Phi_{n} = \infty\) if p has
repeated roots.)

\textbf{Question (Spielman):} Is it true that for monic real-rooted polynomials
p, q of degree n:

\[\mNumber{1}/\Phi_{n}(p \boxplus_{n} q) \ge \mNumber{1}/\Phi_{n}(p) + \mNumber{1}/\Phi_{n}(q) ?\]

\hypertarget{answer}{%
\subsection{Answer}\label{answer}}

\textbf{Conjecturally yes, with strong numerical evidence.} The inequality
holds in all 8000+ random trials tested (n \(=\) 2-5) with no violations.
Proved analytically for \(n = \mNumber{2}\) (equality) and \(n = \mNumber{3}\) (nonnegative surplus
for all centered cubics, with strict inequality generically, via
Cauchy-Schwarz). An analytic proof for \(n \ge \mNumber{4}\) remains open.

\begin{itemize}
\tightlist
\item
  \(n = \mNumber{2}\): equality holds (1/\(\Phi_\mNumber{2}\) is linear in the discriminant).
\item
  \(n = \mNumber{3}\): \textbf{PROVED} via the identity \(\Phi_\mNumber{3}\) * disc \(= \mNumber{18} \ast a_{\mNumber{2}}^{\mNumber{2}}\) and Titu\textquotesingle s lemma.
\item
  \(n \ge \mNumber{4}\): numerically verified, proof incomplete. The \(n = \mNumber{3}\) identity does not
  generalize; the \(\boxplus_{n}\) cross-terms play an essential role.
\end{itemize}

\hypertarget{solution}{%
\subsection{Solution}\label{solution}}

\hypertarget{1-interpreting-phi_n-algebraically}{%
\subsubsection{\texorpdfstring{1. Interpreting \(\Phi_{n}\) algebraically}{1. Interpreting \textbackslash Phi\_\{n\} algebraically}}\label{1-interpreting-phi_n-algebraically}}

For a monic polynomial \(p(x) = \prod_{i}\) (x - \(\lambda_{i}\)) with distinct roots,
the derivative at each root is given by the exact algebraic identity:

\[p'(\lambda_{i}) = \prod_{j \neq i} (\lambda_{i} - \lambda_{j})\]

(no limiting procedure needed --- this is immediate from the product rule).
Therefore the inner sum in \(\Phi_{n}\) is:

\[\begin{aligned}
\sum_{j \neq i} \mNumber{1}/(\lambda_{i} - \lambda_{j}) \\
= [\sum_{j \neq i} \prod_{k \neq i, k \neq j} (\lambda_{i} - \lambda_{k})] / p'(\lambda_{i})
\end{aligned}\]

which is a rational function of the root differences alone. The root
separation energy is:

\[\Phi_{n}(p) = \sum_{i} (\sum_{j \neq i} \mNumber{1}/(\lambda_{i} - \lambda_{j}))^\mNumber{2}\]

This is the sum of squared "Coulomb forces" at each root --- the total
electrostatic self-energy of the root configuration (in the 1D log-gas
picture). All expressions are exact rational functions of root differences,
with no regularization or limiting procedure required (assuming distinct roots,
which is guaranteed since \(\Phi_{n} = \infty\) for repeated roots).

\hypertarget{2-connection-to-the-discriminant}{%
\subsubsection{2. Connection to the discriminant}\label{2-connection-to-the-discriminant}}

The discriminant of p is:

\[\mOpName{disc}(p) = \prod_{i < j} (\lambda_{i} - \lambda_{j})^\mNumber{2}\]

By the AM-QM inequality applied to 1/(\(\lambda_{i} - \lambda_{j}\)):

\[\Phi_{n}(p) \ge n(n-\mNumber{1})^\mNumber{2} / \sum_{i < j} (\lambda_{i} - \lambda_{j})^\mNumber{2}\]

(Cauchy-Schwarz). So \(\mNumber{1}/\Phi_{n}\) is related to the "spread" of roots.
Specifically:

\[\mNumber{1}/\Phi_{n}(p) \le \sum_{i < j} (\lambda_{i} - \lambda_{j})^\mNumber{2} / (n(n-\mNumber{1})^\mNumber{2})\]

\hypertarget{3-finite-free-convolution-and-root-behavior}{%
\subsubsection{3. Finite free convolution and root behavior}\label{3-finite-free-convolution-and-root-behavior}}

The operation \(\boxplus_{n}\) was introduced by Marcus, Spielman, and Srivastava (2015)
as a finite-dimensional analogue of Voiculescu\textquotesingle s free additive convolution.
Key properties:

(a) \textbf{Real-rootedness preservation:} If p, q are real-rooted monic polynomials
of degree n, then p \(\boxplus_{n}\) q is also real-rooted.

(b) \textbf{Expected characteristic polynomial:} If A, B are \(n \times n\) Hermitian with
char. poly. \(p_{A}, p_{B},\) then for a uniformly random conjugation U:

\[E_{U}[\mOpName{char}. \text{poly}. \text{of} A + \mathup{UBU} * ] = p_{A} \boxplus_{n} p_{B}\]

(c) \textbf{Linearization of cumulants:} In the \(n \to \infty\) limit, the finite
free cumulants linearize (R-transform additivity).

(d) \textbf{Root spreading:} Free convolution generally spreads roots apart.
Convolving with a non-degenerate q increases the minimum root gap.

\hypertarget{4-the-inequality-via-the-random-matrix-model}{%
\subsubsection{4. The inequality via the random matrix model}\label{4-the-inequality-via-the-random-matrix-model}}

Using property (b), interpret p \(\boxplus_{n}\) q as the expected characteristic
polynomial of A + UBU* where \(\mOpName{char}(A) = p, \mOpName{char}(B) = q.\)

\textbf{\(\Phi_{n}\) via the random matrix model.}

For Hermitian A with eigenvalues \(\lambda_{\mNumber{1}}, ..., \lambda_{n},\) the root separation
energy \(\Phi_{n}(\mOpName{char}(A)) = \sum_{i} (\sum_{j \neq i} \mNumber{1}/(\lambda_{i} - \lambda_{j}))^{\mNumber{2}}\) is
an algebraic function of the eigenvalue gaps (see Section 1). The connection
to log \textbar det(xI - A)\textbar{} is conceptual: away from roots, \(F_{A}''(x) = -\sum_{i} \mNumber{1}/(x - \lambda_{i})^\mNumber{2},\)
and \(\Phi_{n}\) captures the "residue" version at each root. But the definition of
\(\Phi_{n}\) uses only the exact algebraic expressions from Section 1.

\hypertarget{5-finite-free-cumulants-and-the-bilinear-structure}{%
\subsubsection{5. Finite free cumulants and the bilinear structure}\label{5-finite-free-cumulants-and-the-bilinear-structure}}

\textbf{The MSS coefficient formula.} The \(\boxplus_{n}\) operation acts on coefficients as:

\[c_{k} = \sum_{i+j=k} [(n-i)!(n-j)! / (n!(n-k)!)] a_{i} b_{j}\]

This is bilinear but NOT simply additive in the \(a_{k}\). For example, at \(n = \mNumber{3}\):

\[\begin{aligned}
c_\mNumber{1} = a_\mNumber{1} + b_\mNumber{1} (\text{additive}) \\
c_\mNumber{2} = a_\mNumber{2} + (\mNumber{2}/\mNumber{3}) \ast a_\mNumber{1} \ast b_\mNumber{1} + b_\mNumber{2} (\text{cross}-\text{term}!) \\
c_\mNumber{3} = a_\mNumber{3} + (\mNumber{1}/\mNumber{3}) \ast a_\mNumber{2} \ast b_\mNumber{1} + (\mNumber{1}/\mNumber{3}) \ast a_\mNumber{1} \ast b_\mNumber{2} + b_\mNumber{3}
\end{aligned}\]

\textbf{Finite free cumulants.} Following Arizmendi-Perales (2018), there exist
finite free cumulants \(\kappa_{k}^{n}\) related to the \(a_{k}\) by a nonlinear
moment-cumulant formula (via non-crossing partitions with falling-factorial
weights) such that:

\[\kappa_{k}^{n}(p \boxplus_{n} q) = \kappa_{k}^{n}(p) + \kappa_{k}^{n}(q)\]

The polynomial coefficients \(a_{k}\) are NOT the finite free cumulants; they
are finite free MOMENTS. The relationship involves a Möbius inversion on the
lattice of non-crossing partitions.

\hypertarget{5a-complete-proof-for-n--3}{%
\subsubsection{\texorpdfstring{5a. Complete proof for \(n = \mNumber{3}\)}{5a. Complete proof for n = \textbackslash mNumber\{3\}}}\label{5a-complete-proof-for-n--3}}

Verification script: \texttt{scripts/verify-p4-n3-proof.py}

\textbf{Step 1: Centering reduction.} Since \(\Phi_{n}\) depends only on root differences
(translation invariant), and \(\boxplus_{n}\) commutes with translation (via the random
matrix model: translating A by cI shifts all eigenvalues of A+QBQ* by c),
we may assume WLOG that \(a_\mNumber{1} = b_\mNumber{1} = \mNumber{0}\) (centered polynomials).

\textbf{Step 2: \(\boxplus_\mNumber{3}\) simplifies for centered cubics.} When \(a_\mNumber{1} = b_\mNumber{1} = \mNumber{0}\), the
MSS cross-terms in \(c_\mNumber{2}\) and \(c_\mNumber{3}\) vanish:

\[\begin{aligned}
c_\mNumber{2} = a_\mNumber{2} + (\mNumber{2}/\mNumber{3}) \ast \mNumber{0} \ast \mNumber{0} + b_\mNumber{2} = a_\mNumber{2} + b_\mNumber{2} \\
c_\mNumber{3} = a_\mNumber{3} + (\mNumber{1}/\mNumber{3}) \ast a_\mNumber{2} \ast \mNumber{0} + (\mNumber{1}/\mNumber{3}) \ast \mNumber{0} \ast b_\mNumber{2} + b_\mNumber{3} = a_\mNumber{3} + b_\mNumber{3}
\end{aligned}\]

So \(\boxplus_\mNumber{3}\) reduces to plain coefficient addition for centered cubics.

\textbf{Step 3: The key identity.} For a centered cubic \(p(x) = x^\mNumber{3} \mBridgeOperator{+} a_\mNumber{2} \ast x \mBridgeOperator{+} a_\mNumber{3}\)
with distinct real roots (requiring \(a_\mNumber{2} < \mNumber{0}\) and disc \(= -\mNumber{4} \ast a_{\mNumber{2}}^{\mNumber{3}} - \mNumber{27} \ast a_{\mNumber{3}}^{\mNumber{2}} > \mNumber{0}\)):

\[\Phi_\mNumber{3}(p) \ast \mOpName{disc}(p) = \mNumber{18} \ast a_\mNumber{2}^\mNumber{2} (\text{EXACT})\]

Equivalently:

\[\begin{aligned}
\mNumber{1}/\Phi_\mNumber{3}(p) = \mOpName{disc}(p) / (\mNumber{18} \ast a_\mNumber{2}^\mNumber{2}) \\
= (-\mNumber{4} \ast a_\mNumber{2}^\mNumber{3} - \mNumber{27} \ast a_\mNumber{3}^\mNumber{2}) / (\mNumber{18} \ast a_\mNumber{2}^\mNumber{2}) \\
= -\mNumber{2} \ast a_\mNumber{2}/\mNumber{9} - \mNumber{3} \ast a_\mNumber{3}^\mNumber{2} / (\mNumber{2} \ast a_\mNumber{2}^\mNumber{2})
\end{aligned}\]

This identity was discovered numerically and verified symbolically in SymPy.
It follows from the explicit formula \(\Phi_{\mNumber{3}} = \sum_{i} (\mNumber{3} \ast l_{i}/(\mNumber{3} \ast l_{i}^{\mNumber{2}} \mBridgeOperator{+} e_{\mNumber{2}}))^\mNumber{2}\)
(where \(e_\mNumber{1} = \mNumber{0}\)) combined with the discriminant \(= \prod_{i < j}(l_{i} - l_{j})^{\mNumber{2}}.\)

\textbf{Step 4: Superadditivity via Cauchy-Schwarz.} Write s \(= -a_{\mNumber{2}} > \mNumber{0}, t = -b_{\mNumber{2}} > \mNumber{0}, u = a_{\mNumber{3}}, v = b_{\mNumber{3}}.\) Then:

\[\begin{aligned}
\mNumber{1}/\Phi(p) = \mNumber{2}s/\mNumber{9} - \mNumber{3}u^\mNumber{2}/(\mNumber{2}s^\mNumber{2}) \\
\mNumber{1}/\Phi(q) = \mNumber{2}t/\mNumber{9} - \mNumber{3}v^\mNumber{2}/(\mNumber{2}t^\mNumber{2}) \\
\mNumber{1}/\Phi(p \boxplus_\mNumber{3} q) = \mNumber{2}(s+t)/\mNumber{9} - \mNumber{3}(u+v)^\mNumber{2}/(\mNumber{2}(s+t)^\mNumber{2})
\end{aligned}\]

The surplus is:

\[\begin{aligned}
\text{surplus} = \mNumber{1}/\Phi(\text{conv}) - \mNumber{1}/\Phi(p) - \mNumber{1}/\Phi(q) \\
= (\mNumber{3}/\mNumber{2}) \ast [u^\mNumber{2}/s^\mNumber{2} + v^\mNumber{2}/t^\mNumber{2} - (u+v)^\mNumber{2}/(s+t)^\mNumber{2}]
\end{aligned}\]

This is non-negative by Titu\textquotesingle s lemma (Engel form of Cauchy-Schwarz):

\[u^\mNumber{2}/s^\mNumber{2} + v^\mNumber{2}/t^\mNumber{2} \ge (u+v)^\mNumber{2}/(s^\mNumber{2}+t^\mNumber{2}) [\text{Titu}'s \text{lemma}]\]

and since \(s^\mNumber{2}+t^\mNumber{2} \le (s+t)^\mNumber{2}\) (because 2st \(>\) 0):

\[(u+v)^\mNumber{2}/(s^\mNumber{2}+t^\mNumber{2}) \ge (u+v)^\mNumber{2}/(s+t)^\mNumber{2}\]

Combining: \(\text{surplus} \ge \mNumber{0}\). Equality iff \(u/s^\mNumber{2} = v/t^\mNumber{2}\) and \(s = t\) or \(u = v = \mNumber{0}\).

\textbf{QED for \(n = \mNumber{3}\).}

\hypertarget{5b-what-the-proof-requires-for-n--4-gap}{%
\subsubsection{\texorpdfstring{5b. What the proof requires for \(n \ge \mNumber{4}\) (GAP)}{5b. What the proof requires for n \textbackslash ge \textbackslash mNumber\{4\} (GAP)}}\label{5b-what-the-proof-requires-for-n--4-gap}}

For \(n \ge \mNumber{4}\), the \(n = \mNumber{3}\) approach does not directly generalize:

(a) The identity \(\Phi_{n}\) * disc \(=\) const * \(a_\mNumber{2}^\mNumber{2}\) FAILS for \(n \ge \mNumber{4}\). At \(n = \mNumber{4}\),
the product \(\Phi_\mNumber{4}\) * disc depends on \(a_\mNumber{3}\) and \(a_\mNumber{4}\) as well.

(b) \(\boxplus_\mNumber{4}\) has a cross-term even for centered polynomials: \(c_\mNumber{4} = a_\mNumber{4} \mBridgeOperator{+} (\mNumber{1}/\mNumber{6}) \ast a_\mNumber{2} \ast b_\mNumber{2} \mBridgeOperator{+} b_\mNumber{4}\).
Unlike \(n = \mNumber{3}\), centering does NOT reduce \(\boxplus_\mNumber{4}\) to plain coefficient addition.

(c) The cross-term is ESSENTIAL: plain coefficient addition fails superadditivity
\textasciitilde29\% of the time for centered quartics, while \(\boxplus_\mNumber{4}\) gives 0 violations.

A correct proof for \(n \ge \mNumber{4}\) likely requires exploiting the specific bilinear
structure of the MSS convolution formula or the random matrix interpretation.

\hypertarget{5c-the-hermite-heat-semigroup-and-coulomb-flow}{%
\subsubsection{5c. The Hermite heat semigroup and Coulomb flow (NEW --- PROVED)}\label{5c-the-hermite-heat-semigroup-and-coulomb-flow}}

The following results establish the finite analog of the free de Bruijn
identity by a purely algebraic argument via the backward heat equation.

\textbf{Setup.} Let $\mathrm{He}_n$ denote the degree-$n$ probabilist's Hermite polynomial,
and define the \emph{scaled Hermite polynomial}
$\mathrm{He}_t(x) = t^{n/2}\,\mathrm{He}_n(x/\sqrt{t})$.
The \emph{Hermite heat semigroup} is the one-parameter family
$p_t = p \boxplus_n \mathrm{He}_t$.

\textbf{Lemma 1 (Hermite backward heat equation).}
$\partial_t \mathrm{He}_t(x) = -\tfrac{1}{2}\,\partial_x^2 \mathrm{He}_t(x)$.

\emph{Proof.} Write $\mathrm{He}_t(x) = \sum_{m=0}^{\lfloor n/2 \rfloor} C_m\, t^m\, x^{n-2m}$
where $C_m = (-1)^m \, n! / (m!\, 2^m\, (n-2m)!)$.
Comparing the coefficient of $t^{m-1} x^{n-2m}$ on both sides reduces to
the Hermite recurrence identity:
$m\, C_m = -\tfrac{1}{2}(n-2m+2)(n-2m+1)\,C_{m-1}$,
which follows from $C_m/C_{m-1} = -(n-2m+2)(n-2m+1)/(2m)$.
Verified symbolically for $n = 2, \ldots, 7$.\quad$\square$

\textbf{Theorem 1 (Backward heat equation for $\boxplus_n$).}
$\partial_t \, p_t(x) = -\tfrac{1}{2}\,\partial_x^2\, p_t(x)$.

That is, $p_t = p \boxplus_n \mathrm{He}_t$ satisfies the backward heat equation as
a polynomial in $x$ and $t$.

\emph{Proof.} Write $p_t(x) = x^n + \sum_{k=1}^n c_k(t)\,x^{n-k}$ where
$c_k(t) = \sum_{i+j=k} w(n,i,j)\,a_i\,h_j(t)$,
with $w(n,i,j) = \frac{(n-i)!(n-j)!}{n!(n-i-j)!}$ the MSS weights and
$h_j(t)$ the coefficients of $\mathrm{He}_t$.

By Lemma~1, $dh_j/dt = -\tfrac{1}{2}(n-j+2)(n-j+1)\,h_{j-2}(t)$.
Substituting into $dc_k/dt$ and comparing with the coefficient of $x^{n-k}$
in $-\tfrac{1}{2}\,p_t''$ reduces the entire identity to:

\textbf{MSS Weight Identity:}
\[
  w(n,i,j)\,(n-j+2)(n-j+1) \;=\; (n-i-j+2)(n-i-j+1)\,w(n,i,j-2).
\]

\emph{Proof of weight identity.} Both sides equal
$\frac{(n-i)!\,(n-j+2)!}{n!\,(n-i-j)!}$,
since the LHS multiplies $(n-j)!$ by $(n-j+2)(n-j+1)$ to get $(n-j+2)!$,
and the RHS cancels $(n-i-j+2)(n-i-j+1)$ from $(n-i-j+2)!$ to get $(n-i-j)!$.
Verified exhaustively for $n = 2, \ldots, 9$.\quad$\square$

\textbf{Theorem 2 (Finite Coulomb Flow).} Let $\gamma_1(t) < \cdots < \gamma_n(t)$
be the roots of $p_t$. Then:
\[
  \frac{d\gamma_k}{dt} = S_k(\gamma) := \sum_{j \neq k} \frac{1}{\gamma_k - \gamma_j}.
\]

\emph{Proof.} Implicit differentiation of $p_t(\gamma_k(t), t) = 0$ gives
$\gamma_k' = -(\partial_t p_t)(\gamma_k) / p_t'(\gamma_k)$.
By Theorem~1, this equals $\tfrac{1}{2}\,p_t''(\gamma_k)/p_t'(\gamma_k)$.
The standard root identity $p''(\gamma_k) = 2\,p'(\gamma_k)\,S_k(\gamma)$
(which follows from $p''(x) = \sum_{i \neq j} \prod_{l \neq i,j}(x - \gamma_l)$
evaluated at $x = \gamma_k$) gives $\gamma_k' = S_k(\gamma)$.\quad$\square$

Verified numerically: fitted proportionality constant $c = 1.00000006 \pm 2 \times 10^{-8}$
for $n = 3, 4, 5$ (30 random trials each, max relative error $< 10^{-7}$).

\textbf{Corollary 1 (Finite De Bruijn Identity).} Let
$H'_n(p) = \sum_{i<j} \log|\lambda_i - \lambda_j|$ (unnormalized log-discriminant).
Then along the Hermite heat semigroup:
\[
  \frac{d}{dt}\, H'_n(p_t) = \Phi_n(p_t).
\]

\emph{Proof.} By the chain rule:
$\frac{d}{dt} H'_n = \sum_{i<j} \frac{\gamma_i' - \gamma_j'}{\gamma_i - \gamma_j}
= \sum_k \gamma_k' \cdot S_k = \sum_k S_k^2 = \Phi_n$,
where the rearrangement step pairs each difference $\gamma_i - \gamma_j$ with
the corresponding term in $S_k$, and the last equality uses Theorem~2.\quad$\square$

This is the finite analog of Voiculescu's free de Bruijn identity
$\frac{d}{dt}\chi(\mu_t) = \tfrac{1}{2}\Phi^*(\mu_t)$.

\textbf{Theorem 3 ($\Phi_n$ monotonicity along the Coulomb flow).}
Along the Hermite heat semigroup:
\[
  \frac{d\Phi_n}{dt} = -2 \sum_{k < j} \frac{(S_k - S_j)^2}{(\gamma_k - \gamma_j)^2} \;\le\; 0,
\]
with equality only at the Hermite equilibrium (where all $S_k$ are equal).

\emph{Proof.} From $\Phi_n = \sum_k S_k^2$, the chain rule gives
$\frac{d\Phi_n}{dt} = 2\sum_k S_k\,\dot{S}_k$
where $\dot{S}_k = \sum_{j \neq k} \frac{-(S_k - S_j)}{(\gamma_k - \gamma_j)^2}$
(using $\dot{\gamma}_k = S_k$ from Theorem~2).
Thus $\frac{d\Phi_n}{dt} = -2\sum_{k \neq j} \frac{S_k(S_k - S_j)}{(\gamma_k - \gamma_j)^2}$.
Symmetrizing by pairing $(k,j)$ with $(j,k)$:
$S_k(S_k - S_j) + S_j(S_j - S_k) = (S_k - S_j)^2$,
giving the claimed sum-of-squares formula.\quad$\square$

\textbf{Corollary 2.} $1/\Phi_n(p_t)$ is strictly increasing in $t$. Verified
numerically across 3480 consecutive time pairs for $n = 3, 4, 5, 6$ (0 violations).

\textbf{Remark (Hermite semigroup property).} The scaled Hermite polynomials satisfy
$\mathrm{He}_s \boxplus_n \mathrm{He}_t = \mathrm{He}_{s+t}$
(the Hermite family forms a semigroup under $\boxplus_n$). This follows from the
coefficient formula and is verified numerically to machine precision. Thus the
Hermite heat semigroup $t \mapsto p_t$ is a genuine semigroup:
$p_{s+t} = p_s \boxplus_n \mathrm{He}_t$.

\textbf{Remark (Universal de Bruijn constant).} For a general heat kernel $h$ (not
necessarily Hermite), we have $\frac{d}{dt} H'_n(p \boxplus_n h_t) = \frac{d\sigma_h^2/dt}{n-1}\,\Phi_n$,
where $\sigma_h^2$ is the variance of the kernel roots. For Hermite, $d\sigma_h^2/dt = n-1$,
giving the constant $1$. Verified for Hermite, equally-spaced, and Chebyshev kernels.

\hypertarget{5d-implications-for-the-n-geq-4-gap}{%
\subsubsection{\texorpdfstring{5d. Implications for the $n \ge 4$ gap}{5d. Implications for the n >= 4 gap}}\label{5d-implications-for-the-n-geq-4-gap}}

The results of Section 5c establish the ``de Bruijn half'' of the proof strategy
for the Stam inequality. The classical proof of Stam's inequality uses:
\begin{enumerate}
\def\labelenumi{(\arabic{enumi})}
\tightlist
\item De Bruijn identity: $dH/dt = \Phi$ \quad(\textbf{PROVED}, Corollary~1)
\item $\Phi$ monotonicity: $d\Phi/dt \le 0$ \quad(\textbf{PROVED}, Theorem~3)
\item Entropy superadditivity: $H'_n(p \boxplus q) \ge H'_n(p) + H'_n(q)$
\end{enumerate}

Condition (3) is \textbf{scale-dependent} and does \textbf{not} hold universally:
log-discriminant superadditivity fails at all tested $n$ when root spread is large.
The surplus $H'_n(p \boxplus q) - H'_n(p) - H'_n(q)$ shifts by
$-\tfrac{n(n-1)}{2}\log c$ when both polynomials are scaled by $c$, making
violations common for $c > 1$. (At spread $\sigma = 1$: 3.9\% violations for $n=3$,
1.4\% for $n=4$; at spread $\sigma = 2$: 40\% and 35\% respectively.)

Therefore the ``de Bruijn $+$ log-disc superadditivity $\Rightarrow$ Stam'' route
is \textbf{not viable}. The remaining gap for $n \ge 4$ requires either:
\begin{itemize}\tightlist
\item A \textbf{scale-invariant entropy functional} (replacing $H'_n$) that IS
  superadditive under $\boxplus_n$, or
\item A \textbf{score decomposition / projection} argument --- the finite analog
  of Voiculescu's conjugate-variable Cauchy-Schwarz (1998), or
\item A \textbf{direct algebraic approach} (SOS certificate, Lorentzian polynomial
  cone restriction) that bypasses the de Bruijn route entirely.
\end{itemize}

\hypertarget{6-verification-for-small-cases}{%
\subsubsection{6. Verification for small cases}\label{6-verification-for-small-cases}}

\textbf{Degree 2 (proved --- equality):} \(p(x) = x^\mNumber{2} \mBridgeOperator{+} a_\mNumber{1} \ast x \mBridgeOperator{+} a_\mNumber{2}\).

\[\begin{aligned}
\Phi_\mNumber{2}(p) = \mNumber{2}/(a_\mNumber{1}^\mNumber{2} - \mNumber{4} \ast a_\mNumber{2}) \\
\mNumber{1}/\Phi_\mNumber{2}(p) = (a_\mNumber{1}^\mNumber{2} - \mNumber{4} \ast a_\mNumber{2})/\mNumber{2}
\end{aligned}\]

The \(\boxplus_\mNumber{2}\) formula gives \(c_{\mNumber{1}} = a_{\mNumber{1}}+b_{\mNumber{1}}, c_{\mNumber{2}} = a_{\mNumber{2}} \mBridgeOperator{+} a_{\mNumber{1}} \ast b_{\mNumber{1}}/\mNumber{2} \mBridgeOperator{+} b_{\mNumber{2}}.\) Then:

\[\mNumber{1}/\Phi_\mNumber{2}(p\boxplus q) = (c_\mNumber{1}^\mNumber{2} - \mNumber{4} \ast c_\mNumber{2})/\mNumber{2} = (a_\mNumber{1}^\mNumber{2} - \mNumber{4} \ast a_\mNumber{2})/\mNumber{2} + (b_\mNumber{1}^\mNumber{2} - \mNumber{4} \ast b_\mNumber{2})/\mNumber{2}\]

Surplus \(=\) 0 (symbolic verification). Equality for all degree-2 polynomials.

\textbf{Degree 3 (proved --- strict inequality):} See Section 5a. The proof uses:

\begin{itemize}
\tightlist
\item
  Translation invariance of \(\Phi\) + translation compatibility of \(\boxplus_\mNumber{3}\) to center
\item
  Identity \(\Phi_\mNumber{3}\) * disc \(= \mNumber{18} \ast a_{\mNumber{2}}^{\mNumber{2}}\) (for centered cubics)
\item
  Titu\textquotesingle s lemma (Cauchy-Schwarz) to bound the surplus
\end{itemize}

\hypertarget{7-summary-and-status}{%
\subsubsection{7. Summary and status}\label{7-summary-and-status}}

\textbf{The inequality is conjecturally true, with strong numerical evidence,
analytic proofs for \(n = \mNumber{2}\), 3, and significant structural results for all $n$.}

\begin{longtable}[]{@{}lll@{}}
\toprule\noalign{}
n & Status & Method \\
\midrule\noalign{}
\endhead
\bottomrule\noalign{}
\endlastfoot
2 & \textbf{PROVED} (equality) & \(\mNumber{1}/\Phi_\mNumber{2}\) is linear in disc; \(\boxplus_\mNumber{2}\) preserves exactly \\
3 & \textbf{PROVED} (strict ineq.) & \(\Phi_\mNumber{3} \ast \mOpName{disc} = \mNumber{18}a_\mNumber{2}^\mNumber{2}\) identity + Titu\textquotesingle s lemma \\
4 & Numerically verified & 0/3000 violations; cross-term \(a_\mNumber{2} \ast b_\mNumber{2}/\mNumber{6}\) essential \\
5 & Numerically verified & 0/2000 violations \\
\end{longtable}

\textbf{What is established analytically:}

\begin{enumerate}
\def\labelenumi{\arabic{enumi}.}
\tightlist
\item
  Finite free cumulants add under \(\boxplus_{n}\) (Arizmendi-Perales 2018)
\item
  \(n = \mNumber{2}\): equality (1/\(\Phi_\mNumber{2}\) linear in discriminant)
\item
  \(n = \mNumber{3}\): strict superadditivity (\(\Phi_\mNumber{3}\) * disc identity + Cauchy-Schwarz)
\item
  \(\boxplus_{n}\) commutes with translation (random matrix argument)
\item
  The superadditivity is SPECIFIC to \(\boxplus_{n}\) --- plain coefficient addition
  fails \textasciitilde40\% \((n = \mNumber{3}\) centered excluded: 0\% because \(\boxplus_{\mNumber{3}} =\) addition there)
\item
  \textbf{NEW:} Backward heat equation $\partial_t p_t = -\tfrac{1}{2}\partial_x^2 p_t$
  for $p_t = p \boxplus_n \mathrm{He}_t$ (Theorem~1, Section 5c)
\item
  \textbf{NEW:} Coulomb flow theorem: $d\gamma_k/dt = S_k(\gamma)$ (Theorem~2, Section 5c)
\item
  \textbf{NEW:} Finite de Bruijn identity: $d H'_n(p_t)/dt = \Phi_n(p_t)$ (Corollary~1, Section 5c)
\item
  \textbf{NEW:} $\Phi_n$ monotonicity: $d\Phi_n/dt \le 0$ via SOS identity (Theorem~3, Section 5c)
\item
  \textbf{NEW:} Hermite semigroup: $\mathrm{He}_s \boxplus_n \mathrm{He}_t = \mathrm{He}_{s+t}$
\end{enumerate}

\textbf{What remains open for \(n \ge \mNumber{4}\):}

The Stam inequality proof for $n \ge 4$ is now reduced to a single condition:
\emph{log-discriminant superadditivity}
$H'_n(p \boxplus_n q) \ge H'_n(p) + H'_n(q)$.
This is verified numerically (0 violations for $n \ge 4$) but unproved.
Combined with the established de Bruijn identity and $\Phi_n$ monotonicity,
this would complete the proof.

The earlier approaches also remain viable:

(a) A generalization of the \(\Phi\) * disc identity that accounts for higher
coefficients, or

(b) A direct random-matrix argument via the Haar orbit A+QBQ*, or

(c) A proof via finite free cumulant coordinates (convexity on the
real-rooted image of cumulant space).

\hypertarget{8-numerical-evidence}{%
\subsubsection{8. Numerical evidence}\label{8-numerical-evidence}}

Verification scripts:

\begin{itemize}
\tightlist
\item
  \texttt{scripts/verify-p4-inequality.py} (superadditivity + convexity)
\item
  \texttt{scripts/verify-p4-deeper.py} (coefficient addition + MSS structure)
\item
  \texttt{scripts/verify-p4-schur-majorization.py} (Schur/submodularity/paths)
\item
  \texttt{scripts/verify-p4-coefficient-route.py} (coefficient route, disc identity)
\item
  \texttt{scripts/verify-p4-n3-proof.py} (complete \(n = \mNumber{3}\) symbolic proof + \(n = \mNumber{4}\) exploration)
\item
  \texttt{scripts/prove-p4-coulomb-flow.py} (symbolic proof: backward heat eq.\ $+$ Coulomb flow)
\item
  \texttt{scripts/verify-p4-dphi-structure.py} (SOS formula for $d\Phi_n/dt$)
\item
  \texttt{scripts/verify-p4-root-velocity.py} (numerical root velocity $= S_k$ test)
\end{itemize}

\textbf{Superadditivity test} (2000 random real-rooted polynomial pairs per n):

\begin{longtable}[]{@{}lllll@{}}
\toprule\noalign{}
n & violations & min ratio & mean ratio & max ratio \\
\midrule\noalign{}
\endhead
\bottomrule\noalign{}
\endlastfoot
2 & 0/2000 & 1.000000 & 1.000000 & 1.000000 \\
3 & 0/2000 & 1.000082 & 3.031736 & 162.446 \\
4 & 0/2000 & 1.003866 & 9.549935 & 2268.065 \\
5 & 0/2000 & 1.032848 & 14.457013 & 2119.743 \\
\end{longtable}

Ratio \(= LHS/RHS = [\mNumber{1}/\Phi_{n}(p\boxplus q)] / [\mNumber{1}/\Phi_{n}(p) \mBridgeOperator{+} \mNumber{1}/\Phi_{n}(q)]\); \(ratio \ge \mNumber{1}\)
means inequality holds. Strict inequality for \(n \ge \mNumber{3}\).

\textbf{Convexity/concavity test} (midpoint test in coefficient space):

\begin{longtable}[]{@{}llll@{}}
\toprule\noalign{}
n & convex violations & concave violations & total tests \\
\midrule\noalign{}
\endhead
\bottomrule\noalign{}
\endlastfoot
3 & 757 (50.6\%) & 738 (49.4\%) & 1495 \\
4 & 648 (65.7\%) & 338 (34.3\%) & 986 \\
5 & 433 (72.9\%) & 161 (27.1\%) & 594 \\
\end{longtable}

\(\mNumber{1}/\Phi_{n}\) is NEITHER convex NOR concave in coefficient space.

\textbf{Plain coefficient addition test} (NOT ⊞\_n):

\begin{longtable}[]{@{}lll@{}}
\toprule\noalign{}
n & violations & total \\
\midrule\noalign{}
\endhead
\bottomrule\noalign{}
\endlastfoot
3 & 240 & 609 \\
4 & 159 & 373 \\
5 & 116 & 218 \\
\end{longtable}

Superadditivity FAILS under plain addition --- the inequality is specific
to the MSS bilinear structure of \(\boxplus_{n}\).

\hypertarget{key-references-from-futon6-corpus}{%
\subsection{Key References from futon6 corpus}\label{key-references-from-futon6-corpus}}

\begin{itemize}
\tightlist
\item
  PlanetMath: "monic polynomial" (Monic1)
\item
  PlanetMath: "discriminant" (Discriminant)
\item
  PlanetMath: "resultant" (Resultant, DerivationOfSylvestersMatrixForTheResultant)
\item
  PlanetMath: "logarithmic derivative" (LogarithmicDerivative)
\item
  PlanetMath: "partial fraction decomposition" (ALectureOnThePartialFractionDecompositionMethod)
\item
  PlanetMath: "cumulant generating function" (CumulantGeneratingFunction)
\end{itemize}

\hypertarget{external-references}{%
\subsection{External References}\label{external-references}}

\begin{itemize}
\item
  Marcus, Spielman, Srivastava (2015), ``Interlacing Families II: Mixed Characteristic
  Polynomials and the Kadison-Singer Problem,'' Annals of Math 182(1), 327--350.
  {[}Defines \(\boxplus_{n}\), proves real-rootedness preservation, random matrix interpretation{]}
\item
  Arizmendi, Perales (2018), ``Cumulants for finite free convolution,'' J. Combinatorial
  Theory Ser. A 155, 244--266. arXiv:1702.04761.
  {[}Defines finite free cumulants that linearize under \(\boxplus_{n}]{]}
\item
  Voiculescu (1998), ``The analogues of entropy and of Fisher's information measure
  in free probability theory, V: Noncommutative Hilbert transforms,'' Invent. Math.
  132, 189--227. {[}Free Stam inequality, conjugate variables, de Bruijn identity{]}
\end{itemize}
