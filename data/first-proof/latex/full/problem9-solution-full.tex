\section{Problem 9: Polynomial Detection of Rank-1 Scaling for Quadrifocal Tensors}

\subsection{Problem Statement}

Let n \textgreater= 5. Let A\^{}(1), ..., A\^{}(n) in R\^{}\{3x4\} be Zariski-generic matrices.
For alpha, beta, gamma, delta in {[}n{]}, construct Q\^{}(alpha beta gamma delta) in
R\^{}\{3x3x3x3\} with entry:

\begin{verbatim}
Q^(abgd)_{ijkl} = det[A^(a)(i,:); A^(b)(j,:); A^(g)(k,:); A^(d)(l,:)]
\end{verbatim}

(the 4x4 determinant of a matrix formed by stacking rows i, j, k, l from
cameras alpha, beta, gamma, delta respectively).

Does there exist a polynomial map F: R\^{}\{81*n\^{}4\} -\textgreater{} R\^{}N satisfying:

\begin{enumerate}
\def\labelenumi{\arabic{enumi}.}
\tightlist
\item
  F does not depend on A\^{}(1), ..., A\^{}(n)
\item
  The degrees of the coordinate functions of F do not depend on n
\item
  For lambda in R\^{}\{n x n x n x n\} with lambda\_\{abgd\} != 0 precisely when
  a,b,g,d are not all identical:
  F(lambda\_\{abgd\} Q\^{}(abgd) : a,b,g,d in {[}n{]}) = 0
  if and only if there exist u,v,w,x in (R*)\^{}n such that
  lambda\_\{abgd\} = u\_a v\_b w\_g x\_d for all non-identical a,b,g,d.
\end{enumerate}

\subsection{Answer}

\textbf{Yes.} Such a polynomial map F exists, with coordinate functions of degree 3.

\subsection{Solution}

\subsubsection{1. The quadrifocal tensor as a bilinear form}

The entry Q\^{}(abgd)\_\{ijkl\} = det{[}a\^{}(a)\_i; a\^{}(b)\_j; a\^{}(g)\_k; a\^{}(d)\_l{]} is
the evaluation of the volume form (the unique up-to-scale alternating 4-linear
form on R\^{}4) on four camera row vectors.

\textbf{Key observation (bilinear form reduction):} Fix two camera-row pairs
(gamma, k) and (delta, l), giving vectors c = a\^{}(gamma)\_k and d = a\^{}(delta)\_l
in R\^{}4. Then the map:

\begin{verbatim}
omega(p, q) = det[p; q; c; d]
\end{verbatim}

is an alternating bilinear form on R\^{}4. Since c wedge d is a simple 2-form,
the Hodge dual *(c wedge d) is also simple, so omega has \textbf{rank 2} as a
bilinear form.

Equivalently: the null space of omega is span\{c, d\} (2-dimensional), and
omega induces a non-degenerate alternating form on V/span\{c,d\} = R\^{}2.

\subsubsection{2. The rank-2 constraint and its 3x3 minor formulation}

Since omega has rank 2, for ANY choice of 3 vectors p\_1, p\_2, p\_3 and
3 vectors q\_1, q\_2, q\_3 in R\^{}4:

\begin{verbatim}
det | omega(p_1,q_1)  omega(p_1,q_2)  omega(p_1,q_3) |
    | omega(p_2,q_1)  omega(p_2,q_2)  omega(p_2,q_3) | = 0
    | omega(p_3,q_1)  omega(p_3,q_2)  omega(p_3,q_3) |
\end{verbatim}

(A rank-2 bilinear form has all 3x3 minors vanishing.)

In terms of the Q tensors: choosing p\_m = a\^{}(alpha\_m)\emph{\{i\_m\} and
q\_n = a\^{}(beta\_n)}\{j\_n\}, this becomes:

\begin{verbatim}
det [Q^(alpha_m, beta_n, gamma, delta)_{i_m, j_n, k, l}]_{3x3} = 0
\end{verbatim}

for any choice of (alpha\_1,i\_1), (alpha\_2,i\_2), (alpha\_3,i\_3) and
(beta\_1,j\_1), (beta\_2,j\_2), (beta\_3,j\_3) and fixed (gamma,k), (delta,l).

This is a \textbf{degree-3 polynomial} in the Q entries, independent of the cameras.

\subsubsection{3. Effect of rank-1 scaling on the 3x3 minor}

Now consider the scaled tensors T\^{}(abgd) = lambda\_\{abgd\} Q\^{}(abgd).
The 3x3 matrix becomes:

\begin{verbatim}
M_{mn} = lambda_{alpha_m, beta_n, gamma, delta} * Q^(alpha_m, beta_n, gamma, delta)_{i_m, j_n, k, l}
\end{verbatim}

This is the Hadamard (entrywise) product of two 3x3 matrices:

\begin{itemize}
\tightlist
\item
  Lambda\_\{mn\} = lambda\_\{alpha\_m, beta\_n, gamma, delta\} (depends on camera indices only)
\item
  Omega\_\{mn\} = Q\^{}(alpha\_m, beta\_n, gamma, delta)\_\{i\_m, j\_n, k, l\} (the bilinear form)
\end{itemize}

\textbf{If lambda is rank-1:} lambda\_\{abgd\} = u\_a v\_b w\_g x\_d, so
Lambda\_\{mn\} = u\_\{alpha\_m\} v\_\{beta\_n\} w\_gamma x\_delta. This factors as
Lambda = (u\_\{alpha\_m\})\emph{m * (v}\{beta\_n\})\_n\^{}T (scaled by the constant w\_gamma x\_delta).
So Lambda has matrix rank 1.

For a rank-1 matrix Lambda, the Hadamard product M = Lambda ∘ Omega equals
diag(u) * Omega * diag(v) (up to the scalar w\_gamma x\_delta). Since similar
transformations preserve rank: rank(M) = rank(Omega) = 2 \textless{} 3, so det(M) = 0.

Therefore: \textbf{rank-1 lambda implies all 3x3 minors of scaled T vanish.} ✓

\subsubsection{4. Converse: non-rank-1 lambda gives nonzero minors (for generic cameras)}

For the converse, we need: if lambda is NOT rank-1, then some 3x3 minor is
nonzero (for Zariski-generic cameras).

\textbf{The argument is algebraic-geometric, not via Hadamard rank bounds.} Define

\begin{verbatim}
P(A^{(1)}, ..., A^{(n)}) = det [T^(alpha_m, beta_n, gamma, delta)_{i_m, j_n, k, l}]_{3x3}
\end{verbatim}

for a specific choice of row/column/fixed indices. This is a polynomial in
the camera entries (with lambda fixed). We claim P is not the zero polynomial
when lambda is not rank-1 in its first two indices. Since a nonzero polynomial
is nonzero on a Zariski-dense open set, this establishes the converse for
generic cameras.

\textbf{Explicit witness (n = 5).} Choose 5 cameras A\^{}(i) as rows of generic
3 x 4 matrices with rational entries:

\begin{verbatim}
A^(1) = [[1,0,0,1],[0,1,0,1],[0,0,1,1]]
A^(2) = [[1,1,0,0],[0,1,1,0],[0,0,1,1]]
A^(3) = [[1,0,1,0],[1,1,0,0],[0,1,0,1]]
A^(4) = [[0,1,1,1],[1,0,1,1],[1,1,0,1]]
A^(5) = [[1,2,3,4],[4,3,2,1],[1,1,1,1]]
\end{verbatim}

Choose a non-rank-1 scaling: lambda\_\{abgd\} = 1 for all non-identical (a,b,g,d)
except lambda\_\{1,2,3,4\} = 2.

\textbf{Non-rank-1 verification.} If lambda were rank-1, then lambda\_\{abgd\} = u\_a v\_b w\_g x\_d.
From lambda\_\{1,2,3,4\} = 2 and lambda\_\{2,2,3,4\} = 1 we get u\_1/u\_2 = 2. But from
lambda\_\{1,1,3,4\} = 1 and lambda\_\{2,1,3,4\} = 1 we get u\_1/u\_2 = 1. Contradiction.

Fix gamma = 3, delta = 4, k = 1, l = 1. Choose row triples
(alpha\_m, i\_m) = (1,1), (2,1), (5,1) and column triples
(beta\_n, j\_n) = (2,1), (5,1), (1,2). The 3x3 Lambda matrix has entries
lambda\_\{alpha\_m, beta\_n, 3, 4\}; specifically Lambda\_\{11\} = lambda\_\{1,2,3,4\} = 2
with all other entries equal to 1. The Omega matrix (unscaled Q values) has
det(Omega) = 0 (confirming rank-2), but the Hadamard product M = Lambda ∘ Omega
satisfies det(M) = -24 != 0 (verified by direct computation of nine 4x4
determinants). This exhibits P as not the zero polynomial.

\textbf{Remark (Hadamard product interpretation).} The matrix M\_\{mn\} is the
Hadamard product Lambda ∘ Omega, where Lambda carries the scaling entries
and Omega carries the bilinear form values. The upper bound
rank(Lambda ∘ Omega) \textless= rank(Lambda) * rank(Omega) provides context but is
not used in the proof; the converse relies entirely on the polynomial
nonvanishing argument above.

\subsubsection{5. All matricizations from the same construction}

The construction in Section 2-4 tests the rank-1 condition on the (1,2)-
matricization of lambda (fixing modes 3,4). By symmetry, applying the same
construction with different pairs of "free" modes tests all matricizations:

\begin{itemize}
\tightlist
\item
  Fix modes (3,4), vary modes (1,2): test lambda\_\{(alpha,beta),(gamma,delta)\}
\item
  Fix modes (2,4), vary modes (1,3): test lambda\_\{(alpha,gamma),(beta,delta)\}
\item
  Fix modes (2,3), vary modes (1,4): test lambda\_\{(alpha,delta),(beta,gamma)\}
\end{itemize}

(The other fixings are redundant by symmetry of the rank-1 test.)

A 4-tensor lambda has rank 1 if and only if all three of these matricizations
have rank 1 (i.e., are outer products of vectors).

\textbf{Tensor factor compatibility lemma.} If all three matricizations have rank 1:

\begin{itemize}
\tightlist
\item
  Mode-(1,2) vs (3,4) rank 1 gives lambda\_\{abgd\} = f\_\{ab\} g\_\{gd\}.
\item
  Mode-(1,3) vs (2,4) rank 1 gives lambda\_\{abgd\} = h\_\{ag\} k\_\{bd\}.
  From the first: lambda\_\{a1,b1,g,d\} / lambda\_\{a2,b2,g,d\} = f\_\{a1,b1\} / f\_\{a2,b2\},
  independent of g, d. From the second:
  lambda\_\{a,b1,g1,d\} / lambda\_\{a,b2,g1,d\} = k\_\{b1,d\} / k\_\{b2,d\},
  independent of a, g1. Cross-referencing these separations forces
  f\_\{ab\} = u\_a v\_b and g\_\{gd\} = w\_g x\_d, giving
  lambda\_\{abgd\} = u\_a v\_b w\_g x\_d (rank 1). QED.
\end{itemize}

\subsubsection{6. Construction of F}

\textbf{Definition of F:} The coordinate functions of F are all 3x3 minors:

\begin{verbatim}
F_{choice} = det [T^(alpha_m, beta_n, gamma, delta)_{i_m, j_n, k, l}]_{m,n=1,2,3}
\end{verbatim}

taken over all choices of:

\begin{itemize}
\tightlist
\item
  Three "row" camera-row pairs (alpha\_m, i\_m) for m = 1,2,3
\item
  Three "column" camera-row pairs (beta\_n, j\_n) for n = 1,2,3
\item
  Fixed camera-row pairs (gamma, k) and (delta, l)
\end{itemize}

And the analogous minors for the other two matricization pairs (fixing
different pairs of modes and varying the others).

\textbf{Properties:}

\begin{enumerate}
\def\labelenumi{\arabic{enumi}.}
\item
  \textbf{Camera-independent:} Each F\_\{choice\} is a degree-3 polynomial in the
  T entries. The coefficient is ±1 (from the determinant expansion). No
  camera parameters appear.
\item
  \textbf{Degree independent of n:} Each coordinate function has degree exactly 3.
\item
  \textbf{Characterization:} F = 0 iff all three matricizations of lambda have
  rank 1, iff lambda is rank-1, for Zariski-generic cameras (by Sections 3-5).
\end{enumerate}

\textbf{Codimension count:} The number of coordinate functions N grows with n
(there are O(n\^{}8) choices for the 3x3 minors from the (1,2) vs (3,4)
matricization alone), but the degree stays fixed at 3.

\subsubsection{7. Geometric interpretation}

The Q tensors are the \textbf{quadrifocal tensors} from multiview geometry (the
4-view analog of the fundamental matrix and trifocal tensor). The rank-2
bilinear form structure is the classical rank constraint from projective
geometry: four collinear points in P\^{}3 impose a codimension constraint on
the stacked camera rows.

The rank-1 scaling lambda = u ⊗ v ⊗ w ⊗ x corresponds to the natural
gauge freedom in multiview geometry: rescaling each camera\textquotesingle s contribution
independently for each of the four "roles" (which row selection it
contributes). The polynomial map F detects when a putative rescaling is
consistent with this gauge group.

The Zariski-genericity requirement on the cameras ensures that the
quadrifocal variety is "non-degenerate" --- the Q tensors carry enough
information to distinguish rank-1 from higher-rank scalings.

\subsection{Key References from futon6 corpus}

\begin{itemize}
\tightlist
\item
  PlanetMath: "Segre map" (SegreMap) --- Segre embedding and rank-1 tensors
\item
  PlanetMath: "determinantal varieties" (SegreMap) --- varieties defined by minors
\item
  PlanetMath: "tensor rank" / "simple tensor" (SimpleTensor) --- rank-1 tensors
\item
  PlanetMath: "exterior algebra" --- alternating multilinear forms
\item
  PlanetMath: "Hadamard conjecture" (HadamardConjecture) --- Hadamard matrices
\end{itemize}
