\hypertarget{problem-9-polynomial-detection-of-rank-1-scaling-for-quadrifocal-tensors}{%
\section{Problem 9: Polynomial Detection of Rank-1 Scaling for Quadrifocal Tensors}\label{problem-9-polynomial-detection-of-rank-1-scaling-for-quadrifocal-tensors}}

\hypertarget{problem-statement}{%
\subsection{Problem Statement}\label{problem-statement}}

Let n \textgreater= 5. Let \(A^{1},\) ..., \(A^{n} \in R^{3 \times 4}\) be Zariski-generic matrices.
For \(\alpha\), \(\beta\), \(\gamma\), \(\delta\) in {[}n{]}, construct Q\^{}(alpha \(\beta\) \(\gamma\) delta) in
\(R^{3 \times 3x3 \times 3}\) with entry:

\[Q^{abgd}_{ijkl} = det[A^{a}(i,:); A^{b}(j,:); A^{g}(k,:); A^{d}(l,:)]\]

(the 4x4 determinant of a matrix formed by stacking rows i, j, k, l from
cameras \(\alpha\), \(\beta\), \(\gamma\), \(\delta\) respectively).

Does there exist a polynomial map F: \(R^{81 \ast n^4}\) -\textgreater{} \(R^{N}\) satisfying:

\begin{enumerate}
\def\labelenumi{\arabic{enumi}.}
\tightlist
\item
  F does not depend on \(A^{1},\) ..., \(A^{n}\)
\item
  The degrees of the coordinate functions of F do not depend on n
\item
  For \(\lambda \in R^{n \times n \times n \times n}\) with \(\lambda_{abgd}\) != 0 precisely when
  \(a,b,g,d\) are not all identical:
  F(\(\lambda_{abgd}\) \(Q^{abgd}\) : \(a,b,g,d\) in {[}n{]}) = 0
  if and only if there exist \(u,v,w,x \in (R \ast )^{n}\) such that
  \(\lambda_{abgd} = u_{a}\) \(v_{b}\) \(w_{g}\) \(x_{d}\) for all non-identical \(a,b,g,d\).
\end{enumerate}

\hypertarget{answer}{%
\subsection{Answer}\label{answer}}

\textbf{Yes.} Such a polynomial map F exists, with coordinate functions of degree 3.

\hypertarget{solution}{%
\subsection{Solution}\label{solution}}

\hypertarget{1-the-quadrifocal-tensor-as-a-bilinear-form}{%
\subsubsection{1. The quadrifocal tensor as a bilinear form}\label{1-the-quadrifocal-tensor-as-a-bilinear-form}}

The entry \(Q^{abgd}_{ijkl}\) = det{[}\(a^{a}_{i}\); \(a^{b}_{j}\); \(a^{g}_{k}\); \(a^{d}_{l}\){]} is
the evaluation of the volume form (the unique up-to-scale alternating 4-linear
form on \(R^4\)) on four camera row vectors.

\textbf{Key observation (bilinear form reduction):} Fix two camera-row pairs
(gamma, k) and (delta, l), giving vectors \(c = a^{\gamma}_{k}\) and \(d = a^{\delta}_{l} \in R^4.\) Then the map:

\[\omega(p, q) = det[p; q; c; d]\]

is an alternating bilinear form on \(R^4.\) Since c wedge d is a simple 2-form,
the Hodge dual *(c wedge d) is also simple, so \(\omega\) has \textbf{rank 2} as a
bilinear form.

Equivalently: the null space of \(\omega\) is span\{c, d\} (2-dimensional), and
\(\omega\) induces a non-degenerate alternating form on V/span\{c,d\} = \(R^2.\)

\hypertarget{2-the-rank-2-constraint-and-its-3x3-minor-formulation}{%
\subsubsection{2. The rank-2 constraint and its 3x3 minor formulation}\label{2-the-rank-2-constraint-and-its-3x3-minor-formulation}}

Since \(\omega\) has rank 2, for ANY choice of 3 vectors \(p_1,\) \(p_2,\) \(p_3\) and
3 vectors \(q_1,\) \(q_2,\) \(q_3 \in R^4:\)

\[\begin{aligned}
det | \omega(p_1,q_1) \omega(p_1,q_2) \omega(p_1,q_3) | \\
| \omega(p_2,q_1) \omega(p_2,q_2) \omega(p_2,q_3) | = 0 \\
| \omega(p_3,q_1) \omega(p_3,q_2) \omega(p_3,q_3) |
\end{aligned}\]

(A rank-2 bilinear form has all 3x3 minors vanishing.)

In terms of the Q tensors: choosing \(p_{m} = a^{\alpha_{m}}\)\emph{\({i_{m}}\) and
\(q_{n} = a^{\beta_{n}}\)}\({j_{n}},\) this becomes:

\[det [Q^{\alpha_{m}, \beta_{n}, \gamma, \delta}_{i_{m}, j_{n}, k, l}]_{3 \times 3} = 0\]

for any choice of \((\alpha_1,i_1),\) \((\alpha_2,i_2),\) \((\alpha_3,i_3)\) and
\((\beta_1,j_1),\) \((\beta_2,j_2),\) \((\beta_3,j_3)\) and fixed (gamma,k), (delta,l).

This is a \textbf{degree-3 polynomial} in the Q entries, independent of the cameras.

\hypertarget{3-effect-of-rank-1-scaling-on-the-3x3-minor}{%
\subsubsection{3. Effect of rank-1 scaling on the 3x3 minor}\label{3-effect-of-rank-1-scaling-on-the-3x3-minor}}

Now consider the scaled tensors \(T^{abgd} = \lambda_{abgd}\) \(Q^{abgd}.\)
The 3x3 matrix becomes:

\[M_{mn} = \lambda_{\alpha_{m}, \beta_{n}, \gamma, \delta} \ast Q^{\alpha_{m}, \beta_{n}, \gamma, \delta}_{i_{m}, j_{n}, k, l}\]

This is the Hadamard (entrywise) product of two 3x3 matrices:

\begin{itemize}
\tightlist
\item
  \(\Lambda_{mn} = \lambda_{\alpha_{m}, \beta_{n}, \gamma, \delta}\) (depends on camera indices only)
\item
  \(\Omega_{mn}\) = Q\^{}(\(\alpha_{m}\), \(\beta_{n},\) \(\gamma\), delta)\_\{\(i_{m}\), \(j_{n},\) k, l\} (the bilinear form)
\end{itemize}

\textbf{If \(\lambda\) is rank-1:} \(\lambda_{abgd} = u_{a}\) \(v_{b}\) \(w_{g}\) \(x_{d},\) so
\(\Lambda_{mn} = u_{\alpha_{m}}\) \(v_{\beta_{n}}\) \(w_{\gamma}\) \(x_{\delta}.\) This factors as
\(\Lambda = (u_{\alpha_{m}})\)\emph{m * (v}\{\(\beta_{n}\)\})\_\(n^{T}\) (scaled by the constant \(w_{\gamma}\) \(x_{\delta}\)).
So \(\Lambda\) has matrix rank 1.

For a rank-1 matrix \(\Lambda\), the Hadamard product \(M = \Lambda\) ∘ \(\Omega\) equals
\(\operatorname{diag}(u) \ast \Omega \ast \operatorname{diag}(v)\) (up to the scalar \(w_{\gamma}\) \(x_{\delta}\)). Since similar
transformations preserve rank: rank(M) = rank(Omega) = 2 \textless{} 3, so det(M) = 0.

Therefore: \textbf{rank-1 \(\lambda\) implies all 3x3 minors of scaled T vanish.} ✓

\hypertarget{4-converse-non-rank-1-lambda-gives-nonzero-minors-for-generic-cameras}{%
\subsubsection{\texorpdfstring{4. Converse: non-rank-1 \(\lambda\) gives nonzero minors (for generic cameras)}{4. Converse: non-rank-1 \textbackslash lambda gives nonzero minors (for generic cameras)}}\label{4-converse-non-rank-1-lambda-gives-nonzero-minors-for-generic-cameras}}

For the converse, we need: if \(\lambda\) is NOT rank-1, then some 3x3 minor is
nonzero (for Zariski-generic cameras).

\textbf{The argument is algebraic-geometric, not via Hadamard rank bounds.} Define

\[P(A^{(1)}, ..., A^{(n)}) = det [T^{\alpha_{m}, \beta_{n}, \gamma, \delta}_{i_{m}, j_{n}, k, l}]_{3 \times 3}\]

for a specific choice of row/column/fixed indices. This is a polynomial in
the camera entries (with \(\lambda\) fixed). We claim P is not the zero polynomial
when \(\lambda\) is not rank-1 in its first two indices. Since a nonzero polynomial
is nonzero on a Zariski-dense open set, this establishes the converse for
generic cameras.

\textbf{Explicit witness (n = 5).} Choose 5 cameras \(A^{i}\) as rows of generic
\(3 \times 4\) matrices with rational entries:

\[\begin{aligned}
A^{1} = [[1,0,0,1],[0,1,0,1],[0,0,1,1]] \\
A^{2} = [[1,1,0,0],[0,1,1,0],[0,0,1,1]] \\
A^{3} = [[1,0,1,0],[1,1,0,0],[0,1,0,1]] \\
A^{4} = [[0,1,1,1],[1,0,1,1],[1,1,0,1]] \\
A^{5} = [[1,2,3,4],[4,3,2,1],[1,1,1,1]]
\end{aligned}\]

Choose a non-rank-1 scaling: \(\lambda_{abgd}\) = 1 for all non-identical (\(a,b,g,d\))
except \(\lambda_{1,2,3,4}\) = 2.

\textbf{Non-rank-1 verification.} If \(\lambda\) were rank-1, then \(\lambda_{abgd} = u_{a}\) \(v_{b}\) \(w_{g}\) \(x_{d}.\)
From \(\lambda_{1,2,3,4}\) = 2 and \(\lambda_{2,2,3,4}\) = 1 we get \(u_1/u_2\) = 2. But from
\(\lambda_{1,1,3,4}\) = 1 and \(\lambda_{2,1,3,4}\) = 1 we get \(u_1/u_2\) = 1. Contradiction.

Fix \(\gamma\) = 3, \(\delta\) = 4, k = 1, l = 1. Choose row triples
(\(\alpha_{m}\), \(i_{m}\)) = (1,1), (2,1), (5,1) and column triples
(\(\beta_{n}\), \(j_{n}\)) = (2,1), (5,1), (1,2). The 3x3 \(\Lambda\) matrix has entries
\(\lambda_{\alpha_{m}, \beta_{n}, 3, 4}\); specifically \(\Lambda_{11} = \lambda_{1,2,3,4}\) = 2
with all other entries equal to 1. The \(\Omega\) matrix (unscaled Q values) has
det(Omega) = 0 (confirming rank-2), but the Hadamard product \(M = \Lambda\) ∘ \(\Omega\)
satisfies det(M) = -24 != 0 (verified by direct computation of nine 4x4
determinants). This exhibits P as not the zero polynomial.

\textbf{Remark (Hadamard product interpretation).} The matrix \(M_{mn}\) is the
Hadamard product \(\Lambda\) ∘ \(\Omega\), where \(\Lambda\) carries the scaling entries
and \(\Omega\) carries the bilinear form values. The upper bound
rank(Lambda ∘ Omega) \textless= rank(Lambda) * rank(Omega) provides context but is
not used in the proof; the converse relies entirely on the polynomial
nonvanishing argument above.

\hypertarget{5-all-matricizations-from-the-same-construction}{%
\subsubsection{5. All matricizations from the same construction}\label{5-all-matricizations-from-the-same-construction}}

The construction in Section 2-4 tests the rank-1 condition on the (1,2)-
matricization of \(\lambda\) (fixing modes 3,4). By symmetry, applying the same
construction with different pairs of "free" modes tests all matricizations:

\begin{itemize}
\tightlist
\item
  Fix modes (3,4), vary modes (1,2): test \(\lambda_{(\alpha,\beta),(\gamma,\delta)}\)
\item
  Fix modes (2,4), vary modes (1,3): test \(\lambda_{(\alpha,\gamma),(\beta,\delta)}\)
\item
  Fix modes (2,3), vary modes (1,4): test \(\lambda_{(\alpha,\delta),(\beta,\gamma)}\)
\end{itemize}

(The other fixings are redundant by symmetry of the rank-1 test.)

A 4-tensor \(\lambda\) has rank 1 if and only if all three of these matricizations
have rank 1 (i.e., are outer products of vectors).

\textbf{Tensor factor compatibility lemma.} If all three matricizations have rank 1:

\begin{itemize}
\tightlist
\item
  Mode-(1,2) vs (3,4) rank 1 gives \(\lambda_{abgd} = f_{ab}\) \(g_{gd}.\)
\item
  Mode-(1,3) vs (2,4) rank 1 gives \(\lambda_{abgd} = h_{ag}\) \(k_{bd}.\)
  From the first: \(\lambda_{a1,b1,g,d}\) / \(\lambda_{a2,b2,g,d} = f_{a1,b1}\) / \(f_{a2,b2},\)
  independent of g, d. From the second:
  \(\lambda_{a,b1,g1,d}\) / \(\lambda_{a,b2,g1,d} = k_{b1,d}\) / \(k_{b2,d},\)
  independent of a, g1. Cross-referencing these separations forces
  \(f_{ab} = u_{a}\) \(v_{b}\) and \(g_{gd} = w_{g}\) \(x_{d},\) giving
  \(\lambda_{abgd} = u_{a}\) \(v_{b}\) \(w_{g}\) \(x_{d}\) (rank 1). QED.
\end{itemize}

\hypertarget{6-construction-of-f}{%
\subsubsection{6. Construction of F}\label{6-construction-of-f}}

\textbf{Definition of F:} The coordinate functions of F are all 3x3 minors:

\[F_{choice} = det [T^{\alpha_{m}, \beta_{n}, \gamma, \delta}_{i_{m}, j_{n}, k, l}]_{m,n=1,2,3}\]

taken over all choices of:

\begin{itemize}
\tightlist
\item
  Three "row" camera-row pairs (\(\alpha_{m}\), \(i_{m}\)) for m = 1,2,3
\item
  Three "column" camera-row pairs (\(\beta_{n}\), \(j_{n}\)) for n = 1,2,3
\item
  Fixed camera-row pairs (gamma, k) and (delta, l)
\end{itemize}

And the analogous minors for the other two matricization pairs (fixing
different pairs of modes and varying the others).

\textbf{Properties:}

\begin{enumerate}
\def\labelenumi{\arabic{enumi}.}
\item
  \textbf{Camera-independent:} Each \(F_{choice}\) is a degree-3 polynomial in the
  T entries. The coefficient is ±1 (from the determinant expansion). No
  camera parameters appear.
\item
  \textbf{Degree independent of n:} Each coordinate function has degree exactly 3.
\item
  \textbf{Characterization:} F = 0 iff all three matricizations of \(\lambda\) have
  rank 1, iff \(\lambda\) is rank-1, for Zariski-generic cameras (by Sections 3-5).
\end{enumerate}

\textbf{Codimension count:} The number of coordinate functions N grows with n
(there are \(O(n^8)\) choices for the 3x3 minors from the (1,2) vs (3,4)
matricization alone), but the degree stays fixed at 3.

\hypertarget{7-geometric-interpretation}{%
\subsubsection{7. Geometric interpretation}\label{7-geometric-interpretation}}

The Q tensors are the \textbf{quadrifocal tensors} from multiview geometry (the
4-view analog of the fundamental matrix and trifocal tensor). The rank-2
bilinear form structure is the classical rank constraint from projective
geometry: four collinear points in \(P^3\) impose a codimension constraint on
the stacked camera rows.

The rank-1 scaling \(\lambda = u\) ⊗ v ⊗ w ⊗ x corresponds to the natural
gauge freedom in multiview geometry: rescaling each camera\textquotesingle s contribution
independently for each of the four "roles" (which row selection it
contributes). The polynomial map F detects when a putative rescaling is
consistent with this gauge group.

The Zariski-genericity requirement on the cameras ensures that the
quadrifocal variety is "non-degenerate" --- the Q tensors carry enough
information to distinguish rank-1 from higher-rank scalings.

\hypertarget{key-references-from-futon6-corpus}{%
\subsection{Key References from futon6 corpus}\label{key-references-from-futon6-corpus}}

\begin{itemize}
\tightlist
\item
  PlanetMath: "Segre map" (SegreMap) --- Segre embedding and rank-1 tensors
\item
  PlanetMath: "determinantal varieties" (SegreMap) --- varieties defined by minors
\item
  PlanetMath: "tensor rank" / "simple tensor" (SimpleTensor) --- rank-1 tensors
\item
  PlanetMath: "exterior algebra" --- alternating multilinear forms
\item
  PlanetMath: "Hadamard conjecture" (HadamardConjecture) --- Hadamard matrices
\end{itemize}
