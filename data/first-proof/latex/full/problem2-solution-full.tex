\hypertarget{problem-2-universal-test-vector-for-rankin-selberg-integrals}{%
\section{Problem 2: Universal Test Vector for Rankin-Selberg Integrals}\label{problem-2-universal-test-vector-for-rankin-selberg-integrals}}

\hypertarget{problem-statement}{%
\subsection{Problem Statement}\label{problem-statement}}

Let F be a non-archimedean local field with ring of integers o. Let \(N_{r}\)
denote the subgroup of \(\mathup{GL}_{r}(F)\) consisting of upper-triangular unipotent
elements. Let \(\psi\): F -\textgreater{} \(\mathbb{C}^{x}\) be a nontrivial additive character of conductor o,
identified with a generic character of \(N_{r}\).

Let \(\Pi\) be a generic irreducible admissible representation of \(\mathup{GL}_{n+\mNumber{1}}(F)\),
realized in its \(\psi^{-\mNumber{1}}\)-Whittaker model \(W(\Pi, \psi^{-\mNumber{1}})\). Must there exist
\(W \in W(\Pi, \psi^{-\mNumber{1}})\) with the following property?

For any generic irreducible admissible representation \(\pi\) of \(\mathup{GL}_{n}(F)\) in its
\(\psi\)-Whittaker model \(W(\pi, \psi)\), let q be the conductor ideal of \(\pi, Q \in F^{\times}\) a generator of \(q^{-\mNumber{1}}\), and \(u_{Q}\) := \(I_{n+\mNumber{1}} \mBridgeOperator{+} Q E_{n,n+\mNumber{1}}\). Then for
some \(V \in W(\pi, \psi)\), the local Rankin-Selberg \(\Integral\)

\[\int_{N_{n} \backslash \mathup{GL}_{n}(F)} W(\operatorname{diag}(g,\mNumber{1}) u_{Q}) V(g) |det g|^{s-\mNumber{1}/\mNumber{2}} dg\]

is finite and nonzero for all \(s \in C\).

\hypertarget{answer}{%
\subsection{Answer}\label{answer}}

\textbf{Yes.} The new vector (essential Whittaker function) of \(\Pi\) serves as a
universal test vector, with the \(u_{Q}\) twist compensating for the conductor of \(\pi\).

\textbf{Confidence: Medium-high.} The argument combines standard Rankin-Selberg
theory (JPSS) with the Bernstein-Zelevinsky theory of the Kirillov model.
The key nondegeneracy claim (Section 3a) uses \(\mathup{GL}_{n}\)-equivariance of the
zeta-\(\Integral\) pairing together with irreducibility of the Kirillov model
(any nonzero function generates the full module under \(\mathup{GL}_{n}\)-translates)
and the PID structure of the fractional ideal ring.
The "nonzero for all s" condition reduces to explicit Laurent polynomial
algebra (Section 2).

\hypertarget{solution}{%
\subsection{Solution}\label{solution}}

\hypertarget{1-rankin-selberg-theory-background}{%
\subsubsection{1. Rankin-Selberg theory background}\label{1-rankin-selberg-theory-background}}

The local Rankin-Selberg \(\Integral I(s, W, V)\) for \(\mathup{GL}_{n+\mNumber{1}} \times \mathup{GL}_{n}\) is:

\[I(s, W, V) = \int_{N_{n} \backslash \mathup{GL}_{n}(F)} W(\operatorname{diag}(g,\mNumber{1})) V(g) |det g|^{s-\mNumber{1}/\mNumber{2}} dg\]

This converges for Re(s) \textgreater\textgreater{} 0 and extends to a rational function of \(q_{F}^{-s}\)
(where \(q_{F}\) = \textbar o/p\textbar). The set of all such integrals as (W, V) vary generates
a fractional ideal of \(\mathbb{C}[q_{F}^{s}, q_{F}^{-s}]\), whose generator is the local
L-factor \(L(s, \Pi \times \pi)\).

The problem modifies this by inserting \(u_{Q} = I_{n+\mNumber{1}} \mBridgeOperator{+} Q E_{n,n+\mNumber{1}}\) into the
argument of W, giving the "twisted" Rankin-Selberg \(\Integral\).

\hypertarget{2-the-condition-finite-and-nonzero-for-all-s}{%
\subsubsection{2. The condition "finite and nonzero for all s"}\label{2-the-condition-finite-and-nonzero-for-all-s}}

For a rational function \(f(q_{F}^{-s})\) to be "finite and nonzero for all \(s \in C\),"
it must have no poles and no zeros when viewed as a function of
\(X = q_{F}^{-s} \in \mathbb{C}^{x}\). Such a rational function must be \(c \ast X^{k} = c \ast q_{F}^{-ks}\)
for some nonzero c and integer k.

So the condition requires: there exists \(V\) such that \(I(s, W, V) = c \ast q_{F}^{-ks}\)
for some nonzero constant c and integer k.

By Section 3a below, the integrals over \(V\) (for fixed \(W = W_\mNumber{0}\)) generate the
full fractional ideal \(I = L(s, \Pi \times \pi)\,\ast\,\mathbb{C}[q_{F}^{s}, q_{F}^{-s}]\).

\textbf{Explicit algebra:} \(I\) is a free rank-1 module over the ring
\(R = \mathbb{C}[q_{F}^{s}, q_{F}^{-s}]\), generated by \(L(s, \Pi \times \pi)\). We seek an element
of \(I\) of the form \(c \ast q_{F}^{-ks}\) (a monomial --- no poles or zeros). Write:

\[c \ast q_{F}^{-ks} = L(s, \Pi \times \pi) \ast P(q_{F}^{-s})\]

where \(P = c \ast q_{F}^{-ks} \ast L(s, \Pi \times \pi)^{-\mNumber{1}}\). Since \(L(s, \Pi \times \pi)^{-\mNumber{1}}\)
is a polynomial in \(q_{F}^{-s}\) (the local L-factor for \(\mathup{GL}_{n+\mNumber{1}} \times \mathup{GL}_{n}\) is a
product of terms (1 - \(\alpha_{i} q_{F}^{-s}\))\^{}\{-1\}, so its reciprocal is a
polynomial), P is a Laurent polynomial in \(q_{F}^{-s}\) --- hence \(P \in R\).
Therefore \(c \ast q_{F}^{-ks} \in I\), and by the spanning property, some \(V \in W(\pi, \psi)\) realizes this element as its Rankin-Selberg \(\Integral\) against \(W_\mNumber{0}\).

\hypertarget{3-the-u_q-twist-and-the-kirillov-model}{%
\subsubsection{\texorpdfstring{3. The \(u_{Q}\) twist and the Kirillov model}{3. The u\_\{Q\} twist and the Kirillov model}}\label{3-the-u_q-twist-and-the-kirillov-model}}

The key role of \(u_{Q}\): right-translating W by \(u_{Q}\) gives a new Whittaker function
\(R(u_{Q})W \in W(\Pi, \psi^{-\mNumber{1}})\). The restriction to the mirabolic subgroup
\(P_{n+\mNumber{1}}\) gives the Kirillov model, and the function:

\[\phi_{Q}(g) := W(\operatorname{diag}(g,\mNumber{1}) u_{Q}) = (R(u_{Q})W)(\operatorname{diag}(g,\mNumber{1}))\]

lies in the Kirillov model of \(\Pi\) restricted to \(\mathup{GL}_{n}\).

We need \(\phi_{Q}\) to be nonzero as a function on \(\mathup{GL}_{n}\). This requires two
separate facts:

\textbf{Fact 1: \(R(u_{Q})W\) is nonzero in \(W(\Pi, \psi^{-\mNumber{1}})\).} Right translation by
\(u_{Q}\) preserves the Whittaker model (since \(u_{Q}\) lies in the unipotent radical
of the standard maximal parabolic, which normalizes \(\psi\)). The map
W \(\to R(u_{Q})W\) is a linear automorphism of \(W(\Pi, \psi^{-\mNumber{1}})\) (with inverse
R(\(u_{Q}^{-\mNumber{1}}\))), so \(R(u_{Q})W \ne\) 0 whenever W \(\ne\) 0.

\textbf{Fact 2: \(\phi_{Q}(g) = W_\mNumber{0}(\operatorname{diag}(g,\mNumber{1})u_{Q})\) is nonzero for \(W_\mNumber{0}\) = new vector.}
This is verified by direct evaluation at \(g = I_{n}\):

\[\phi_{Q}(I_{n}) = W_\mNumber{0}(\operatorname{diag}(I_{n}, \mNumber{1}) · u_{Q}) = W_\mNumber{0}(u_{Q})\]

Now \(u_{Q} = I_{n+\mNumber{1}} \mBridgeOperator{+} Q E_{n,n+\mNumber{1}}\) is an element of the upper-triangular
unipotent subgroup \(N_{n+\mNumber{1}}\) (its only off-diagonal entry is Q in position
(n, n+1)). By the Whittaker-model transformation rule:

\[W_\mNumber{0}(u_{Q}) = \psi^{-\mNumber{1}}(u_{Q}) · W_\mNumber{0}(I_{n+\mNumber{1}})\]

The generic character \(\psi\) of \(N_{n+\mNumber{1}}\) evaluates on superdiagonal entries:
\(\psi(u_{Q}) = \psi(Q)\) (the only superdiagonal entry of \(u_{Q}\) is Q in position
(n, n+1)). With the standard normalization \(W_\mNumber{0}(I_{n+\mNumber{1}}) = \mNumber{1}\):

\[\phi_{Q}(I_{n}) = \psi^{-\mNumber{1}}(Q) · \mNumber{1} = \psi^{-\mNumber{1}}(Q)\]

Since \(\psi\) is a nontrivial character of F and Q \(\ne\) 0 (Q generates \(q^{-\mNumber{1}}), \psi^{-\mNumber{1}}(Q)\) is a nonzero complex number (it lies on the unit circle).
Therefore \(\phi_{Q}(I_{n}) \ne\) 0, so \(\phi_{Q}\) is nonzero as a function on \(\mathup{GL}_{n}\). ∎

(Note: this argument is specific to \(W_\mNumber{0}\) = new vector with \(W_\mNumber{0}(I) = \mNumber{1}\).
For the universality claim we only need \(\phi_{Q}\) nonzero for this particular
\(W_\mNumber{0}\), so the direct computation suffices.)

\hypertarget{3a-nondegeneracy-for-fixed-w-closing-the-universality-gap}{%
\subsubsection{3a. Nondegeneracy for fixed W (closing the universality gap)}\label{3a-nondegeneracy-for-fixed-w-closing-the-universality-gap}}

The standard JPSS nondegeneracy result (Section 2.7) states that as BOTH W
and V vary, the integrals \(I(s, W, V)\) generate the full fractional ideal
\(L(s, \Pi \times \pi) \ast \mathbb{C}[q_{F}^{s}, q_{F}^{-s}]\). For the universality claim, we need
this to hold for FIXED \(W = W_\mNumber{0}\). This requires the following:

\textbf{Lemma (fixed-W spanning).} Let \(\Pi\) be a generic irreducible admissible
representation of \(\mathup{GL}_{n+\mNumber{1}}(F)\), and let \(\phi\) be any nonzero function in the
Kirillov model of \(\Pi\) restricted to \(\mathup{GL}_{n}\). Then the set of integrals

\[{ \int_{N_{n} \backslash \mathup{GL}_{n}(F)} \phi(g) V(g) |det g|^{s-\mNumber{1}/\mNumber{2}} dg : V \in W(\pi, \psi) }\]

generates a nonzero fractional ideal of \(\mathbb{C}[q_{F}^{s}, q_{F}^{-s}]\) (contained in
the full ideal \(L(s, \Pi \times \pi) \ast \mathbb{C}[q_{F}^{s}, q_{F}^{-s}]\)).

\emph{Proof sketch.} The Kirillov model of \(\Pi|_{\mathup{GL}_{n}}\) contains all compactly
supported locally constant functions on \(N_{n} \backslash \mathup{GL}_{n}(F)\) (Bernstein-Zelevinsky
1976, Theorem 5.21 and Corollary 5.22). In particular, \(\phi\) can be
approximated by compactly supported functions. The Rankin-Selberg \(\Integral\)
against a fixed nonzero \(\phi\) defines a nonzero linear functional on \(W(\pi, \psi)\)
(as a function of V), since \(\pi\) is generic and hence \(W(\pi, \psi)\) separates
points on \(N_{n} \backslash \mathup{GL}_{n}(F)\). The resulting set of integrals is a nonzero
\(\mathbb{C}[q_{F}^{s}, q_{F}^{-s}]\)-submodule of \(L(s, \Pi \times \pi) \ast \mathbb{C}[q_{F}^{s}, q_{F}^{-s}]\).

Since \(\mathbb{C}[q_{F}^{s}, q_{F}^{-s}]\) is a PID (a Laurent polynomial ring in one variable
\(q_{F}^{-s}\)), every nonzero submodule of a free rank-1 module is itself free of
rank 1. Let \(L_{\phi}(s) \ast \mathbb{C}[q_{F}^{s}, q_{F}^{-s}]\) be this submodule, where \(L_{\phi}\)
divides \(L(s, \Pi \times \pi)\).

To show \(L_{\phi} = L(s, \Pi \times \pi)\) (i.e., fixed \(\phi\) generates the FULL ideal):

The JPSS theory (1983, Section 2.7) shows that the full ideal is generated
by letting both W and V vary. Varying W (with restriction to \(\mathup{GL}_{n}\))
corresponds to varying \(\phi\) in the full Kirillov model \(K(\Pi)|_{\mathup{GL}_{n}}\). The
ideal generated by ALL \(\phi\) is \(L(s, \Pi \times \pi) \ast R\). We must show that a
SINGLE nonzero \(\phi\) already suffices.

Consider the map \(\Phi: K(\Pi)|_{\mathup{GL}_{n}} \to (\text{fractional ideals of } R)\) defined by
\(\Phi(\phi) = \{ I(s, \phi, V) : V \in W(\pi, \psi) \} \cdot R\). By the JPSS theory,
\(\bigcup_{\phi} \Phi(\phi)\) generates \(L(s, \Pi \times \pi)\) · R. Since R is a PID, \(\Phi(\phi) = L_{\phi}\) · R for some \(L_{\phi}\) dividing \(L(s, \Pi \times \pi)\).

\textbf{Key step.} \(\Phi\) is \(\mathup{GL}_{n}\)-equivariant in the following sense: for \(g_\mNumber{0} \in \mathup{GL}_{n}\), the substitution g \(\to\) g · \(g_\mNumber{0}\) in the \(\Integral\) gives
I(s, \(R(g_\mNumber{0})\phi\), V) = \(|\det g_\mNumber{0}|^{\mNumber{1}/\mNumber{2}-s}\) · I(s, \(\phi\), R\textquotesingle(\(g_\mNumber{0}\))V) where R\textquotesingle{}
denotes the contragredient action on \(W(\pi, \psi)\). Since \(R'(g_\mNumber{0})\) is an
automorphism of \(W(\pi, \psi)\), the set of integrals \{ I(s, \(R(g_\mNumber{0})\phi\), V) :
\(V \in W(\pi, \psi)\) \} equals \{ \(|\det g_\mNumber{0}|^{\mNumber{1}/\mNumber{2}-s}\) · \(I(s, \\phi, V)\) : V \} =
\(|\det g_\mNumber{0}|^{\mNumber{1}/\mNumber{2}-s}\) · \(\Phi(\phi)\). But \(|\det g_\mNumber{0}|^{\mNumber{1}/\mNumber{2}-s} = q_{F}^{-k·s}\) ·
(unit in R) is a unit in the localization, so \(\Phi(R(g_\mNumber{0})\,\phi)\) and \(\Phi(\phi)\)
generate the same fractional ideal. That is: \(L_{R(g_\mNumber{0})\phi} = L_{\phi}\)
(up to units in R) for all \(g_\mNumber{0} \in \mathup{GL}_{n}\).

Now, the Kirillov model \(K(\Pi)|_{\mathup{GL}_{n}}\) is irreducible as a \(\mathup{GL}_{n}-\)
representation (Bernstein-Zelevinsky 1976, Theorem 5.21). Therefore
for any nonzero \(\phi\), the \(\mathup{GL}_{n}\)-translates \{ \(R(g_\mNumber{0})\phi\) : \(g_\mNumber{0} \in \mathup{GL}_{n}\) \}
span all of \(K(\Pi)|_{\mathup{GL}_{n}}\). We use this in two directions:

\textbf{Direction 1 (\(L_{\phi}\) divides \(L(s, \Pi \times \pi)\)):} \(\Phi(\phi)\) ⊆ \(\Phi_{full}\)
(the full ideal generated by all W and V), so \(L_{\phi}\) · R ⊆ \(L(s, \Pi \times \pi)\) · R,
giving \(L(s, \Pi \times \pi) | L_{\phi}\) (in a PID, (a) ⊆ (b) iff b \textbar{} a).

\textbf{Direction 2 (\(L(s, \Pi \times \pi)\) divides \(L_{\phi}\)):} For any nonzero \(\phi' \in K(\Pi)|_{\mathup{GL}_{n}}\), write \(\phi' = \\sum_{i} c_{i} R(g_{i}) \phi\) (finite linear combination
of \(\mathup{GL}_{n}\)-translates, possible by irreducibility). For each \(V\):

\[I(s, \phi', V) = \sum_{i} c_{i} · I(s, R(g_{i})\phi, V)\]

By the equivariance in the key step, I(s, \(R(g_{i})\phi\), V) lies in \(L_{\phi}\) · R
for each \(i\) (since \(\Phi(R(g_{i})\,\phi) = L_{R(g_{i})\phi}\) · \(R = L_{\phi}\) · R). The sum
of elements of \(L_{\phi}\) · R lies in \(L_{\phi}\) · R (it\textquotesingle s a module). Therefore
\(\Phi(\phi')\) ⊆ \(L_{\phi}\) · R for every nonzero \(\phi'\).

Taking the union over all \(\phi'\): the full ideal \(L(s, \Pi \times \pi)\) · \(R = \bigcup_{\phi'} \Phi(\phi')\) ⊆ \(L_{\phi}\) · R. So \(L_{\phi} | L(s, \Pi \times \pi)\).

Combining both directions: \(L_{\phi} = L(s, \Pi \times \pi)\) (up to units in R). ∎

\textbf{Application:} Taking \(\phi = \phi_{Q}\) (the restriction of \(R(u_{Q})W_\mNumber{0}\), which is
nonzero by the Kirillov injectivity lemma), the integrals over \(V\) generate the
full fractional ideal for our fixed \(W_\mNumber{0}\). This closes the universality gap
identified by the reviewer.

\hypertarget{4-nondegeneracy-of-the-pairing}{%
\subsubsection{4. Nondegeneracy of the pairing}\label{4-nondegeneracy-of-the-pairing}}

For a nonzero \(\phi_{Q}\) in the Kirillov model, the Rankin-Selberg pairing:

\[V |\to \int_{N_{n} \backslash \mathup{GL}_{n}(F)} \phi_{Q}(g) V(g) |det g|^{s-\mNumber{1}/\mNumber{2}} dg\]

is nondegenerate as V ranges over \(W(\pi, \psi)\). This is the fundamental
nondegeneracy of the Rankin-Selberg \(\Integral\) (Jacquet-Piatetski-Shapiro-
Shalika 1983, Section 2.7).

More precisely: since \(\phi_{Q}\) is nonzero, the integrals over all \(V \in W(\pi, \psi)\)
generate the full fractional ideal \(L(s, \Pi \times \pi) \ast \mathbb{C}[q_{F}^{s}, q_{F}^{-s}]\).

\hypertarget{5-choosing-w-the-new-vector}{%
\subsubsection{5. Choosing W: the new vector}\label{5-choosing-w-the-new-vector}}

\textbf{Choice:} Let \(W_\mNumber{0}\) be the new vector (essential Whittaker function) of \(\Pi\),
i.e., the vector in \(W(\Pi, \psi^{-\mNumber{1}})\) fixed by the congruence subgroup
\(K_\mNumber{1}(p^{c(\Pi)})\) where c(Pi) is the conductor exponent of \(\Pi\).

\textbf{Properties of \(W_\mNumber{0}\):}

\begin{itemize}
\tightlist
\item
  \(W_\mNumber{0}\) is nonzero (Pi is generic)
\item
  \(W_\mNumber{0}(I_{n+\mNumber{1}}) = \mNumber{1}\) (standard normalization)
\item
  \(R(u_{Q})W_\mNumber{0}\) is nonzero for every \(Q \in F^{\times}\) (as argued in Step 3)
\end{itemize}

\textbf{Universality:} For any generic \(\pi\) of \(\mathup{GL}_{n}(F)\) with conductor ideal q:

\begin{itemize}
\tightlist
\item
  Let Q generate \(q^{-\mNumber{1}}\)
\item
  The function \(\phi_{Q}(g) = W_\mNumber{0}(\operatorname{diag}(g,\mNumber{1})u_{Q})\) is nonzero in the Kirillov model
\item
  By the nondegeneracy (Step 4), there exists \(V \in W(\pi, \psi)\) such that
  \(I(s, W_\mNumber{0}, V)\) is a nonzero element of \(L(s, \Pi \times \pi) \ast \mathbb{C}[q_{F}^{s}, q_{F}^{-s}]\)
\item
  By the algebraic argument (Step 2), V can be chosen to make the \(\Integral\)
  equal to \(c \ast q_{F}^{-ks}\), which is finite and nonzero for all s
\end{itemize}

\hypertarget{6-the-role-of-u_q-in-conductor-matching}{%
\subsubsection{\texorpdfstring{6. The role of \(u_{Q}\) in conductor matching}{6. The role of u\_\{Q\} in conductor matching}}\label{6-the-role-of-u_q-in-conductor-matching}}

The insertion of \(u_{Q} = I \mBridgeOperator{+} Q E_{n,n+\mNumber{1}}\) is essential: it compensates for the
conductor of \(\pi\). Without this twist, for highly ramified \(\pi\), the \(\Integral\)
might degenerate (the new vector of \(\Pi\) would not "see" the ramification of pi).

\textbf{Support analysis via the Casselman-Shalika formula.} By the
Casselman-Shalika formula (Shintani 1976, Casselman-Shalika 1980), the
new vector \(W_\mNumber{0}\) of \(\Pi\) has support contained in

\[N_{n+\mNumber{1}} \ast T^+ \ast K_\mNumber{1}(p^{c(\Pi)})\]

where \(T^+\) is the dominant cone in the diagonal torus. Right translation by
\(u_{Q} = I \mBridgeOperator{+} Q E_{n,n+\mNumber{1}}\) shifts the (n, n+1)-matrix entry by \(Q = \pi_{F}^{-c(\pi)}\).
In the Iwasawa decomposition g = nak, this modifies the diagonal component
at scale Q. Concretely, the support of the twisted function

\[g |\to W_\mNumber{0}(\operatorname{diag}(g,\mNumber{1}) u_{Q})\]

intersects the support of \(K_\mNumber{1}(p^{c(\pi)})\)-fixed vectors in \(W(\pi, \psi)\)
nontrivially, because the Q-shift matches the conductor scale of \(\pi\) exactly:
the new vector of \(\pi\) is supported on \(N_{n} \ast T^+ \ast K_\mNumber{1}(p^{c(\pi)})\) (by the
analogous Casselman-Shalika result for \(\mathup{GL}_{n}\)), and the \(u_{Q}\) twist aligns the
two conductor levels (see Jacquet-Piatetski-Shapiro-Shalika 1981, Section 5,
for the precise support computation in the Rankin-Selberg setting).

This support matching is what makes the \(\Integral\) nondegenerate: it ensures
that the integrand \(W_\mNumber{0}(\operatorname{diag}(g,\mNumber{1})u_{Q})\) * V(g) is not identically zero on
\(N_{n} \backslash \mathup{GL}_{n}(F)\) for appropriate V.

\hypertarget{7-verification-in-special-cases}{%
\subsubsection{7. Verification in special cases}\label{7-verification-in-special-cases}}

\textbf{Both unramified (c(Pi) = c(pi) = 0):} \(u_{Q} = I \mBridgeOperator{+} E_{n,n+\mNumber{1}}\) (Q = 1).
\(W_\mNumber{0}\) is the spherical vector. The \(\Integral\) with \(V_\mNumber{0}\) (spherical for pi)
gives \(L(s, \Pi \times \pi)\) * (correction factor). Choose V to cancel the L-factor.

\textbf{\(\Pi\) unramified, \(\pi\) ramified:} \(u_{Q}\) has \(Q = \pi_{F}^{-c(\pi)}\). The twist
shifts the support of the spherical vector to match the ramification of \(\pi\).
This is the classical "conductor-lowering" mechanism.

\textbf{Both ramified:} The new vector of \(\Pi\) combined with the \(u_{Q}\) twist gives a
function in the Kirillov model whose support is compatible with the
conductor of \(\pi\). Nondegeneracy follows from the general JPSS theory.

\hypertarget{8-summary}{%
\subsubsection{8. Summary}\label{8-summary}}

\begin{enumerate}
\def\labelenumi{\arabic{enumi}.}
\tightlist
\item
  The answer is YES: the new vector \(W_\mNumber{0}\) of \(\Pi\) is a universal test vector
\item
  For any \(\pi\) with conductor q, the twist \(u_{Q}\) (Q generates \(q^{-\mNumber{1}}\)) ensures
  the Rankin-Selberg pairing is nondegenerate
\item
  Nondegeneracy of the Kirillov restriction: \(R(u_{Q})W_\mNumber{0}\) is always nonzero
\item
  The fractional ideal structure of Rankin-Selberg integrals allows
  choosing V to make the \(\Integral\) a monomial \(c \ast q_{F}^{-ks}\)
\item
  This monomial is finite and nonzero for all \(s \in C\)
\end{enumerate}

\hypertarget{references}{%
\subsection{References}\label{references}}

\begin{itemize}
\tightlist
\item
  I. N. Bernstein, A. V. Zelevinsky, "Representations of the group GL(n,F)
  where F is a local non-archimedean field," Russian Math. Surveys 31:3
  (1976), 1-68. {[}Theorem 5.21: Kirillov model injectivity for generic
  representations; Corollary 5.22: density of compactly supported functions{]}
\item
  J. W. Cogdell, "Lectures on L-functions, converse theorems, and
  functoriality for \(\mathup{GL}_{n},"\) Fields Institute Lectures, 2004, Section 3.1.
  {[}Survey of Kirillov model and Whittaker model properties{]}
\item
  H. Jacquet, I. I. Piatetski-Shapiro, J. A. Shalika, "Rankin-Selberg
  convolutions," Amer. J. Math. 105 (1983), 367-464, Section 2.7.
  {[}Nondegeneracy of the Rankin-Selberg pairing; fractional ideal structure{]}
\item
  H. Jacquet, I. I. Piatetski-Shapiro, J. A. Shalika, "Conducteur des
  représentations du groupe linéaire," Math. Ann. 256 (1981), 199-214,
  Section 5. {[}Support of new vectors and conductor matching{]}
\item
  W. Casselman, J. Shalika, "The unramified principal series of p-adic
  groups II: the Whittaker function," Compositio Math. 41 (1980), 207-231.
  {[}Casselman-Shalika formula for Whittaker function support{]}
\end{itemize}

\hypertarget{key-references-from-futon6-corpus}{%
\subsection{Key References from futon6 corpus}\label{key-references-from-futon6-corpus}}

\begin{itemize}
\tightlist
\item
  PlanetMath: "representation theory" -\/- admissible representations
\item
  PlanetMath: "L-function" -\/- L-factors and analytic properties
\item
  PlanetMath: "locally compact group" -\/- p-adic groups
\end{itemize}
