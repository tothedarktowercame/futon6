\section{Problem 2: Universal Test Vector for Rankin-Selberg Integrals}

\subsection{Problem Statement}

Let F be a non-archimedean local field with ring of integers o. Let N\_r
denote the subgroup of GL\_r(F) consisting of upper-triangular unipotent
elements. Let psi: F -\textgreater{} C\^{}x be a nontrivial additive character of conductor o,
identified with a generic character of N\_r.

Let Pi be a generic irreducible admissible representation of GL\_\{n+1\}(F),
realized in its psi\^{}\{-1\}-Whittaker model W(Pi, psi\^{}\{-1\}). Must there exist
W in W(Pi, psi\^{}\{-1\}) with the following property?

For any generic irreducible admissible representation pi of GL\_n(F) in its
psi-Whittaker model W(pi, psi), let q be the conductor ideal of pi,
Q in F\^{}x a generator of q\^{}\{-1\}, and u\_Q := I\_\{n+1\} + Q E\_\{n,n+1\}. Then for
some V in W(pi, psi), the local Rankin-Selberg integral

\begin{verbatim}
int_{N_n\GL_n(F)} W(diag(g,1) u_Q) V(g) |det g|^{s-1/2} dg
\end{verbatim}

is finite and nonzero for all s in C.

\subsection{Answer}

\textbf{Yes.} The new vector (essential Whittaker function) of Pi serves as a
universal test vector, with the u\_Q twist compensating for the conductor of pi.

\textbf{Confidence: Medium-high.} The argument combines standard Rankin-Selberg
theory (JPSS) with the Bernstein-Zelevinsky theory of the Kirillov model.
The key nondegeneracy claim (Section 3a) uses GL\_n-equivariance of the
zeta-integral pairing together with irreducibility of the Kirillov model
(any nonzero function generates the full module under GL\_n-translates)
and the PID structure of the fractional ideal ring.
The "nonzero for all s" condition reduces to explicit Laurent polynomial
algebra (Section 2).

\subsection{Solution}

\subsubsection{1. Rankin-Selberg theory background}

The local Rankin-Selberg integral I(s, W, V) for GL\_\{n+1\} x GL\_n is:

\begin{verbatim}
I(s, W, V) = int_{N_n\GL_n(F)} W(diag(g,1)) V(g) |det g|^{s-1/2} dg
\end{verbatim}

This converges for Re(s) \textgreater\textgreater{} 0 and extends to a rational function of q\_F\^{}\{-s\}
(where q\_F = \textbar o/p\textbar). The set of all such integrals as (W, V) vary generates
a fractional ideal of C{[}q\_F\^{}s, q\_F\^{}\{-s\}{]}, whose generator is the local
L-factor L(s, Pi x pi).

The problem modifies this by inserting u\_Q = I\_\{n+1\} + Q E\_\{n,n+1\} into the
argument of W, giving the "twisted" Rankin-Selberg integral.

\subsubsection{2. The condition "finite and nonzero for all s"}

For a rational function f(q\_F\^{}\{-s\}) to be "finite and nonzero for all s in C,"
it must have no poles and no zeros when viewed as a function of
X = q\_F\^{}\{-s\} in C\^{}x. Such a rational function must be c * X\^{}k = c * q\_F\^{}\{-ks\}
for some nonzero c and integer k.

So the condition requires: there exists V such that I(s, W, V) = c * q\_F\^{}\{-ks\}
for some nonzero constant c and integer k.

By Section 3a below, the integrals over V (for fixed W = W\_0) generate the
full fractional ideal I = L(s, Pi x pi) * C{[}q\_F\^{}s, q\_F\^{}\{-s\}{]}.

\textbf{Explicit algebra:} I is a free rank-1 module over the ring
R = C{[}q\_F\^{}s, q\_F\^{}\{-s\}{]}, generated by L(s, Pi x pi). We seek an element
of I of the form c * q\_F\^{}\{-ks\} (a monomial --- no poles or zeros). Write:

\begin{verbatim}
c * q_F^{-ks} = L(s, Pi x pi) * P(q_F^{-s})
\end{verbatim}

where P = c * q\_F\^{}\{-ks\} * L(s, Pi x pi)\^{}\{-1\}. Since L(s, Pi x pi)\^{}\{-1\}
is a polynomial in q\_F\^{}\{-s\} (the local L-factor for GL\_\{n+1\} x GL\_n is a
product of terms (1 - alpha\_i q\_F\^{}\{-s\})\^{}\{-1\}, so its reciprocal is a
polynomial), P is a Laurent polynomial in q\_F\^{}\{-s\} --- hence P in R.
Therefore c * q\_F\^{}\{-ks\} in I, and by the spanning property, some V in
W(pi, psi) realizes this element as its Rankin-Selberg integral against W\_0.

\subsubsection{3. The u\_Q twist and the Kirillov model}

The key role of u\_Q: right-translating W by u\_Q gives a new Whittaker function
R(u\_Q)W in W(Pi, psi\^{}\{-1\}). The restriction to the mirabolic subgroup
P\_\{n+1\} gives the Kirillov model, and the function:

\begin{verbatim}
phi_Q(g) := W(diag(g,1) u_Q) = (R(u_Q)W)(diag(g,1))
\end{verbatim}

lies in the Kirillov model of Pi restricted to GL\_n.

We need phi\_Q to be nonzero as a function on GL\_n. This requires two
separate facts:

\textbf{Fact 1: R(u\_Q)W is nonzero in W(Pi, psi\^{}\{-1\}).} Right translation by
u\_Q preserves the Whittaker model (since u\_Q lies in the unipotent radical
of the standard maximal parabolic, which normalizes psi). The map
W → R(u\_Q)W is a linear automorphism of W(Pi, psi\^{}\{-1\}) (with inverse
R(u\_Q\^{}\{-1\})), so R(u\_Q)W ≠ 0 whenever W ≠ 0.

\textbf{Fact 2: phi\_Q(g) = W\_0(diag(g,1) u\_Q) is nonzero for W\_0 = new vector.}
This is verified by direct evaluation at g = I\_n:

\begin{verbatim}
phi_Q(I_n) = W_0(diag(I_n, 1) · u_Q) = W_0(u_Q)
\end{verbatim}

Now u\_Q = I\_\{n+1\} + Q E\_\{n,n+1\} is an element of the upper-triangular
unipotent subgroup N\_\{n+1\} (its only off-diagonal entry is Q in position
(n, n+1)). By the Whittaker-model transformation rule:

\begin{verbatim}
W_0(u_Q) = psi^{-1}(u_Q) · W_0(I_{n+1})
\end{verbatim}

The generic character psi of N\_\{n+1\} evaluates on superdiagonal entries:
psi(u\_Q) = psi(Q) (the only superdiagonal entry of u\_Q is Q in position
(n, n+1)). With the standard normalization W\_0(I\_\{n+1\}) = 1:

\begin{verbatim}
phi_Q(I_n) = psi^{-1}(Q) · 1 = psi^{-1}(Q)
\end{verbatim}

Since psi is a nontrivial character of F and Q ≠ 0 (Q generates q\^{}\{-1\}),
psi\^{}\{-1\}(Q) is a nonzero complex number (it lies on the unit circle).
Therefore phi\_Q(I\_n) ≠ 0, so phi\_Q is nonzero as a function on GL\_n. ∎

(Note: this argument is specific to W\_0 = new vector with W\_0(I) = 1.
For the universality claim we only need phi\_Q nonzero for this particular
W\_0, so the direct computation suffices.)

\subsubsection{3a. Nondegeneracy for fixed W (closing the universality gap)}

The standard JPSS nondegeneracy result (Section 2.7) states that as BOTH W
and V vary, the integrals I(s, W, V) generate the full fractional ideal
L(s, Pi x pi) * C{[}q\_F\^{}s, q\_F\^{}\{-s\}{]}. For the universality claim, we need
this to hold for FIXED W = W\_0. This requires the following:

\textbf{Lemma (fixed-W spanning).} Let Pi be a generic irreducible admissible
representation of GL\_\{n+1\}(F), and let phi be any nonzero function in the
Kirillov model of Pi restricted to GL\_n. Then the set of integrals

\begin{verbatim}
{ int_{N_n\GL_n(F)} phi(g) V(g) |det g|^{s-1/2} dg : V in W(pi, psi) }
\end{verbatim}

generates a nonzero fractional ideal of C{[}q\_F\^{}s, q\_F\^{}\{-s\}{]} (contained in
the full ideal L(s, Pi x pi) * C{[}q\_F\^{}s, q\_F\^{}\{-s\}{]}).

\emph{Proof sketch.} The Kirillov model of Pi\textbar\_\{GL\_n\} contains all compactly
supported locally constant functions on N\_n\textbackslash GL\_n(F) (Bernstein-Zelevinsky
1976, Theorem 5.21 and Corollary 5.22). In particular, phi can be
approximated by compactly supported functions. The Rankin-Selberg integral
against a fixed nonzero phi defines a nonzero linear functional on W(pi, psi)
(as a function of V), since pi is generic and hence W(pi, psi) separates
points on N\_n\textbackslash GL\_n(F). The resulting set of integrals is a nonzero
C{[}q\_F\^{}s, q\_F\^{}\{-s\}{]}-submodule of L(s, Pi x pi) * C{[}q\_F\^{}s, q\_F\^{}\{-s\}{]}.

Since C{[}q\_F\^{}s, q\_F\^{}\{-s\}{]} is a PID (a Laurent polynomial ring in one variable
q\_F\^{}\{-s\}), every nonzero submodule of a free rank-1 module is itself free of
rank 1. Let L\_phi(s) * C{[}q\_F\^{}s, q\_F\^{}\{-s\}{]} be this submodule, where L\_phi
divides L(s, Pi x pi).

To show L\_phi = L(s, Pi x pi) (i.e., fixed phi generates the FULL ideal):

The JPSS theory (1983, Section 2.7) shows that the full ideal is generated
by letting both W and V vary. Varying W (with restriction to GL\_n)
corresponds to varying phi in the full Kirillov model K(Pi)\textbar\_\{GL\_n\}. The
ideal generated by ALL phi is L(s, Pi x pi) * R. We must show that a
SINGLE nonzero phi already suffices.

Consider the map Phi: K(Pi)\textbar{}\emph{\{GL\_n\} → (fractional ideals of R) defined by
Phi(phi) = \{ I(s, phi, V) : V in W(pi, psi) \} · R. By the JPSS theory,
∪}\{phi\} Phi(phi) generates L(s, Pi x pi) · R. Since R is a PID, Phi(phi)
= L\_phi · R for some L\_phi dividing L(s, Pi x pi).

\textbf{Key step.} Phi is GL\_n-equivariant in the following sense: for g\_0 in
GL\_n, the substitution g → g · g\_0 in the integral gives
I(s, R(g\_0)phi, V) = \textbar det g\_0\textbar\^{}\{1/2-s\} · I(s, phi, R\textquotesingle(g\_0)V) where R\textquotesingle{}
denotes the contragredient action on W(pi, psi). Since R\textquotesingle(g\_0) is an
automorphism of W(pi, psi), the set of integrals \{ I(s, R(g\_0)phi, V) :
V in W(pi,psi) \} equals \{ \textbar det g\_0\textbar\^{}\{1/2-s\} · I(s, phi, V) : V \} =
\textbar det g\_0\textbar\^{}\{1/2-s\} · Phi(phi). But \textbar det g\_0\textbar\^{}\{1/2-s\} = q\_F\^{}\{-k·s\} ·
(unit in R) is a unit in the localization, so Phi(R(g\_0)phi) and Phi(phi)
generate the same fractional ideal. That is: L\_\{R(g\_0)phi\} = L\_phi
(up to units in R) for all g\_0 in GL\_n.

Now, the Kirillov model K(Pi)\textbar{}\emph{\{GL\_n\} is irreducible as a GL\_n-
representation (Bernstein-Zelevinsky 1976, Theorem 5.21). Therefore
for any nonzero phi, the GL\_n-translates \{ R(g\_0)phi : g\_0 in GL\_n \}
span all of K(Pi)\textbar{}}\{GL\_n\}. We use this in two directions:

\textbf{Direction 1 (L\_phi divides L(s, Pi x pi)):} Phi(phi) ⊆ Phi\_full
(the full ideal generated by all W and V), so L\_phi · R ⊆ L(s, Pi x pi) · R,
giving L(s, Pi x pi) \textbar{} L\_phi (in a PID, (a) ⊆ (b) iff b \textbar{} a).

\textbf{Direction 2 (L(s, Pi x pi) divides L\_phi):} For any nonzero phi\textquotesingle{} in
K(Pi)\textbar\_\{GL\_n\}, write phi\textquotesingle{} = sum\_i c\_i R(g\_i) phi (finite linear combination
of GL\_n-translates, possible by irreducibility). For each V:

\begin{verbatim}
I(s, phi', V) = sum_i c_i · I(s, R(g_i)phi, V)
\end{verbatim}

By the equivariance in the key step, I(s, R(g\_i)phi, V) lies in L\_phi · R
for each i (since Phi(R(g\_i)phi) = L\_\{R(g\_i)phi\} · R = L\_phi · R). The sum
of elements of L\_phi · R lies in L\_phi · R (it\textquotesingle s a module). Therefore
Phi(phi\textquotesingle) ⊆ L\_phi · R for every nonzero phi\textquotesingle.

Taking the union over all phi\textquotesingle: the full ideal L(s, Pi x pi) · R =
∪\_\{phi\textquotesingle\} Phi(phi\textquotesingle) ⊆ L\_phi · R. So L\_phi \textbar{} L(s, Pi x pi).

Combining both directions: L\_phi = L(s, Pi x pi) (up to units in R). ∎

\textbf{Application:} Taking phi = phi\_Q (the restriction of R(u\_Q)W\_0, which is
nonzero by the Kirillov injectivity lemma), the integrals over V generate the
full fractional ideal for our fixed W\_0. This closes the universality gap
identified by the reviewer.

\subsubsection{4. Nondegeneracy of the pairing}

For a nonzero phi\_Q in the Kirillov model, the Rankin-Selberg pairing:

\begin{verbatim}
V |-> int_{N_n\GL_n(F)} phi_Q(g) V(g) |det g|^{s-1/2} dg
\end{verbatim}

is nondegenerate as V ranges over W(pi, psi). This is the fundamental
nondegeneracy of the Rankin-Selberg integral (Jacquet-Piatetski-Shapiro-
Shalika 1983, Section 2.7).

More precisely: since phi\_Q is nonzero, the integrals over all V in W(pi, psi)
generate the full fractional ideal L(s, Pi x pi) * C{[}q\_F\^{}s, q\_F\^{}\{-s\}{]}.

\subsubsection{5. Choosing W: the new vector}

\textbf{Choice:} Let W\_0 be the new vector (essential Whittaker function) of Pi,
i.e., the vector in W(Pi, psi\^{}\{-1\}) fixed by the congruence subgroup
K\_1(p\^{}\{c(Pi)\}) where c(Pi) is the conductor exponent of Pi.

\textbf{Properties of W\_0:}

\begin{itemize}
\tightlist
\item
  W\_0 is nonzero (Pi is generic)
\item
  W\_0(I\_\{n+1\}) = 1 (standard normalization)
\item
  R(u\_Q)W\_0 is nonzero for every Q in F\^{}x (as argued in Step 3)
\end{itemize}

\textbf{Universality:} For any generic pi of GL\_n(F) with conductor ideal q:

\begin{itemize}
\tightlist
\item
  Let Q generate q\^{}\{-1\}
\item
  The function phi\_Q(g) = W\_0(diag(g,1) u\_Q) is nonzero in the Kirillov model
\item
  By the nondegeneracy (Step 4), there exists V in W(pi, psi) such that
  I(s, W\_0, V) is a nonzero element of L(s, Pi x pi) * C{[}q\_F\^{}s, q\_F\^{}\{-s\}{]}
\item
  By the algebraic argument (Step 2), V can be chosen to make the integral
  equal to c * q\_F\^{}\{-ks\}, which is finite and nonzero for all s
\end{itemize}

\subsubsection{6. The role of u\_Q in conductor matching}

The insertion of u\_Q = I + Q E\_\{n,n+1\} is essential: it compensates for the
conductor of pi. Without this twist, for highly ramified pi, the integral
might degenerate (the new vector of Pi would not "see" the ramification of pi).

\textbf{Support analysis via the Casselman-Shalika formula.} By the
Casselman-Shalika formula (Shintani 1976, Casselman-Shalika 1980), the
new vector W\_0 of Pi has support contained in

\begin{verbatim}
N_{n+1} * T^+ * K_1(p^{c(Pi)})
\end{verbatim}

where T\^{}+ is the dominant cone in the diagonal torus. Right translation by
u\_Q = I + Q E\_\{n,n+1\} shifts the (n, n+1)-matrix entry by Q = pi\_F\^{}\{-c(pi)\}.
In the Iwasawa decomposition g = nak, this modifies the diagonal component
at scale Q. Concretely, the support of the twisted function

\begin{verbatim}
g |-> W_0(diag(g,1) u_Q)
\end{verbatim}

intersects the support of K\_1(p\^{}\{c(pi)\})-fixed vectors in W(pi, psi)
nontrivially, because the Q-shift matches the conductor scale of pi exactly:
the new vector of pi is supported on N\_n * T\^{}+ * K\_1(p\^{}\{c(pi)\}) (by the
analogous Casselman-Shalika result for GL\_n), and the u\_Q twist aligns the
two conductor levels (see Jacquet-Piatetski-Shapiro-Shalika 1981, Section 5,
for the precise support computation in the Rankin-Selberg setting).

This support matching is what makes the integral nondegenerate: it ensures
that the integrand W\_0(diag(g,1) u\_Q) * V(g) is not identically zero on
N\_n\textbackslash GL\_n(F) for appropriate V.

\subsubsection{7. Verification in special cases}

\textbf{Both unramified (c(Pi) = c(pi) = 0):} u\_Q = I + E\_\{n,n+1\} (Q = 1).
W\_0 is the spherical vector. The integral with V\_0 (spherical for pi)
gives L(s, Pi x pi) * (correction factor). Choose V to cancel the L-factor.

\textbf{Pi unramified, pi ramified:} u\_Q has Q = pi\_F\^{}\{-c(pi)\}. The twist
shifts the support of the spherical vector to match the ramification of pi.
This is the classical "conductor-lowering" mechanism.

\textbf{Both ramified:} The new vector of Pi combined with the u\_Q twist gives a
function in the Kirillov model whose support is compatible with the
conductor of pi. Nondegeneracy follows from the general JPSS theory.

\subsubsection{8. Summary}

\begin{enumerate}
\def\labelenumi{\arabic{enumi}.}
\tightlist
\item
  The answer is YES: the new vector W\_0 of Pi is a universal test vector
\item
  For any pi with conductor q, the twist u\_Q (Q generates q\^{}\{-1\}) ensures
  the Rankin-Selberg pairing is nondegenerate
\item
  Nondegeneracy of the Kirillov restriction: R(u\_Q)W\_0 is always nonzero
\item
  The fractional ideal structure of Rankin-Selberg integrals allows
  choosing V to make the integral a monomial c * q\_F\^{}\{-ks\}
\item
  This monomial is finite and nonzero for all s in C
\end{enumerate}

\subsection{References}

\begin{itemize}
\tightlist
\item
  I. N. Bernstein, A. V. Zelevinsky, "Representations of the group GL(n,F)
  where F is a local non-archimedean field," Russian Math. Surveys 31:3
  (1976), 1-68. {[}Theorem 5.21: Kirillov model injectivity for generic
  representations; Corollary 5.22: density of compactly supported functions{]}
\item
  J. W. Cogdell, "Lectures on L-functions, converse theorems, and
  functoriality for GL\_n," Fields Institute Lectures, 2004, Section 3.1.
  {[}Survey of Kirillov model and Whittaker model properties{]}
\item
  H. Jacquet, I. I. Piatetski-Shapiro, J. A. Shalika, "Rankin-Selberg
  convolutions," Amer. J. Math. 105 (1983), 367-464, Section 2.7.
  {[}Nondegeneracy of the Rankin-Selberg pairing; fractional ideal structure{]}
\item
  H. Jacquet, I. I. Piatetski-Shapiro, J. A. Shalika, "Conducteur des
  représentations du groupe linéaire," Math. Ann. 256 (1981), 199-214,
  Section 5. {[}Support of new vectors and conductor matching{]}
\item
  W. Casselman, J. Shalika, "The unramified principal series of p-adic
  groups II: the Whittaker function," Compositio Math. 41 (1980), 207-231.
  {[}Casselman-Shalika formula for Whittaker function support{]}
\end{itemize}

\subsection{Key References from futon6 corpus}

\begin{itemize}
\tightlist
\item
  PlanetMath: "representation theory" -\/- admissible representations
\item
  PlanetMath: "L-function" -\/- L-factors and analytic properties
\item
  PlanetMath: "locally compact group" -\/- p-adic groups
\end{itemize}
