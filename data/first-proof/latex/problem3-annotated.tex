\section{Problem 3: Markov Chain with ASEP Polynomial Stationary Distribution}

\subsection*{Problem statement}
Construct a nontrivial Markov chain on permutations of a partition $\lambda$ whose stationary distribution is the normalized interpolation ASEP polynomial weight at $q=1$.

\subsection*{Answer}
Yes: the inhomogeneous multispecies $t$-PushTASEP yields the required stationary law.

\subsection*{Annotated proof sketch}
\begin{enumerate}[leftmargin=1.5em]
\item Define a finite-state continuous-time Markov chain with explicit local transition rates that depend only on $(x,t)$ and species order (not on polynomial values).
\item Invoke the stationary measure theorem identifying the stationary weights with ASEP/Macdonald objects at $q=1$\footnote{Ayyer--Martin--Williams, arXiv: \url{https://arxiv.org/abs/2403.10485}.}.
\item Use irreducibility (vacancy transport and positive-rate swaps) to conclude uniqueness of the stationary distribution on the finite state space\footnote{Markov-chain basics: \url{https://planetmath.org/markovchain}.}.
\end{enumerate}

\subsection*{Selected citations}
\begin{itemize}[leftmargin=1.5em]
\item Inhomogeneous multispecies $t$-PushTASEP: \url{https://arxiv.org/abs/2403.10485}
\item Multiline queues to Macdonald polynomials: \url{https://doi.org/10.1353/ajm.2022.0007}
\item Macdonald polynomial definition page: \url{https://planetmath.org/macdonaldpolynomials}
\end{itemize}

\subsection*{Background and prerequisites}
\textbf{Field:} Markov Chains / Algebraic Combinatorics / Symmetric Functions.\\
\textbf{What you need:} Basic Markov chain theory (irreducibility, uniqueness of stationary distributions) and some familiarity with symmetric functions, particularly Macdonald polynomials and their specializations.\\
\textbf{Way in:} Levin--Peres--Wilmer, \emph{Markov Chains and Mixing Times} (AMS, 2nd ed.)\ for stochastic foundations; Macdonald, \emph{Symmetric Functions and Hall Polynomials} (OUP, 2nd ed.)\ for the algebraic-combinatorial side.


\clearpage
