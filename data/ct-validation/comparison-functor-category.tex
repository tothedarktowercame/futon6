\documentclass[10pt,landscape]{article}
\usepackage[margin=1.5cm,landscape]{geometry}
\usepackage{xcolor}
\usepackage{multicol}
\usepackage{enumitem}
\usepackage{tikz}
\usepackage{tcolorbox}
\usepackage{booktabs}
\usepackage{tabularx}
\usepackage{amssymb,amsmath}
\usepackage{parskip}

% Wiring diagram colors
\definecolor{comp}{HTML}{2563EB}      % components — blue
\definecolor{compbg}{HTML}{DBEAFE}
\definecolor{wire-adv}{HTML}{DC2626}  % wire/adversative — red
\definecolor{wire-caus}{HTML}{EA580C} % wire/causal — orange
\definecolor{wire-cons}{HTML}{9333EA} % wire/consequential — purple
\definecolor{wire-clar}{HTML}{0891B2} % wire/clarifying — teal
\definecolor{port}{HTML}{16A34A}      % ports — green
\definecolor{portbg}{HTML}{DCFCE7}
\definecolor{label}{HTML}{CA8A04}     % wire labels — amber
\definecolor{labelbg}{HTML}{FEF9C3}
\definecolor{dimtext}{HTML}{6B7280}   % dimmed text

% Annotation commands
\newcommand{\comp}[1]{\colorbox{compbg}{\textcolor{comp}{\textsf{\textbf{#1}}}}}
\newcommand{\wA}[1]{\textcolor{wire-adv}{\textbf{#1}}}      % adversative
\newcommand{\wC}[1]{\textcolor{wire-caus}{\textbf{#1}}}      % causal
\newcommand{\wQ}[1]{\textcolor{wire-cons}{\textbf{#1}}}      % consequential
\newcommand{\wL}[1]{\textcolor{wire-clar}{\textbf{#1}}}      % clarifying
\newcommand{\prt}[1]{\colorbox{portbg}{\textcolor{port}{\textsf{#1}}}}  % port
\newcommand{\lbl}[1]{\colorbox{labelbg}{\textcolor{label}{\scriptsize #1}}} % label
\newcommand{\mathsnip}[1]{\textcolor{dimtext}{\texttt{#1}}}  % math (dimmed)

\pagestyle{empty}

\begin{document}

\begin{center}
{\LARGE\bfseries Wiring Diagram Comparison: Functor Category}\\[4pt]
{\large PlanetMath vs nLab --- same concept, different wiring}\\[2pt]
{\small Generated 2026-02-10 from unified metatheory v3.0}
\end{center}

\vspace{6pt}

% Legend
\begin{tcolorbox}[colback=gray!5, colframe=gray!40, title={\small\bfseries Legend: Wiring Diagram Elements}, fonttitle=\sffamily]
\begin{tabularx}{\textwidth}{XXXXXXX}
\comp{component} &
\wA{adversative} &
\wC{causal} &
\wQ{consequential} &
\wL{clarifying} &
\prt{port/anaphora} &
\lbl{wire label} \\
\scriptsize scope binding &
\scriptsize but, however &
\scriptsize because, since &
\scriptsize therefore, hence &
\scriptsize that is, namely &
\scriptsize the above, similarly &
\scriptsize strategy, explain \\
\end{tabularx}
\end{tcolorbox}

\vspace{4pt}

\begin{multicols}{2}

% ============================================================
% LEFT COLUMN: PlanetMath
% ============================================================
\begin{tcolorbox}[colback=blue!2, colframe=comp, title={\sffamily\bfseries PlanetMath: Functor Category}, fonttitle=\sffamily\small]
{\small\sffamily 5,949 chars $\cdot$ 19 components $\cdot$ 10 wires $\cdot$ 2 ports $\cdot$ ratio 2:1}
\end{tcolorbox}

\small

\comp{Let $\mathcal{C},\mathcal{D}$ be categories}.
\comp{Consider the class} of all (covariant) functors from $\mathcal{C}$ to $\mathcal{D}$. \comp{Consider} the following sets and operations
on the class: \comp{for each $S:\mathcal{C}\to\mathcal{D}$}, define a set $\hom(S,T)$:
\comp{$\tau \in \hom(S,T)$} iff $\tau$ is a natural transformation from $S$ to $T$.
For \comp{every $\tau \in \hom(S,T)$} and \comp{every $\eta \in \hom(T,U)$}, define
$\eta\circ\tau$ to be the natural transformation $\eta\bullet\tau: S\Rightarrow U$
\wC{given} by $(\eta\bullet\tau)_A := \eta_A\circ\tau_A$
\comp{for every $A \in \mathcal{C}$}. \comp{$1_S \in \hom(S,S)$} is the identity
natural transformation.

It is not hard to see that $\hom(S,T)$ is indeed a set (\wC{since} $\mathcal{D}$ is a
category, $\hom(T(A),S(A))$ is a set \comp{for every $A \in \mathcal{C}$},
\wQ{so that} $\hom(S,T)$ is a ``subset'' of the Cartesian product\ldots).

The class, together with the hom sets and composition, has all the
properties of a category, and is called the \emph{functor category} of
$\mathcal{C}$ and $\mathcal{D}$, and is denoted by $\mathcal{D}^{\mathcal{C}}$.

If the categories are small, \wQ{it follows} from the discussion above that
the functor category is also a category in the ``traditional'' sense.
\wA{However}, \comp{for any $U$} in a universe $\mathfrak{U}$, \mathsnip{[\ldots]}
\wQ{so that} the functor category of two $U$-small categories is again
$U$-small.

\prt{This means} that using Grothendieck's axiom,
the functor category is a ``legitimate'' category, \wC{since}
\comp{$S \in \mathfrak{U}$} for \comp{every} $S$.

\wC{Because} the objects of a functor category are functors, a morphism between
two objects in a functor category is a natural transformation between two functors.
\wQ{In fact}, a functor category is a special case of \prt{the same}
construction\ldots

\vspace{6pt}
\begin{tcolorbox}[colback=blue!3, colframe=comp, boxrule=0.3pt]
\scriptsize\sffamily
\textbf{Component signature:}
constrain/such-that~\textbf{7},
quant/universal~\textbf{7},
bind/let~\textbf{2},
assume/consider~\textbf{2},
assume/explicit~\textbf{1}\\[2pt]
\textbf{Wire signature:}
\textcolor{wire-caus}{causal~\textbf{5}},
\textcolor{wire-cons}{consequential~\textbf{4}},
\textcolor{wire-adv}{adversative~\textbf{1}}
\end{tcolorbox}

% ============================================================
% RIGHT COLUMN: nLab
% ============================================================
\columnbreak

\begin{tcolorbox}[colback=green!2, colframe=port, title={\sffamily\bfseries nLab: Functor Category}, fonttitle=\sffamily\small]
{\small\sffamily 9,951 chars $\cdot$ 8 components $\cdot$ 22 wires $\cdot$ 3 ports $\cdot$ ratio 1:3}
\end{tcolorbox}

\small

\comp{Let $\mathcal{C}$ be} an accessible category that is \emph{not}
essentially small. \wQ{In fact} $[\mathcal{C}, Set]$ is equivalent to the
full subcategory of \mathsnip{[\ldots]}.

\lbl{The idea is} that we can form a new category whose objects are
functors and whose morphisms are natural transformations.

A functor category is also called a \wL{``that is''} a category of diagrams
or a category of presheaves. \prt{Similarly}, one can consider enriched
functor categories.

The category $[I,C]$ of functors from a small category $I$ to $C$, with
natural transformations between them, is \wL{more precisely} the category
where: \comp{for any $a:A$}, a morphism $\gamma: F \to G$ consists of
components \comp{$\gamma_a$}: $F(a) \to G(a)$ \comp{for each $a$},
subject to naturality: \comp{for any $a,b:A$} and $f:a\to b$, the square commutes.

\wA{But} this definition requires $I$ to be small.
\wA{However}, there are ways around this restriction.

\wC{Because} functor categories are central to topos theory, there is a
well-developed theory of their properties.
\lbl{It is well known} that $[C^{op}, Set]$ is a topos
\comp{for each} small $C$. \wQ{Hence} it has all limits and colimits.

\wA{But} size issues intervene when $C$ is large.
\wA{Nevertheless}, one can work with accessible categories.

\wQ{It follows} that the Yoneda embedding $y: C \hookrightarrow [C^{op}, Set]$
is fully faithful. \wL{That is}, $C$ embeds as a full subcategory of its
presheaf category.

\prt{The above} construction generalises to enriched categories.
The enriched functor category $[C,D]_V$ has \prt{the same} objects
\wA{but} morphisms are $V$-natural transformations.

\wC{Since} enriched functor categories satisfy \wL{that is} the
\wQ{so that} one obtains a closed monoidal structure.

\vspace{6pt}
\begin{tcolorbox}[colback=green!3, colframe=port, boxrule=0.3pt]
\scriptsize\sffamily
\textbf{Component signature:}
quant/universal~\textbf{6},
bind/let~\textbf{1},
assume/consider~\textbf{1}\\[2pt]
\textbf{Wire signature:}
\textcolor{wire-caus}{causal~\textbf{7}},
\textcolor{wire-adv}{adversative~\textbf{6}},
\textcolor{wire-clar}{clarifying~\textbf{5}},
\textcolor{wire-cons}{consequential~\textbf{4}}
\end{tcolorbox}

\end{multicols}

\vspace{4pt}

% ============================================================
% BOTTOM: Comparison table
% ============================================================
\begin{tcolorbox}[colback=white, colframe=black!60, title={\sffamily\bfseries Rewiring Analysis}, fonttitle=\sffamily]
\begin{multicols}{2}

\small
\begin{tabular}{lrrrl}
\toprule
\textbf{Element} & \textbf{PM} & \textbf{nLab} & \textbf{$\Delta$} & \textbf{Interpretation} \\
\midrule
\multicolumn{5}{l}{\textit{Components (scope bindings)}} \\
constrain/such-that & 7 & 0 & $-7$ & PM: explicit precision \\
quant/universal & 7 & 6 & $-1$ & $\approx$ shared \\
bind/let & 2 & 1 & $-1$ & $\approx$ shared \\
\midrule
\multicolumn{5}{l}{\textit{Wires (connectives)}} \\
wire/adversative & 1 & 6 & $+5$ & nLab: nuancing \\
wire/clarifying & 0 & 5 & $+5$ & nLab: reformulation \\
wire/causal & 5 & 7 & $+2$ & nLab: more justification \\
wire/consequential & 4 & 4 & $0$ & shared derivation \\
\midrule
\multicolumn{5}{l}{\textit{Density (per 1000 chars)}} \\
components & 3.19 & 0.80 & & PM 4$\times$ denser \\
wires & 1.68 & 2.21 & & nLab 1.3$\times$ denser \\
\bottomrule
\end{tabular}

\columnbreak

\textbf{The rewiring as exposition morphism:}

\smallskip
To transform PM $\to$ nLab:
\begin{itemize}[nosep,leftmargin=1em]
\item \textcolor{comp}{Remove} 7 precision constraints (such-that)
\item \textcolor{wire-adv}{Add} 5 adversative wires (but, however)
\item \textcolor{wire-clar}{Add} 5 clarifying wires (that is, namely)
\item \textcolor{label}{Add} epistemic/pedagogical labels
\end{itemize}

\smallskip
\textbf{Component:Wire ratio encodes style:}
\begin{itemize}[nosep,leftmargin=1em]
\item PM ratio 2:1 $\to$ \emph{reference} (define precisely)
\item nLab ratio 1:3 $\to$ \emph{tutorial} (explain connections)
\item The same mathematical content; different wiring
\end{itemize}

\smallskip
{\small\itshape The rewiring is a morphism in a category of expositions.
It preserves the mathematical content (shared components)
while transforming the pedagogical structure (wires).}

\end{multicols}
\end{tcolorbox}

\end{document}
